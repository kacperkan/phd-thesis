\chapter{Introduction}
\label{chap:introduction}

                   With the advent of deep learning, research have been
                   exploring varying ways to apply it to computer graphics.
One of the most recent and promising approaches is neural rendering.
Neural rendering is a field that combines deep learning and computer graphics
to generate realistic images of 3D scenes.
The neural radiance field (NeRF) is a popular neural rendering technique that
represents a 3D scene as a continuous function that maps 3D coordinates to
radiance values.
NeRF has shown impressive results in generating photorealistic images of 3D
scenes.
However, NeRF has limitations in terms of memory and computational
requirements, which makes it difficult to scale to large scenes.

                   To alleviate the problem, \textcite{kerbl20233d} proposed a
                   new technique---3D Gaussian Splatting (3DGS).
3DGS is a neural rendering technique that represents a 3D scene as a set of 3D Gaussian that are splatted to an image space using algorithm proposed by~\textcite{zwicker2001ewa}.
 In contrast to NeRF, 3DGS is more memory efficient and can be used to render
 large scenes.
It can also render scenes with millions of points in real-time on a single
GPU.

                   In this thesis, we focus on those two milestone techniques
                   in neural rendering and address their fundamental
                   problem---lack of controllability.

\section{Motivation and challenges}

  NeRF and 3DGS are both impressive techniques that can generate realistic
  images.
  However, a single scene representation needs to be trained on a high-end GPU
  for hours or even days just to render a novel view at the inference time.
  However, any type of controllability is difficult to achieve with those
  models.
  That includes changing the lighting conditions, subject's attributes or even
  the scene itself.
  We see imbuing those models with controllability as a an important step
  towards making them more useful in practice.
  Our proposed models are designed to address this issue.

  One may ask why the controllability is a feat sought after to be researched.
  We see the inspiration in how human artists work.
  Imagine an artist working on 3D game where they need an asset, like a 3D
  mesh, to be created.
  Such a mesh takes much effort since it includes modeling, creating a UV map
  which can then be textured.
  After the process is finish, the artist's supervisor may task him to change
  the model to some extent which requires the artist to redo all the effort
  again.
  Such a process is not limited to 3D assets as meshes and could be applied to
  3DGS or NeRF.
  However, 3DGS and NeRFs are volumetric in nature.
  Our exploited and well-established practices no longer apply to them since
  volumetric representations do not have the underlying surface
  representation.
  For that reason, we see a couple of avenues which we explore in this thesis.

  Firstly, \textcite{park2021nerfies} proposed NeRFies, a model that creates a
  volumetric representation of a person from a self-captured sequence with a
  phone camera.
  Since the inception of NeRFs~\cite{mildenhall2020nerf}, it was among the
  first works the achieved such a high quality of reconstructions from a
  casual videos.
  In its primal form, NeRFies were unable to control the avatar in any other
  way than by a linear interpolation of latent embeddings that embedded the
  video's time dimension.
  The follow-up work, HyperNeRF~\cite{park2021hypernerf} handles this issue by
  projecting the learnable embeddings with $\latentdimension$ onto a
  lower-dimensional space~$\real^\projecteddimension$ where
  $\projecteddimension {\ll}\latentdimension$.
  After the assumption that the $\projecteddimension{=}2$ is enough to explain
  the sequence variability, that projected embedding becomes a 2-dimensional
  space that can be traversed in an interpretable way.
  However, that space is not intuitive since the projection is a non-linear
  operation and one cannot predict how values affect the results.
  To mitigate that issue, we propose to leverage smoothness of Multilayer
  Perceptrons (MLPs)~\cite{tancik2020fourier,park2021nerfies} to constrain the
  projection via sparse supervision.
  We realize our approach as a weakly-supervised MLP that out of many images
  from the sequence (we assume at least 100 frames in our work) only a few are
  provided with a coarse annotation.
  Such annotations denote what values a chosen attribute takes and where its
  effect spans in the image space.
  We show that our method, which we dubbed CoNeRF~\cite{kania2022conerf} and
  published at the CVPR 2022 conference, imbues NeRFs with a flexible
  editability feature without the lose of the rendering quality.

  Secondly, approaches such as CoNeRF~\cite{kania2022conerf},
  EditNeRF~\cite{liu2021editing} or FigNeRF~\cite{xie2021fig} focus solely on
  static elements of the scene, hence their controllability is limited to
  changing colors or textures in general.
  HyperNeRF~\cite{park2021hypernerf} arises as a potential solution due to its
  ability to model object deformations.
  However, our initial experiments showed that those changes cannot handle
  motions that affect a subject globally, \eg, jumping jacks performed by a
  person.
  To solve the issue, \textcite{fang2022fast} proposes to model the
  deformation via a multi-scale voxel structure which works well in the
  synthetic setting, such as the one proposed by~\textcite{pumarola2021d}.

  There exists a plethora of works that approach the problem from the another
  angle---instead of modeling the motion purely from data, they use a template
  model in the form of a 3D mesh to canonicalize deformed
  points~\cite{zielonka2022instant}.
  Such methods rely on the accuracy of the \textit{registration}, \ie, fitting
  the template mesh to subject.
  Since the registration methods~\cite{feng2021learning, zielonka2022towards}
  are imperfect estimators, they inherently contain registration errors.
  Those deviations are exacerbated by learnable radiance field models which
  assume a perfectly calibrated scene.
  The authors of those approaches usually mitigate the issue with additional
  latent space~\cite{grassal2022neural,kania2022conerf,martin2021nerf} that
  requires thousands of video frames to learn an avatar of high-fidelity that
  reacts correctly to deformations such as wrinkles on the forehead.
  At the same time, performing the registration on the large scale is
  costly~\cite{cao2022authentic}.
  In this thesis, we seek a remedy for those obstacles.
  We propose a method that is data-efficient, easy to improve with a minimal
  user input and can model realistic deformation dependent changes in the
  subject.
  Inspired by classical methods in character texturing~\cite{oat2007animated}
  and motion modeling~\cite{lewis2014practice}, we propose
  BlendFields~\cite{kania2023blendfields}, an \textit{homage} to traditional
  blendshapes~\cite{lewis2014practice}.
  We build on VolTeMorph's~\cite{garbin2024voltemorph} approach to point
  canonicalization to provide a data-efficient way to control the character.
  We further introduce a physically-based mixture of predefined, learned from
  data wrinkle templates that represent expression-dependent skin
  deformations.
  Our proposed was acknowledged by the reviewers and was accepted to the CVPR
  2023 conference.

  Thirdly, having the texture and coarse mesh-based controllability, we strive
  for control scene settings directly.
  The inverse rendering of 3D scenes is an ill-posed problem where many
  different lighting settings may explain the same light
  effects~\cite{patow2003survey}.
  To facilitate solving the problem, many approaches use datasets of single
  object's images captured under different lighting
  conditions~\cite{chen2022hallucinated,yang2023crnerf,rudnev2022nerfosr}.
  These approaches cannot decouple albedo from the lighting
  effects~\cite{chen2022hallucinated,yang2023crnerf} or need additional neural
  networks to predict correct shadows~\cite{rudnev2022nerfosr} which limits
  methods' practicality.
  We propose to use recently proposed 2D Gaussian Splatting~\cite{huang20242d}
  which exhibit remarkable quality of the surface reconstruction.
  Together with our precomputed radiance transfer from classical computer
  graphics approaches~\cite{slomp2006gentle,ramamoorthi2001envmap}, our
  LumiGauss achieves state-of-the-art reconstruction quality with the ability
  to render novel lighting conditions with high fidelity.
  Our work received positive reviews for the WACV 2025 conference.

  Finally, volumetric representation are computationally intensive to render,
  compared to the traditional mesh representation.
  For NeRFs~\cite{mildenhall2020nerf}, it takes seven days on V100 NVidia GPU
  to train for single scene, and more than 60 seconds to render a single
  image---way beyond any practical applications.
  Although many approaches have been proposed to speed up the rendering
  process~\cite{garbin2021fastnerf,yu2021plenoctrees,reiser2021kilonerf,hedman2021baking,mueller2022instant},
  they usually make a trade-off between memory requirements, quality, and
  rendering speed.
  3D Gaussian Splatting~\cite{kerbl20233d} (3DGS) rose as an alternative to NeRF, offering both high-quality rendering at interactive frame rates.
  However, those frame rates could be achieved with the most advanced GPU
  units available at that time.
  As we see the potential in 3DGS to be a viable canonical representation for
  3D data, akin to 3D meshes, a need for its adaptability to different
  computational resources exists.
  Meshes can be adapted easily with levels of detail (LoD) approaches that
  remove detail from meshes that do not affect the general object's perception
  if necessary.

\section{Research objectives}
  In this thesis, we explore different avenues of radiance field
  controllability.
  With this goal in mind, we aim at answering the following research questions:
  \begin{enumerate}[label=(\textbf{RQ \arabic*})]
    \item \label{enum:texturing}
          Can we imbue a Neural Radiance Field (NeRF) with a controllability
          by providing sparse annotations to the training dataset?
          How many annotations suffice to learn smooth interpolation
          capabilities between controlled values?

    \item \label{enum:movement}
          Are extreme facial expressions known from the literature sufficient
          to learn expression-dependent details that extrapolate to
          expressions unseen at the training time?

    \item \label{enum:lighting}
          Is it possible to learn an underlying radiance transfer function of
          a scene from images taken in ``in-the-wild'' setting?
          Can such a transfer function generalize to unknown environment maps?

    \item \label{enum:resources}
          How to learn a single 3D Gaussian Splatting (3DGS) representation
          that can be adapted to different computational regimes at inference
          time in a feed-forward mode?
  \end{enumerate}

  Each of these questions is a representative of possible among many others
  controllability directions for radiance fields.
  In case of this thesis, we present summarize our methods as controls of:
  texture~\cite{kania2022conerf,kania2023blendfields},
  shape~\cite{kania2023blendfields}, lightning~\cite{kaleta2024lumigauss} and
  use of resources~\cite{kania2024clog}.
  To answer~\ref{enum:texturing}, we describe our
  CoNeRF~\cite{kania2022conerf}.
  It is one of the pioneering works that uses sparsely annotated frames to
  continuously control the subject in a post-hoc manner.
  We leverage the fact that MLP used in NeRFs are smooth functions biased
  towards low frequency signals~\cite{tancik2020fourier}.
  For that reason, NeRF can learn to interpolate smoothly the annotation
  signal between frames of high similarity.
  We show that this assumption is sufficient to obtain both novel view
  synthesis and novel attribute synthesis with a single model.

  We further move towards answering~\ref{enum:movement}.
  We introduce BlendFields~\cite{kania2023blendfields}, achieving two primary
  goals: ability to generalize unknown expressions via a predefined face mesh
  template, and a mixture model of training expressions that can produce
  spatially coherent, expression-dependent wrinkles on the face from as few as
  three expressions.
  In our work, we build on VolTeMorph~\cite{garbin2024voltemorph} to achieve
  to former, and focus on the latter contribution.
  Inspired by texture maps in classical computer graphics
  pipelines~\cite{oat2007animated}, we define a set of learnable radiance
  fields, each being overfit to a particular extreme expression from the
  training set.
  We define an extreme expression as one of the possible expressions involving
  the most facial muscles.
  Building on VolTeMorph~\cite{garbin2024voltemorph} allows us to use an
  underlying tetrahedral mesh to compute physical quantities such as the
  volume change of tetrahedra under a given expression.
  We use those quantities to linearly interpolate between the pretrained
  radiance fields.
  We mix the tetrahedra independently which makes rendering novel expressions
  possible.
  For example, BlendFields can render one of the eyebrows raised while
  maintaining the other in a natural position, which is a difficult expression
  to make for majority of people.

  Our LumiGauss~\cite{kaleta2024lumigauss} answers~\ref{enum:lighting}.
  In contrary to common approaches~\cite{rudnev2022nerfosr}, we posit that the
  radiance transfer function known among computer graphics
  researchers~\cite{ramamoorthi2001envmap} can be learned from unconstrained
  photo collection under varying lighting conditions.
  To this end, we use 2DGS~\cite{huang20242d} which gives us a smooth shape
  representation, difficult to achieve when using 3DGS~\cite{kerbl20233d}.
  With the use of contributed priors, we induce learning Gaussian's Spherical
  Harmonics such that they correctly react to changing environment maps.
  Not only our approach is fast to train, but renders realistic scenes and
  object's shadows even under novel lighting conditions.

  All the contributions so far are affected by a specific disadvantage---a
  single model needs to be trained from scratch and it can be deployed only on
  high-end hardware.
  We then ask if we can train a single model that is adaptable at inference
  time to different hardware regimes~\ref{enum:resources}.
  We propose CLoG~\cite{kania2024clog} as a potential remedy.
  Our approach uses the fact that one can constrain the number of Gaussians in
  3DGS~\cite{kerbl20233d} to a specific number, such that it can be formed as
  a 2D grid by simple reshape operation.
  With a specific training coarse-to-fine training protocol we contributed,
  the model learns a representation that converges to a high-quality
  volumetric structure.
  In the second training stage, we leverage the fact that Gaussians's can be
  spatially sorted in such a manner that their descriptors are placed next to
  each other if they are similar.
  That forms a low-frequency image which can be modulated with an
  off-the-shelf continuous upsampling architecture~\cite{vasconcelos2023cuf}.
  The architecture outputs a new grid of the given resolution.
  We show in our work that such an approach achieves remarkable results and
  can output any number of Gaussians at inference time with high quality.

\section{Contributions}
  Building on those questions, we structure this thesis in several chapters
  corresponding to the answers.
  An answer is in a form of a scientific article where we introduce the following:
  \begin{itemize}
    \item A novel approach for controlling trained radiance fields with the use of sparsely annotated images from a casually captured data.
    \item A new model capable of blending trained radiance fields from multi-view frames in an interpretable way and extrapolating to novel human expressions for the trained subject.
    \item The first use of Gaussian Splatting methods that learns a coherent shape representation of a subject and an ability to distill the varying lighting conditions in the data to a radiance transfer function.
    \item A novel paradigm for learning Gaussian Splatting models as 2D grids to achieve flexibility to adapt the model to different computational regimes at inference time.
  \end{itemize}

  % not using primes (not canonicalized parameters)
\begin{figure}
  \centering
  \begin{subfigure}[t]{0.38\linewidth}
    \centering
    \includegraphics[width=\linewidth]{assets/conerf/img/beta-importance-annotations.pdf}
    \caption{annotated samples}
  \end{subfigure}
  \hfill{}
  \begin{subfigure}[t]{0.595\linewidth}
    \centering
    \includegraphics[width=\linewidth]{assets/conerf/img/beta-importance-preds.pdf}
    \caption{renderings from the trained model}
  \end{subfigure}
  \caption{{\bf Example with annotations and controlled attributes --}
    We show an example of how our \conerf is capable of controlling attributes
    selected sparse annotations at the training time.
    Top row shows possible value combinations ($\color{red}{-}$ denoting closed
    eye, $\color{orange}{0}$ neutral position and $\color{green}{+}$ an open eye)
    and $\{\bbeta_1,\bbeta_2\}$ possible values of attributes that are not
    explicitly learned from the annotations but purely from data (please
    see~\cref{chap:conerf} for further explanation).
  }
  \label{fig:conerf-teaser}
\end{figure}
  \subsection{Texture from Partial Information}
    Existing NeRF-based approaches in 2021 were simple models---they were
    overfit to a single subject, for a new subject the model needed to be
    retrained from scratch, and the editability capabilities were
    limited~\cite{liu2021editing}.
    NeRFactor~\cite{zhang2021nerfactor} could be considered a more
    sophisticated model.
    However, it worked only for simple scene with calibrated scenes and could
    not handle any motion.

    We approach those issues in our CoNeRF~\cite{kania2022conerf} published at
    the CVPR 2021 conference.
    Inspired by HyperNeRF~\cite{park2021hypernerf}, we propose to revisit the
    weak supervision in the context of radiance fields.
    Specifically, under an assumption that one can provide a few of sparse
    annotations to the dataset, we can leverage the smoothness of the neural
    networks to propagate the annotation across the dataset.
    The annotations also consist of what regions in the image space it refers
    to and hence we can train a semantic segmentation radiance fields that
    decouples attribute controls.
    We show an example in~\cref{fig:early-conerf-teaser-intro} where the left
    side represents the annotations present in the data and the right one
    possible manipulations at the inference time.
    We note that those annotations are easy to make in a matter of a few
    minutes for a single dataset.
    However, the complexity of the annotations grows with the number of
    attributes to control.

  \subsection{Expression from Few-Shot Learning}
    \conerf~\cite{kania2022conerf} is capable of rendering complex motions
    given sufficient amount of data and provided annotations.
    However, motions as the ones a person performs daily when speaking are
    infeasible in practice.
    We propose a solution to target that issue.
    We introduce \blendfields~\cite{kania2023blendfields} from the CVPR 2023
    proceedings which learns motion-dependent face deformation from data.
    We build on VolTeMorph~\cite{garbin2024voltemorph} which uses existing
    face template models, such as FLAME~\cite{li2017flame}, to learn a single
    canonical representation, akin to the canonicalization module in CoNeRF.
    Internally, \blendfields creates a tetrahedral cage around the face model.
    For each sample along the ray in NeRF, it moves the points to a ``neutral
    position'', chosen once prior to the training procedure.
    Such a procedure comes insufficient to model realistic facial features,
    such as wrinkles.
    We contribute a novel approach to modeling those deformations.
    We compute a deformation gradient of each tetrahedra for a given
    expression which is a physically-based and easy to interpret quantity.
    The value serves us to smoothly transition face regions textures to
    appropriate colors.
    We obtain the colors from NeRFs branches, each overfit to particular
    expression.
    In principle, \blendfields predicts face bases which are conditioned on
    the face expression vector to output the final point color.
    Our framework works well even when only a few ``extreme'' expressions are
    provided such as grinning face with closed eyes and wide open mouth with
    open eyes.

  \subsection{Light from Unconstrained Images}
    In both approaches above, we tackle the problem of texture
    controllability.
    They assume an ideal case scenario where a capture can be taken in an
    idealized environment with constant camera exposure and lighting.
    Moreover, the produced colors are blended together, making the change of
    light impossible in practice.
    We then ask the questions if we can decouple an intrinsic color of the
    subject and change of that color stemming from the environment light.
    We answer that question with our \lumigauss~\cite{kaleta2024lumigauss}
    published at the WACV 2025 conference that, indeed, we can achieve that by
    learning the radiance transfer function directly from data.
    For that end, we train a 2DGS~\cite{huang20242d} model to obtain a smooth
    and spatially coherent surface of objects.
    On top of the other attributes known from 3DGS~\cite{kerbl20233d}, we
    imbue our Gaussians with additional features corresponding to the radiance
    transfer, expressed as Spherical Harmonics~\cite{green2003grittydetail}.
    We show in our experiments that such a formulation is sufficient to train
    a model that reacts to changing environment maps in a realistic manner and
    renders images of higher fidelity than prior approaches.

  \subsection{Levels of Detail in One Model}
    All the contributions above require considerable computation hardware to
    be trained on and then run inference in near real-time frame rates.
    Drawing an inspiration from the gaming industry where Levels of Details
    (LoD) for meshes is used heavily, we introduce \clog~\cite{kania2024clog}.
    Our approach trains a single Gaussian Splatting model such that it can be
    modulated at inference time and adapted to the target computational
    requirements with a minimal loss on the rendering quality.
    That allows us to democratize the use of 3DGS and deploy a single model
    even on handheld devices.
    We show later that even under a strict case where only 2,000
    Gaussians\footnote{Common 3DGS models can achieve from $10^5$ to even
    $10^6$ Gaussians.
    } can be used, the object is still recognizable.
    The most important contribution of our model is that it is continuous by
    design while the existing baselines assume prior the training the model
    how many LoD the model should comprise at inference.
    To change the size of particular LoD in those baselines requires training
    the whole model from scratch, imposing a significant computational burden.

\section{Thesis outline}
  This thesis is structured as follows.
  We start by introducing the preliminaries related to the Neural Radiance
  Fields and Gaussian Splatting in~\cref{chap:background}---a common theme in
  all works that appear later.
  We then move towards describing CoNeRF~\cite{kania2022conerf}
  in~\cref{chap:conerf}, our approach to control trained neural radiance
  fields by using sparse annotations.
  In~\cref{chap:blenfields}, we describe
  BlendFields~\cite{kania2023blendfields} that can produce realistic
  expression-dependent texture from just a few multi-view frames of the
  subjects.
  \cref{chap:lumigauss} introduces the LumiGauss~\cite{kaleta2024lumigauss}, a Gaussian Splatting model that learns a radiance transfer function for novel lighting rendering capabilities.
  Finally, we bring a general method, CLoG~\cite{kania2024clog}, that uses
  Gaussian Splatting to learn continuous levels of detail while training only
  a single model.
  We conclude the thesis in~\cref{chap:final} where we also explore possible
  future avenues that can be undertaken.

\section{Publications not included in the thesis}
  We attach a list of articles that are related to and can be used to in neural rendering approaches:
  \begin{itemize}
    \item\!\!\!\fullciteallauthors{kania2020ucsg},
    \item\!\!\!\fullciteallauthors{kania2021trajevae},
    \item\!\!\!\!\!\fullciteallauthors{stypulkowski2021representing},
    \item\!\!\!\!\fullciteallauthors{xia2023densify},
    \item\!\!\!\fullciteallauthors{esposito2024geogen},
    \item\!\!\!\!\fullciteallauthors{spurek2024modeling}.
  \end{itemize}

