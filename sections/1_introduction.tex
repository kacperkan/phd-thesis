\chapter{Introduction}
\label{chap:introduction}

With the advent of deep learning, research have been exploring varying ways to
apply it to computer graphics.
One of the most recent and promising approaches is neural rendering.
Neural rendering is a field that combines deep learning and computer graphics
to generate realistic images of 3D scenes.
The neural radiance field (NeRF) is a popular neural rendering technique that
represents a 3D scene as a continuous function that maps 3D coordinates to
radiance values.
NeRF has shown impressive results in generating photorealistic images of 3D
scenes.
However, NeRF has limitations in terms of memory and computational
requirements, which makes it difficult to scale to large scenes.

To alievate the problem, \citet{kerbl20233d} proposed a new technique---3D
Gaussian Splatting (3DGS).
3DGS is a neural rendering technique that represents a 3D scene as a set of 3D Gaussian that are splatted to an image space using algorithm proposed by~\citet{zwicker2001ewa}.
In contrast to NeRF, 3DGS is more memory efficient and can be used to render
large scenes.
It can also render scenes with millions of points in real-time on a single
GPU.

In this thesis, we focus on those two milestone techniques in neural rendering
and address their fundamental problem---lack of controllability.

\section{Motivation and problem statement}

\section{Research objectives}

\section{Contributions}

\section{Thesis outline}

  % Deep learning

  % Neural rendering

  % neural radiance field

  % 3D Gaussian Splatting

