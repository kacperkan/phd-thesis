\definecolor{fixedop}{HTML}{6E8DBC}
\definecolor{learnableop}{HTML}{82B26C}
\newcommand{\myscheme}{
  {
      \tikzset{external/export=false}
      \tikzsetnextfilename{scheme}
      \centering
      \resizebox{\linewidth}{!}{
        \begin{tikzpicture}[
            >=stealth',
            shorten >= 2pt,
            shorten <= 2pt,
            thick,
            align=center,
            fixed/.style={
                rectangle,
                draw=fixedop,
                thick,
                line width=0.5mm,
                rounded corners=4pt
              },
            learnable/.style={
                rectangle,
                draw=learnableop,
                thick,
                line width=0.5mm,
                rounded corners=4pt
              },
          ]

          \node (inputimages) {\includegraphics[width=4cm]{\lumigaussdirname/folder_for_kacper/scheme/input-images.pdf}};
          \node[fixed, above=0.5cm of inputimages.north, text width=4cm] (cameras) {calibrated\\cameras from SfM\\$\{\camera_c\}$};
          \node[right=1cm of cameras.east]  (rawgausses) {\includegraphics[trim={2bp 0 0 0}, clip]{\lumigaussdirname/folder_for_kacper/scheme/raw-gausses.pdf}};
          \node[learnable,above right=1cm and 1cm of rawgausses.east, minimum width=2cm] (normal) {normal $\normal_k$};
          \node[learnable,above right=0cm and 1cm of rawgausses.east, minimum width=2cm] (albedo) {albedo $\albedo_k$};
          \node[learnable,above right=-1cm and 1cm of rawgausses.east, minimum width=2cm] (sh) {SH $\transferfunsh_k$};

          \node[fixed, above right=-1cm and 4cm of rawgausses.east, minimum width=3cm, minimum height=1cm] (shadowed) {shadowed\\(\cref{eq:lumigauss-shadowed-radiance})};
          \node[fixed, above right=0.5cm and 4cm of rawgausses.east, minimum width=3cm, minimum height=1cm] (unshadowed) {unshadowed\\(\cref{eq:lumigauss-unshadowed-radiance})};

          \node[right=8cm of rawgausses.east] (coloredgausses) {\includegraphics{\lumigaussdirname/folder_for_kacper/scheme/colored-gausses.pdf}};
          \node[below=0.5cm of coloredgausses.south] (rendered) {\includegraphics{\lumigaussdirname/folder_for_kacper/scheme/rendered-image.pdf}};

          \node[below=0em of inputimages.south, text width=5cm, anchor=north] (unconstrained) {Unconstrained photo collection\\$\{\image_c\}$};

          \node[learnable, text width=4cm, right=1cm of inputimages] (envmap) {Environment map embedding \\(\cref{eq:lumigauss-lightsh})};
          \node[right=2cm of envmap.east] (shball) {\includegraphics[trim={2pt 2pt 2pt 2pt}, clip]{\lumigaussdirname/folder_for_kacper/scheme/sh.pdf}};

          \node[below=-0.5em of rendered.south, text width=4cm,anchor=north] {Render image\\with novel view and light};

          \node[fixed, right=2cm of unconstrained.east, minimum height=0.7cm, minimum width=4cm] (fixedopnode) {fixed operation};
          \node[learnable, right=0cm of fixedopnode.east, minimum height=0.7cm, minimum width=4cm] (learnableopnode) {learnable operation};

          \draw [->] (inputimages.north) -- (cameras.south);
          % \draw [->] (unconstrained.north) -- (inputimages.south);
          \draw [->] (cameras.east) -- (rawgausses.west);
          \draw [->] (coloredgausses.south) to  node[right] {$\splatting(\cdot)$}  (rendered.north);
          \draw [->, rounded corners=2pt] (rawgausses.east) |- (normal.west);
          \draw [->, rounded corners=2pt] (rawgausses.east) |- (albedo.west);
          \draw [->, rounded corners=2pt] (rawgausses.east) |- (sh.west);

          \draw [->, rounded corners=2pt] (normal.east) -- ($(normal.east)+(12pt, 0)$) |-  (unshadowed.west);
          \draw [->, rounded corners=2pt] (albedo.east) -- ($(albedo.east)+(12pt, 0)$) |- (unshadowed.west);
          \draw [->, rounded corners=2pt] (albedo.east) --  ($(albedo.east)+(12pt, 0)$) |- (shadowed.west);
          \draw [->, rounded corners=2pt] (sh.east) --  ($(sh.east)+(12pt, 0)$) |-  (shadowed.west);

          \draw [->, rounded corners=2pt] (unshadowed.east) -- ($(unshadowed.east)+(12pt, 0)$) |- (coloredgausses.west);
          \draw [->, rounded corners=2pt] (shadowed.east)-- ($(shadowed.east)+(12pt, 0)$) |- (coloredgausses.west);

          \draw [->] (inputimages.east) -> (envmap.west);
          \draw [->] (envmap.east) -> (shball.west);

          \draw [->, rounded corners=2pt] (shball.north) |- ([yshift=0.2cm]shadowed.south west);
          \draw [->, rounded corners=2pt] (shball.north) |- ([yshift=0.2cm]unshadowed.south west);

          \node[above=-1em of rawgausses.north] {{Raw 2D Gaussians} \cite{zhang2024gaussian}};
          \node[above=-1em of coloredgausses.north] {{Relighted 2D Gaussians} (\textbf{Ours})};
        \end{tikzpicture}
      }
    }
}

\begin{figure*}[!t]
  \centering
  \myscheme
  \vspace{-2em}
  \caption{
    \textbf{Pipeline --}
    \lumigauss learns the relightable 2D Gaussian~\cite{zhang2024gaussian} representation from unconstrained photo collection with variable camera parameters and lighting conditions.
    Each of $k$ Gaussians holds: a normal $\normal_k$, albedo $\albedo_k$, and
    learnable transfer function $\transferfunsh_k$.
    Our contributed method composes the Gaussians in two
    modes---\textit{shadowed} and \textit{unshadowed}.
    The \textit{shadowed} model reconstructs additional shadows
    (see~\cref{fig:lumigauss-teaser}) on top of the unshadowed model thanks to
    our proposed use of a radiance transfer function.
    The Gaussians are splatted~\cite{kerbl20233d,zhang2024gaussian} to render
    the output image in a novel view and light.
  }
  \label{fig:lumigauss-scheme}
  \vspace{-1.1em}
\end{figure*}
\section{Method}
  \label{sec:lumigauss-method}

  \subsection{Preliminaries on Radiance Transfer}
    \label{subsec:lumigauss-preliminaries}
    The rendering equation, in its simplified form~\cite{green2003grittydetail}, is an integral function that represents light $L(\mathbf{x}, \direction_o)$ exiting point $\mathbf{x}$ along the vector $\direction_o$:
    \begin{equation}
      \label{eq:lumigauss-brdf}
      L(\bx, \direction_o)\!
      =\!\int_{s} f_r(x, \direction_o, \direction_i) L_i(\bx, \direction_i) \transferfun(\bx, \direction_i) d\direction_i
    \end{equation}
    where $f_r(\cdot)$ is a BRDF function, $L_i(\cdot)$ an incoming light along the vector $\direction_i$, and $\transferfun(\cdot)$ is a radiance transfer function.
    Intuitively, $f_r(\cdot)$ represents the surface material, $L_i(\cdot)$
    represents the intensity and color of the illumination, and
    $\transferfun(\cdot)$ is a term that takes into account shadows or light
    reflections from other surfaces.
    Depending on the formulation of those functions, the rendering equation
    can range from a straightforward and inaccurate light model to a highly
    complex and accurate one.

    \paragraph{Unshadowed model}
      One example of a reflection model that can be represented with
      \cref{eq:lumigauss-brdf} is the diffuse surface reflection model, also
      known as \textit{dot product lighting}.
      A diffuse BRDF reflects light uniformly, making the lighting view-independent and simplifying the BRDF as follows:
      \begin{equation}
        \label{eq:lumigauss-dotillum}
        L_{D}(\bx) = \frac{\rho(\bx)}{\pi} \int_{s} L_i(\bx, \direction_i)
        \max(\mathbf{n}(\bx) \cdot \direction_i, 0) d\direction_i
        \end{equation} where $\rho(\cdot)$ is the surface albedo,
        $\mathbf{n}(\bx)$ a surface normal at the point $x$.
      Shadows are neglected.

      The incoming light $\light_i(\bx, \direction_i)$ can be represented in
      several ways.
      In this work, we assume that the scene is illuminated with an
      \textbf{omnidirectional environment map} that is parametrized using
      spherical harmonics (SH) of degree $n$ with $(n{+}1)^2$ coefficients.
      Because the environment map is positioned infinitely far from the scene, the light is position-independent, and thus, the rendering equation is further simplified:
      \begin{equation}
        \label{eq:lumigauss-unshadowed}
        \light_{U}(\bx) = \frac{\rho(\bx)}{\pi} \int_{s} \light_i(\direction_i) \max(\mathbf{n}(\bx) \cdot \direction_i, 0) d\direction_i
      \end{equation}
      With illumination parametrized with SH, we can evaluate the integral
      in~\cref{eq:lumigauss-unshadowed} using a closed-form solution from
      Eq.~(12) in \cite{ramamoorthi2001envmap}.
      From this point onward, we refer to rendering with
      \cref{eq:lumigauss-unshadowed} as \textit{unshadowed}.

    \paragraph{Shadowed model}
      In addition to the \textit{unshadowed} lighting model, we propose a
      \textit{shadowed} model, where $\transferfun(\bx, \direction_i)$ is
      parameterized using spherical harmonics (SH) and learned from training
      data.
      In $\transferfun(\bx, \direction_i)$, SH represents a spherical signal
      that quantifies the light arriving from each direction of the
      environment map to an associated point in space.
      The \textit{shadowed} model is derived by replacing the dot product term in~\cref{eq:lumigauss-unshadowed}:
      \begin{equation}
        \label{eq:lumigauss-shadowed}
        L_{S}(\bx) = \frac{\rho(\bx)}{\pi} \int_{s} L_i(\direction_i)
        \transferfun(\bx, \direction_i) d\direction_i.
      \end{equation}
      In addition to modeling shadows, this approach also has the potential to
      model the interreflection of light between objects in the scene.

      Using SH of the same degree for both the environment map and transfer
      function allows efficient evaluation of the rendering equation
      \cref{eq:lumigauss-shadowed}.
      A key SH property simplifies the integral of two SH-based functions to a
      dot product of their coefficients, thanks to SH orthogonality.
      With this property \cref{eq:lumigauss-shadowed} can be re-written as:
      \begin{equation}
        \label{eq:lumigauss-shadowed_sh}
        L_{S}(\bx) = \frac{\rho(\bx)}{\pi} \lightsh \cdot \transferfunsh,
        \end{equation} where $\lightsh \in R^{(n+1)^2}$ are the SH
        coefficients of $L_i(\direction_i)$ and $\transferfunsh \in
        R^{(n+1)^2}$ are the SH coefficients of $D(\bx, \direction_i)$.
      Please see~\cite{slomp2006gentle, green2003grittydetail} for derivation.
      This property is commonly used in real-time rendering where the radiance
      transfer function is pre-computed and
      only~\cref{eq:lumigauss-shadowed_sh} is evaluated at runtime.

  \subsection{\lumigauss}
    \lumigauss creates a 3D representation of a relightable model using 2D Gaussians~\cite{huang20242d} from $c\leq C$ images taken \textit{in-the-wild}  $\{\mathcal{I}_c\}_{c=1}^C$ with associated calibrated cameras~$\{\camera_c\}_{c=1}^C$.
    Our goal is to find Gaussian parameters
    $\mathcal{G}{=}\{\translation_k,\rotation_k,\scaling_k,\opacity_k,
    \albedo_k,\transferfunsh_k\}^K_{k=1}$ that after the
    rasterization~\cite{kerbl20233d} recreate those images.
    We optimize Gaussians by minimizing the objective:
    \begin{equation}
      \argmin_{\gaussians, \latents,\params} \mathbb{E}_{\camera_c\sim \{\camera_c\}} \underbrace{\loss{rgb}\!\left(\splatting(\camera_c\;|\;\gaussians, \latents, \params), \image_c\right)}_{\text{\cref{subsec:lumigauss-reconstruction}}}  + \underbrace{\mathcal{R}(\gaussians)}_{\text{\cref{subsec:lumigauss-constraints}}},
    \end{equation}
    where $\latents{=}\{\embedding_c\}_{c=1}^C$ is a set of scene-dependent, learnable environment embeddings, $\loss{rgb}$ is a photometric objective that compares the rendered image from an operator $\mathcal{S}(\cdot)$~(\cref{eq:lumigauss-splatting-op}), and $\mathcal{R}$ are additional regularization terms.
    In contrast to 2DGS~\cite{huang20242d}, for each Gaussian we model the
    base color $\albedo$ as diffuse\footnote{As
    per~\cref{subsec:lumigauss-preliminaries}, view-dependent effects are not
    modeled in diffuse reflections.
    }, and introduce SH coefficients for the transfer function $\transferfunsh$\footnote{These coefficients correspond to a single channel in practice.}. 2DGS provides smooth normals that make relighting possible.

    In what follows, we drop the dependence of functional forms on the
    positions $\bx$ we introduced in~\cref{subsec:lumigauss-preliminaries} for
    brevity.

    \paragraph{Relighting}
      To handle the diverse lighting conditions in \textit{in-the-wild}
      images, we associate each training image with a learnable latent code
      $\embedding_\sceneindex$ that encodes its lighting conditions.
      Using this embedding, we predict the environment map coefficients via an MLP:

      \begin{equation} \label{eq:lumigauss-lightsh} \lightsh_\sceneindex = \text{MLP}(\embedding_\sceneindex | \params), \end{equation}

      where $\lightsh_\sceneindex \in \mathbb{R}^{3 \times (n+1)^2}$ represents the SH coefficients of the environment map, and $n{=}2$ is the SH degree.
      As shown in~\cite{ramamoorthi2001envmap}, second-order SH is sufficient
      to approximate environment lighting in many scenarios.

      The predicted illumination is used in the rendering process in one of
      two ways: \textit{unshadowed} and \textit{shadowed}.
      Those two approaches correspond to~\cref{eq:lumigauss-unshadowed}
      and~\cref{eq:lumigauss-shadowed_sh} respectively, and are described
      below.

    \paragraph{Unshadowed model}
      For the unshadowed scenario, we follow~\cref{eq:lumigauss-unshadowed},
      which integrates light over the hemisphere in the direction of the
      surface normal.
      The color $\radiance_k$, \textit{radiance}, for each Gaussian $\mathcal{G}_k$ given its normal $\normal_k$ and the illumination parameters $\mathbf{l}_\sceneindex$ equates to:
      % \begin{align} \radiance_k = \albedo_k \cdot \irradiance_k, \quad
      % \irradiance_k = \mathbf{l}_\sceneindex \cdot \normal_k \end{align}
      \begin{equation}
        \label{eq:lumigauss-unshadowed-radiance}
        \radiance_k = \albedo_k \odot \underbrace{\normal_k^t M(\lightsh_k) \normal_k}_{\text{unshadowed irradiance}},
      \end{equation}
      where $M$ is a $4{\times}4$ matrix derived from the SH parameters of the environment map.
      It is the closed form solution of the integral in
      \cref{eq:lumigauss-unshadowed}, please see Eq.
      (12) in \cite{ramamoorthi2001envmap} for details.

      This simple yet effective model already imbues the model with relighting
      capabilities.
      However, as described in~\cref{fig:lumigauss-shadowed_unshadowed} it
      does not capture shadows correctly, limiting the output's fidelity.

      \def\scale{1.0}
\def\distbelow{2.5em}
\def\distabove{2.5em}
\def\height{256bp}

\newcommand{\shadowedoverpicone}{
  \begin{overpic}[abs,height=\height]{\lumigaussdirname/folder_for_kacper/wild2d-shadowed_unshadowed_fig1.pdf}
    \put(90,200){$\normal$}
    \put(155,240){$\direction_1$}
    \put(65,130){$\direction_2$}
  \end{overpic}
}

\newcommand{\shadowedoverpictwo}{
  \begin{overpic}[abs,height=\height]{\lumigaussdirname/folder_for_kacper/wild2d-shadowed_unshadowed_fig2.pdf}
    \put(155,200){$\normal$}
    \put(220,240){$\direction_1$}
    \put(130,130){$\direction_2$}
  \end{overpic}
}

\newcommand{\shadowedoverpicthree}{
  \begin{overpic}[abs,height=\height]{\lumigaussdirname/folder_for_kacper/wild2d-shadowed_unshadowed_fig3.pdf}
    \put(155,200){$\normal$}
    \put(220,240){$\direction_1$}
    \put(130,130){$\direction_2$}
  \end{overpic}
}
\newcommand{\shadowedunshadowed}{
  {
      \tikzset{external/export=false}
      \tikzsetnextfilename{shadowed_unshadowed}
      \resizebox{0.9\linewidth}{!}{
        \begin{tikzpicture}[
            >=stealth',
            overlay/.style={
                anchor=south west,
                draw=black,
                rectangle,
                line width=0.8pt,
                outer sep=0,
                inner sep=0,
              },
            font=\Huge
          ]
          \matrix[matrix of nodes, column sep=1pt, row sep=0pt, ampersand replacement=\&, inner sep=0, outer sep=0, font=\Huge] (shadowedunshadowed) {
            \shadowedoverpicone{} \&
            \shadowedoverpictwo{} \&
            \shadowedoverpicthree{} \\
          };
          \node[below=\distbelow of shadowedunshadowed-1-1.south, anchor=base, scale=\scale] {
            $\begin{aligned}
                \transferfunsh(\direction_1) & {=}\max(0, \cos(\normal, \direction_1)) \\
                \transferfunsh(\direction_2) & {=}\max(0, \cos(\normal, \direction_2))
              \end{aligned}$
          };

          \node[below=\distbelow of shadowedunshadowed-1-2.south, anchor=base, scale=\scale] {
            $\begin{aligned}
                \transferfunsh(\direction_1) & {=}\max(0, \cos(\normal, \direction_1)) \\
                \transferfunsh(\direction_2) & {=}\max(0, \cos(\normal, \direction_2))
              \end{aligned}$
          };

          \node[below=\distbelow of shadowedunshadowed-1-3.south, anchor=base, scale=\scale] {
            $\begin{aligned}
                \transferfunsh(\direction_1) & {=}\max(0, \cos(\normal, \direction_1)) \\
                \transferfunsh(\direction_2) & {<}\max(0, \cos(\normal, \direction_2))
              \end{aligned}$
          };

          \node[above=\distabove of shadowedunshadowed-1-1.north, anchor=base, scale=\scale, align=center]{
            % Unshadowed \\ = shadowed radiance \\
            $\radiance = \shadowedradiance$ \\
            (\cref{eq:lumigauss-unshadowed-radiance} \& \cref{eq:lumigauss-shadowed-radiance})
          };

          \node[above=\distabove of shadowedunshadowed-1-2.north, anchor=base, scale=\scale, align=center]{
            $\radiance$ does not incorporate \\
            the light blocker
          };

          \node[above=\distabove of shadowedunshadowed-1-3.north, anchor=base, scale=\scale, align=center]{
            $\shadowedradiance$ incorporates \\
            the light blocker
          };

        \end{tikzpicture}
      }
    }
}

\begin{figure}[!t]
  \centering
  % \includegraphics[width=\linewidth]{\lumigaussdirname/wild2d-shadowed_unshadowed.pdf}
  \shadowedunshadowed
  \caption{
    Unshadowed $\radiance$ (\cref{eq:lumigauss-unshadowed-radiance}) and
    shadowed $\shadowedradiance$ (\cref{eq:lumigauss-shadowed-radiance}) may
    give the same output color if a Gaussian is fully exposed to the
    environment light.
    In the case of any occluder, $\radiance$ does not handle, and the color
    does not change.
    However, our proposed $\shadowedradiance$ properly reacts to the occluder
    and makes the output color darker.
  }

  \label{fig:lumigauss-shadowed_unshadowed}
\end{figure}
    \paragraph{Shadowed model}
      To effectively capture shadows in the model, we redefine the output
      color of a Gaussian as $\shadowedradiance_k$, a function of learnable
      radiance transfer function $\transferfun_k$ parametrized by spherical
      harmonics $\transferfunsh_k \in \mathbb{R}^{(n+1)^2}$, light
      $\lightsh_c$ and albedo $\albedo_k$.
      Using a learned radiance transfer function (instead of fixing it to
      capture light from the hemisphere above the normal as we do in
      \textit{unshadowed}) allows for creating shadows, as described in
      \cref{subsec:lumigauss-preliminaries}.
      Overall, following \cref{eq:lumigauss-shadowed_sh}, the output shadowed color or \textit{radiance} reduces to:
      \begin{equation}
        \label{eq:lumigauss-shadowed-radiance}
        \shadowedradiance_{k}   = \albedo_k \odot \underbrace{\sum_{i=1}^{(n+1)^2}{\mathbf{l}_c^i \cdot \mathbf{d}_{k}^i}}_{\text{shadowed irradiance}},
      \end{equation}
      As we show later in the experiments, the addition of shadows leads to
      more accurate relighting.
      Additionally, it does not require learnable MLP to reconstruct shadows
      at the inference stage, differentiating it from
      NeRF-OSR~\cite{rudnev2022nerfosr} and making our approach applicable to
      rendering engines directly.

      % We note that the model can be extended further to include specular
      % reflections~\tocite{gaussian codec avatars}, which we leave as an
      % attractive future venue.

      \def\spysize{24px}
\def\offset{36px}
\def\height{64px}
\newcommand{\inpscene}[1]{\includegraphics[height=\height]{\lumigaussdirname/folder_for_kacper/relighting/#1.png}}

\newcommand{\relighting}{
  \centering
  \tikzsetnextfilename{qualitative_ours}
  \resizebox{\linewidth}{!}{
    \begin{tikzpicture}[
        >=stealth',
        overlay/.style={
            anchor=south west,
            draw=black,
            rectangle,
            line width=0.8pt,
            outer sep=0,
            inner sep=0,
          },
        font=\Huge
      ]
      \matrix[
        matrix of nodes,
        column sep=0pt,
        row sep=0pt,
        ampersand replacement=\&,
        inner sep=0,
        outer sep=0
      ] (pictures) {
        \inpscene{orig_1} \&
        \inpscene{rec_1} \&
        \inpscene{novel_1} \\
        \inpscene{orig_2} \&
        \inpscene{rec_2} \&
        \inpscene{novel_2} \\
        \inpscene{orig_3} \&
        \inpscene{rec_3} \&
        \inpscene{novel_3} \\
        \inpscene{orig_4} \&
        \inpscene{rec_4} \&
        \inpscene{novel_4} \\
      };
      \matrix[
        matrix of nodes,
        column sep=0pt,
        row sep=0pt,
        ampersand replacement=\&,
        inner sep=0,
        outer sep=0,
        right=of pictures
      ] (chair) {
        \inpscene{obj_1} \\
        \inpscene{obj_2} \\
        \inpscene{obj_3} \\
        \inpscene{obj_4} \\
      };
      \matrix[
        matrix of nodes,
        column sep=0pt,
        row sep=0pt,
        ampersand replacement=\&,
        inner sep=0,
        outer sep=0,
        right=of chair
      ] (chair) {
        \inpscene{fic_1} \\
        \inpscene{fic_2} \\
        \inpscene{fic_3} \\
        \inpscene{fic_4} \\
      };

    \end{tikzpicture}
  }
}

\begin{figure}[!t]
  \centering
  % \begin{overpic}[abs,width=\linewidth]{images/folder_for_kacper/relighting/relit.pdf}
  \begin{overpic}[percent,width=\linewidth]{\lumigaussdirname/folder_for_kacper/relighting/wild2d-relight_items_3items.pdf}
    \put(5,49.5) {\footnotesize Target Image}
    \put(26.5,49.5) {\footnotesize Reconstruction}
    \put(49.5,49.5) {\footnotesize Novel View}
    \put(75.2,49.5) {\footnotesize Relit Objects}
  \end{overpic}
  \vspace{-3em}
  % \relighting
  \caption{\textbf{Scene reconstruction and relightning --}
    Reconstruction and relighting capabilities of \lumigauss.
    \lumigauss reproduces sharp and clean landmarks, and the learned environment lighting enables accurate scene relighting.
    We use learned environment maps to relight the scene from novel viewpoints
    and then relight arbitrary objects within a graphics engine.
  }
  \label{fig:lumigauss-qualitative_ours}
\end{figure}
      % \subsection{Optimization} \asia{tutaj nie jest to ładnie sformułoane,
      % ale mam nadzieje ze zrozumiale w miare}

  \subsection{Physical constraints}
    \label{subsec:lumigauss-constraints}
    The regularizations proposed in 2DGS~\cite{huang20242d} keep the Gaussians
    close to the surface and smooth locally, which is crucial in our
    relighting scenario.
    Aside from them, we propose new loss terms based on the physical light
    properties that restrict the optimization from achieving degenerate,
    \textit{non-relightable} cases.
    We restrict radiance transfer $\transferfun_k$ function to remain within the range of 0 to 1, where 0 indicates complete shadowing and 1 signifies full exposure~to~lighting:
    % we retarget $\transferfun_k$ to serve as a mask that attenuates the light
    % environment's components, creating a shadow \asia{not sure if we can say
    % we retarget. We use it exactly how it was designed to be used}:
    \begin{equation}
      \begin{split}
        \loss{0-1} & =\mathbb{E}_k\mathbb{E}_{\direction_i} [\|\max(\transferfun_k(\direction_i), 1)-1\|^2_2 \\
                   & \qquad\quad+ \|\min(\transferfun_k(\direction_i), 0)\|^2_2],
      \end{split}
    \end{equation}
    and allow the environment light to remain in the $\mathbb{R}_{+}$ domain:
    \begin{equation}
      \loss{+} = \mathbb{E}_k\mathbb{E}_{\direction_i}\|\min(\light_\sceneindex(\direction_i), 0)\|^2_2,
    \end{equation}
    which allows the environment light to brighten the scene arbitrarily.

    The shadowed radiance transfer should remain close to the unshadowed
    version.
    If not, the shadowed version might include light from any direction,
    resulting in degenerate solutions and incorrect relighting.
    We visualize the shadowed and unshadowed transfer functions in
    \cref{fig:lumigauss-shadowed_unshadowed}.
    To address this issue, we propose the following loss function:
    \begin{equation}
      \loss{\removespace{\shadowed}$\leftrightarrow$\removespace{\unshadowed}} = \mathbb{E}_k\mathbb{E}_{\direction_i}\|\max(\normal_k \cdot \direction_i, 0) - \transferfun_k(\direction_i)\|^2_2,
    \end{equation}
    
\begin{table}[t]
  \centering
  \setlength{\tabcolsep}{4pt}
  \resizebox{\linewidth}{!}{
    \begin{tabular}{lcccccccccccc}
      \toprule
      \multirow{2}[2]{*}{Method}                    & \multicolumn{4}{c}{Landwehrplatz} & \multicolumn{4}{c}{Ludwigskirche} & \multicolumn{4}{c}{Staatstheater}                                                                                                                                                                       \\
      \cmidrule(lr){2-5} \cmidrule(lr){6-9} \cmidrule(lr){10-13}
                                                    & PSNR $\uparrow$                   & MSE $\downarrow$                  & MAE $\downarrow$                  & SSIM $\uparrow$ & PSNR $\uparrow$ & MSE $\downarrow$ & MAE $\downarrow$ & SSIM $\uparrow$ & PSNR $\uparrow$ & MSE $\downarrow$ & MAE $\downarrow$ & SSIM $\uparrow$ \\
      \midrule
      Yu et al. $_{u/s}$\cite{yu2020self}           & 15.17                             & 0.033                             & 0.133                             & 0.376           & 17.87           & 0.017            & 0.097            & 0.378           & 15.28           & 0.032            & 0.138            & 0.385           \\
      Philip et al. \cite{philip2019multi}          & 12.28                             & 0.062                             & 0.179                             & 0.319           & 16.63           & 0.023            & 0.113            & 0.367           & 12.34           & 0.065            & 0.200            & 0.272           \\
      NeRF-OSR \cite{rudnev2022nerfosr}             & 16.65                             & 0.024                             & 0.114                             & 0.501           & 18.72           & 0.014            & 0.090            & 0.468           & 15.43           & 0.029            & 0.133            & 0.517           \\
      NeRF-OSR* \cite{rudnev2022nerfosr}            & 15.66                             & 0.029                             & -                                 & -               & 19.34           & 0.012            & -                & -               & 16.35           & 0.027            & -                & -               \\
      SR-TensoRF \cite{chang2024srtensorf}          & 16.74                             & 0.024                             & 0.093                             & 0.653           & 17.30           & 0.021            & 0.096            & 0.542           & 15.43           & 0.030            & 0.111            & 0.632           \\
      FEGR \cite{wang2023fegr}                      & 17.57                             & 0.018                             & -                                 & -               & 21.53           & 0.007            & -                & -               & 17.00           & 0.023            & -                & -               \\
      SOL-NeRF \cite{solnerf}                       & 17.58                             & 0.028                             & -                                 & 0.618           & 21.23           & 0.008            & -                & 0.749           & 18.18           & 0.019            & -                & 0.680           \\
      NeuSky \cite{gardner2023neusky}               & 18.31                             & 0.016                             & -                                 & -               & 22.50           & 0.005            & -                & -               & 16.66           & 0.023            & -                & -               \\
      \textbf{Ours}                                 & 18.01                             & 0.017                             & 0.096                             & 0.778           & 19.59           & 0.012            & 0.085            & 0.700           & 17.02           & 0.021            & 0.107            & 0.729           \\
      \midrule
      Yu et al. \cite{yu2020self}                   & 15.84                             & 0.028                             & 0.123                             & 0.392           & 18.71           & 0.014            & 0.088            & 0.400           & 15.43           & 0.031            & 0.136            & 0.363           \\
      Philip et al. $_{d/s}$ \cite{philip2019multi} & 12.85                             & 0.054                             & 0.169                             & 0.164           & 17.37           & 0.019            & 0.105            & 0.429           & 11.85           & 0.070            & 0.210            & 0.184           \\
      NeRF-OSR $_{d/s}$ \cite{rudnev2022nerfosr}    & 17.38                             & 0.021                             & 0.106                             & 0.576           & 19.86           & 0.011            & 0.080            & 0.626           & 15.83           & 0.026            & 0.128            & 0.556           \\
      \textbf{Ours}~$_{d/s}$                        & 18.40                             & 0.016                             & 0.094                             & 0.746           & 20.13           & 0.011            & 0.080            & 0.727           & 17.24           & 0.020            & 0.105            & 0.715           \\
      \midrule
      $\text{\textbf{Ours}}~\dagger$                & 15.03                             & 0.034                             & 0.139                             & 0.58            & 19.34           & 0.015            & 0.094            & 0.693           & 16.09           & 0.028            & 0.124            & 0.665           \\
      $\text{\textbf{Ours}}~\ddagger$               & 17.59                             & 0.019                             & 0.100                             & 0.733           & 19.05           & 0.016            & 0.097            & 0.680           & 16.83           & 0.022            & 0.110            & 0.694           \\
      $\text{\textbf{Ours}}\setminus\loss{0-1}$     & 18.30                             & 0.016                             & 0.095                             & 0.744           & 20.15           & 0.010            & 0.080            & 0.734           & 17.25           & 0.020            & 0.105            & 0.712           \\
      $\text{\textbf{Ours}}\setminus\loss{+}$       & 17.35                             & 0.020                             & 0.104                             & 0.728           & 20.17           & 0.012            & 0.081            & 0.729           & 17.10           & 0.020            & 0.106            & 0.703           \\

      \bottomrule
    \end{tabular}
  }
  \caption{\textbf{Quantitative results --}
    Comparison between our \lumigauss and selected baselines for two
    different.
    We report the reconstruction quality regarding PSNR, MSE, MAE, SSIM on
    full and 4x downsampled image resolutions.
    \textit{u/s} denotes using upsampled, \textit{d/s} downsampled images for the evaluation, and the last delimited area presents the ablation study on downsampled data.
    We denote NeRF-OSR \cite{rudnev2022nerfosr} results reproduced by
    FEGR~\cite{wang2023fegr} with *.
    We use $\dagger$ to further annotate our approach where we remove loss
    terms
    $\loss{\removespace{\shadowed}$\leftrightarrow$\removespace{\unshadowed}}$,
    $\loss{rec}(\unshadowed)$ from the second training stage.
    In $\ddagger$, we omit the first training stage.
    Compared to the baselines, \lumigauss achieves reconstructions of high
    fidelity.
    It reliably produces smooth surfaces and sharp edges, reflected in its
    high SSIM values.
    Additionally, our proposed components either enhance reconstruction or
    preserve physical accuracy without negatively impacting the results.
    Please note, that NeuSky~\cite{gardner2023neusky} is a concurrent work,
    published prior to the WACV's deadline at ECCV 2024.
  }

  \label{tab:lumigauss-results_osr_eval}
\end{table}

    
\begin{table}[t]
  \centering
  % \setlength{\tabcolsep}{4pt} \resizebox{\linewidth}{!}{
  \begin{tabular}{lccc}
    \toprule
    \textbf{Method}                   & \textbf{Training time} & \textbf{FPS} \\
    \midrule
    NeRF-OSR \cite{rudnev2022nerfosr} & 31h                    & 0.003        \\
    NeuSky \cite{gardner2023neusky}   & 14h                    & 0.004        \\
    \textbf{Ours}                     & 1h 20min               & 20.7         \\
    \bottomrule
  \end{tabular}
  \caption{\textbf{Performance comparison --}
    Training time and inference speed comparison between the baselines and our
    \lumigauss.
  }
  % }
  \label{tab:lumigauss-performance}
\end{table}

    The applied transfer function inherently accounts for shadows and
    interreflections.
    To focus specifically on modeling shadows and restrict the use of~\cref{eq:lumigauss-shadowed-radiance} for other cases, we impose a loss function ensuring that shadowed radiance should not be brighter than unshadowed one:
    \begin{equation}
      \loss{\removespace{\shadowed}} = \mathbb{E}_k\mathbb{E}_{\direction_i}\| \max(\transferfun_k(\direction_i) -\max(\normal_k \cdot \direction_i, 0), 0) \|^2_2,
    \end{equation}
    Those losses are weighted with scalars $\{\lambda_{1,\dots,4}\}$ fixed across experiments and contribute to our regularization term:
    \begin{equation}
      \mathcal{R}(\gaussians) = \lumigausslossweight{0-1}\loss{0-1} + \lumigausslossweight{+}\loss{+} + \lumigausslossweight{\removespace{\shadowed}$\leftrightarrow$\removespace{\unshadowed}}\loss{\removespace{\shadowed}$\leftrightarrow$\removespace{\unshadowed}} +
      \lumigausslossweight{\removespace{\shadowed}}\loss{\removespace{\shadowed}}
    \end{equation}
    Calculating it exactly requires us to compute the expectation over the
    hemisphere $\mathbb{S}^2$.
    Instead, we approximate the expectations over directions $\direction_i$
    with a Monte Carlo estimator by randomly sampling the SH lobe with $N$
    samples at each training step.

  \subsection{Reconstruction}
    \label{subsec:lumigauss-reconstruction}
    We render images using the splatting algorithm $\splatting(\cdot)$
    proposed in 2DGS~\cite{huang20242d}.
    We compare the rendered images with ground-truth $\{\image_c\}$ taken with
    $\{\camera_c\}$ cameras.
    Our method builds on 2DGS~\cite{huang20242d} and therefore our reconstruction loss $\loss{rgb}$ follows the following term:
    \begin{align}
      \loss{rgb}                             & = \lumigausslossweight{rec}(\shadowed)\loss{rec}(\shadowed) + \lumigausslossweight{rec}(\unshadowed)\loss{rec}(\unshadowed), \\
      \loss{rec}(\{\shadowed, \unshadowed\}) & = \loss{1}(\{\shadowed, \unshadowed\}) + \lumigausslossweight{D-SSIM}\loss{D-SSIM}(\{\shadowed, \unshadowed\}),
    \end{align}
    where the $\loss{1}$ is the $L_1$ loss comparing either the image rendered from our shadowed or unshadowed models and $\loss{D-SSIM}$ is a differentiable D-SSIM~\cite{ssim} further improving the quality.
    We use $\lumigausslossweight{D-SSIM}{=}0.2$ throughout all the
    experiments.
    Our proposed $\loss{rec}(\unshadowed)$ resembles a pretraining stage.
    % Additionally, it keeps the shadowed model from explaining the brighter
    % regions with the inverse of the shadows (which our formulation does not
    % forbid). %We set $\lossweight{rec}$ to a small value to not bias the
    % optimization.
    As the more complex shadowed model lands in local minima if trained from
    scratch, we initiate the training with
    $\lumigausslossweight{rec}(\shadowed){=}0.0$ and
    $\lumigausslossweight{rec}(\unshadowed){=}1.0$.
    Once the simpler model converges, we switch
    $\lumigausslossweight{rec}(\shadowed){=}1.0$ and
    $\lumigausslossweight{rec}(\unshadowed)$ to a small value so as not to
    deteriorate the quality of the model.
    In short, the shadowed model explains the parts of an image with shadows,
    which the unshadowed could not with its simpler lighting model.

    % \asia{For "consistency" we also add photometric loss for unshadowed
    % version with very small weight - is there a good place to put it? Maybe
    % we can skip it? Idk, @KK any idea how to do it nicely?}

    % 4) unshadowed loss - photometric loss comes from simple unshadowed
    % version (L1 + DSSIM combination)

    % Lshadowed = Ldotproduct + Lphotometric unshadowed

    % \asia{-> not sure if needed. I am still running grid search. At first I
    % needed it, now I think we can have one stage only:} We train only
    % unshadowed version as pretraining, then we have combined version. In
    % first stage we have simpler model... 

