
\section{Related Works}
  \label{sec:lumigauss-related_works}

  \paragraph{Relighting}
    Relighting outdoor scenes is a key challenge in computer graphics and
    VR/AR.
    Early works \cite{lalonde2009webcam, troccoli2005relighting,
    haber2009relighting, xing2013lighting, sunkavalli2007factored,
    duchene2015multi, barron2014shape} used training-free methods like
    statistical inference.
    Deep learning approaches, such as Yu~\etal~\cite{yu2020self} with a neural
    renderer, and Philip~\etal~\cite{philip2019multi} with proxy geometry,
    face limitations in reconstruction quality and viewpoint flexibility.

    NeRF-based methods~\cite{mildenhall2020nerf} enabled simultaneous
    viewpoint and lighting changes.
    However, methods like~\cite{zhang2021nerfactor, zeng2023nrhints,
    srinivasan2021nerv} handle a single illumination only or specific
    illumination setup during.
    Others are object-specific, such as for faces \cite{sun2021nelf}.
    Many unconstrained photo collection methods focus on appearance, not
    lighting, complicating integration with other graphical
    components~\cite{martin2021nerfw, chen2022hallucinated, yang2023crnerf,
    li2023msnerf}.

    NeRF-based approaches, such as \cite{pun2023lightsim} and \cite{urbanir},
    focus on inverse rendering for outdoor scenes, particularly in
    applications like autonomous driving.
    However, these methods are designed for single video sequences rather than
    unstructured photo collections.
    Rudnev~\etal~\cite{rudnev2022nerfosr} proposed a method for relighting
    landmarks from unconstrained photo collections, using NeRF with external
    lighting extraction.
    Similarly, \cite{li2022neulighting} compresses the per-image illumination
    into a disentangled latent vector.
    Wang~\etal~\cite{wang2023fegr} target static scenes and works with
    unconstrained photo collections but rely on costly mesh extraction.
    Some methods incorporate additional priors, environmental assumptions, or
    regularizations~\cite{solnerf, yang2023complementary}.
    Gardner~\etal~\cite{gardner2023neusky} leverage externally trained models
    to provide environmental lighting priors.
    Despite their potential, these methods cannot be used in real-time
    applications due to NeRF’s slow training and rendering times.

    In contrast, the TensoRF-based approach by
    Chang~\etal~\cite{chang2024srtensorf} aligns time information and sun
    direction with images for relighting, eliminating the need for external
    lighting models.
    While this method is faster than NeRF, it still lacks seamless integration
    with graphics engines and is unsuitable for synthetic light integration.

    Notable Gaussian Splatting works designed for unconstrained photo
    collections \cite{kerbl20233d, dahmani2024swagsplattingwildimages,
    zhang2024gaussian, xu2024wildgs, wang2024wegs} focused on appearance
    editing, not seamless graphical component integration.
    Relightable Gaussian approaches, like \cite{gao2023relightable,
    liang2024gs, shi2023gir}, tackle material decomposition but are not
    adapted to handle varying lighting conditions of \textit{in-the-wild}
    training setup.
    Radiance transfer properties, employed in a similar way to \lumigauss, are
    utilized in \cite{zhang2024prtgaussianefficientrelightingusing,
    saito2024relightable}.
    However, these methods rely on a burdensome dataset setup, restricting
    their applicability to specific use cases.

  \paragraph{Gaussian Splatting}
    Kerbl~\etal~\cite{kerbl20233d} introduced a notion of using learnable 3D
    Gaussian primitives from point clouds.
    Those Gaussians are parametrized with 3D covariance matrix $\Sigma_k$ and their location $\translation_k$:
    \begin{equation}
      \gaussians(\translation) = \exp(\frac{1}{2}(\translation - \translation_k)^\top\Sigma_k^{-1}(\translation - \translation_k)),
    \end{equation}
    where the covariance matrix is factorized into a scaling diagonal matrix $\scaling_k$ and a rotation matrix $\rotation_k$ as $\Sigma_k{=}\rotation_k\scaling_k\scaling_k^\top\rotation_k^\top$. An image is rendered with a splatting operator $\splatting(\cdot)$ which projects Gaussians into the camera coordinates with a world-to-camera matrix and then to image plane with a local affine transformation~\cite{zwicker2001ewa}:
    \begin{equation}
      \label{eq:lumigauss-splatting-op}
      \splatting(\camera_c\;|\;\gaussians) = \sum_{k=1}^K\radiance_k\opacity_k\gaussians_k\prod^{k-1}_{j=1}(1 - \opacity_j \gaussians_k).
    \end{equation}
    The operator produces an RGB image, given a calibrated camera matrix
    $\camera_c$ and their additional Gaussians' attributes: their colors
    $\radiance$ and opacities $\opacity$.
    Attributes are learned using a stochastic gradient descent.

    Huan~\etal~\cite{huang20242d} argues that 3DGS although producing
    high-quality images, the implicit surface representation is noisy,
    limiting its applicability in relighting scenarios.
    They propose using 2D Gaussians instead to create smooth, coherent meshes
    thanks to their exact 2D surfel projection.
    We leverage that representation in our \lumigauss---a relightable model
    that decouples albedo, environment light and shadows thanks to our
    proposed physical constraints.
