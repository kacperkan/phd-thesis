
\section{Dataset Processing}

  \textbf{Occluders.
  }
  To exclude occluders from training images we use masks provided with OSR
  dataset~\cite{rudnev2022nerfosr}.

  \textbf{Test set.}
  We test our approach on 5 viewpoints for each scene, as it was originally
  proposed in \cite{rudnev2022nerfosr}.
  For testing, we use test masks provided by \cite{rudnev2022nerfosr} and we
  stricly follow their evaluation protocol.
  For SSIM, we report the average value over the segmentation mask, utilizing
  the scikit-image implementation with a window size of 5 and eroding the
  segmentation mask by the same window size to exclude the influence of pixels
  outside the mask on the metric value.

  \textbf{Testing with ground truth environment map.}
  The authors of \cite{gardner2023neusky} made an effort to recover steps for
  environment map preprocessing and alignment.
  The preprocessing step is available in their repository, accessible at:
  \url{https://github.com/JADGardner/neusky/blob/main/notebooks/nerfosr_envmaps.ipynb}.
  The detailed discussion on SOL-NeRF \cite{solnerf} approach to environment
  map alignment is included in the NeuSky main paper \cite{gardner2023neusky}
  and also confirmed with SOL-NeRF authors.

\section{Implementation details}

  The appearance embedding vector is set to a size of 24 dimensions.
  For predicting the environment map, we use MLP with 3 fully-connected layers
  of size 64.
  We trained all models for 40000 iterations, the first training stage is set
  to 20000 iterations.
  The learning rate for MLP and embedding is set to 0.002, which after first
  training stage is reduced to 0.0002.
  We train gaussian spherical harmonics with a learning rate of 0.002.
  We set the loss function weights as follows: for
  $\lumigausslossweight{0-1}=0.001$, for $\lumigausslossweight{+}=0.05$, for
  $\lumigausslossweight{\removespace{\shadowed}$\leftrightarrow$\removespace{\unshadowed}}
  \in \{1.0, 10.0\}$, for
  $\lumigausslossweight{\removespace{\shadowed}}=10.0$.
  In the second training stage we set
  $\lumigausslossweight{\removespace{\shadowed}}=0.001$.

  We adhere to the original Gaussian splatting densification and pruning
  protocols, with a densification interval of 500 iterations and an opacity
  reset interval of 3000 iterations.
  We apply regularizations to align Gaussians with surfaces, as originally
  described in \cite{huang20242d}.
  Additionally, we utilize the dual visibility concept proposed in
  \cite{huang20242d}.
  This ensures that the Gaussians are always correctly oriented towards the
  camera.
  Dual visibility effectively produces consistent world normals, with visible
  normals being consistent and non-visible ones contributing minimally to the
  rendering.
  Regularization of Spherical Harmonics $\mathbf{d}_{k}$ is dependent on
  gaussian normals.
  Since normals are rotated to always face the camera, to maintain alignment
  between each Gaussian's normal and its associated \( \mathbf{d}_k \), we
  also rotate \( \mathbf{d}_k \) accordingly.

  % \begin{figure*}[!b] \centering
% \includegraphics[width=\linewidth]{images/folder_for_kacper/qual_gt_shadows/wild2d-qual_gt_shadows.pdf}
% \caption{ \textbf{\change{imo to nie wyglada ladnie. Moze zostawic tylko osr?
% myslalam, zeby zrobic porownanie cieni swiatlem punktowym, ale nie mam chyba
% na to sily. Moze wysłać to do appendixu} Qualitative comparison of renderings
% with gt env map for the same phot session.} Results for NERF-OSR, ST-TensoRF
% reported originally in \cite{chang2024srtensorf}.}
% \label{fig:qual_gt_shadows} \end{figure*}
\renewcommand{\inputimage}[1]{\includegraphics[height=\height,trim={8bp 0 8bp 0}, clip]{assets/5_lumigauss/folder_for_kacper/qual_relight_albedo_norm/#1}}
\renewcommand{\inputimageshadows}[1]{\includegraphics[height=\height,trim={8bp 0 8bp 0}, clip]{assets/5_lumigauss/folder_for_kacper/qual_gt_shadows/#1}}

\newcommand{\shadowsfigure}{
  \centering
  \tikzset{external/export next=false}
  \resizebox{0.5\linewidth}{!}{
    {
        \def\height{64bp}
        \def\spysize{24bp}
        \def\offset{36bp}
        \def\scale{0.8}
        \def\scaleleft{0.8}
        \def\distabove{0.5em}
        \def\distleft{0.0em}
        \def\clipleft{2cm}
        \def\clipright{2cm}
        \begin{tikzpicture}[
            >=stealth',
            title/.style={
                anchor=base,
                align=center,
                scale=\scale,
              },
            spy using outlines={circle, red, magnification=3, connect spies, size=\spysize}
          ]

          \matrix[
            matrix of nodes,
            anchor=west,
            column sep=0pt,
            row sep=0pt,
            ampersand replacement=\&,
            inner sep=0,
            outer sep=0,
          ] (shadows) {
            \inputimageshadows{osr-shadows-albedo.png} \&
            \inputimageshadows{osr-shadows.png}       \\
            \inputimageshadows{sr-shadows-albedo.png} \&
            \inputimageshadows{sr-shadows.png}        \\
            \inputimageshadows{lumigauss-shadows-albedo.png} \&
            \inputimageshadows{lumigauss-shadows.png} \\
            \inputimageshadows{gt-shadows.png}        \\
          };

          \node[above=\distabove of shadows-1-1.north, title]{Relit Reconstruction};
          \node[above=\distabove of shadows-1-2.north, title]{Predicted Shadows};

          \node[left=\distleft of shadows-1-1.west, align=center,scale=\scaleleft]{\rotatebox{90}{NeRF-OSR~\cite{rudnev2022nerfosr}}};
          \node[left=\distleft of shadows-2-1.west, align=center,scale=\scaleleft]{\rotatebox{90}{SR-TensorRF~\cite{chang2024srtensorf}}};
          \node[left=\distleft of shadows-3-1.west, align=center,scale=\scaleleft]{\rotatebox{90}{\textbf{Ours}}};
          \node[left=\distleft of shadows-4-1.west, align=center,scale=\scaleleft]{\rotatebox{90}{Ground Truth}};

          \foreach \y in {1,...,3}{
              \foreach \x in {1,2}{
                  \edef\temp{\noexpand\spy[anchor=center,magnification=2] on ([xshift=-1.6em,yshift=-0.7em] shadows-\y-\x.center) in node[below left=3.75bp and 3.75bp] at (shadows-\y-\x.north east);}
                  \temp
                }
            }
          \spy[anchor=center,magnification=2] on ([xshift=-1.6em,yshift=-0.7em] shadows-4-1.center) in node[below left=3.75bp and 3.75bp] at (shadows-4-1.north east);
        \end{tikzpicture}
      }
  }
}

\begin{figure}[!t]
  \centering
  \shadowsfigure
  \caption{ \textbf{Qualitative comparison of scene reconstruction for the selected photo session -- }
    Results for NeRF-OSR~\cite{rudnev2022nerfosr} and \lumigauss were
    generated using ground truth enviroment maps for selected photo session,
    while ST-TensoRF~\cite{chang2024srtensorf} used extracted timestamp.
    Results for NeRF-OSR and SR-TensorF reported originally in
    \cite{chang2024srtensorf}.
  }
  \label{fig:lumigauss-qual_gt_shadows}
\end{figure}
\section{Qualitative comparison - additional results}
  In~\cref{fig:lumigauss-qual_gt_shadows} we show the qualitative comparison
  of our method, NeRF-OSR, and SR-TensoRF.
  We show the landmark relit with ground truth envoronment map for NeRF-OSR
  and \lumigauss.
  SR-TensoRF reconstructs ground truth using only daytime (timestamp).

  In~\cref{fig:lumigauss-qualitative_appendix}, we show the qualitative
  comparison of our method, NeRF-OSR, and SR-TensoRF.
  We use the \textit{default synthetic} environment map provided by
  \cite{rudnev2022nerfosr}.
  This environment map was used for visualisation purposes in
  \cite{chang2024srtensorf}.
  We use it to ensure a fair comparison and consistency with results from
  concurrent works.
  We also present albedo and normals extracted from the reconstructed scene.
  Please note that our model produces much cleaner results.
  Compared to the baselines, it reconstructs sharp features in small elements
  of the buildings, which is also reflected in the quantitative
  results~\cref{tab:lumigauss-results_osr_eval}.
  \lumigauss also gracefully smooths out the elements of scenes that are variable across the images, such as trees and clouds.
  On the other hand, NeRF-OSR and SR-TensoRF produce artifacts that negatively
  impact the output reconstructions.

  In \cref{fig:lumigauss-qual_normal_albedo_appendix} we present additional
  comparison with concurrent works.
  We focus on normal and albedo quality.
  
\begin{figure}[t]
  \centering
  \includegraphics[width=\linewidth]{\lumigaussassets/folder_for_kacper/normals_albedo_comp_appendix/wild2d-albedo_normals_comp_text.pdf}
  \caption{ \textbf{Qualitative comparison of predicted albedo and rendered normals.}
    Results for NERF-OSR, FEGR, SOL-NERF, NeuSky reported originally in
    \cite{gardner2023neusky}.
  }
  \label{fig:lumigauss-qual_normal_albedo_appendix}
\end{figure}

  In \cref{fig:lumigauss-qual_more_viewpoints_supp_img} we present additional
  results of novel view synthesis and comparison with concurrent works.
  Similarly to NeRF-OSR, we relight our scenes with the \textbf{default
  synthetic} map provided by NeRF-OSR for visualization purposes.
  This environment map does not correspond to GT images.

\section{Ablation study -- additional results }
  In \cref{fig:lumigauss-ablation_img} we present renders from training
  without selected regularization terms.

  \renewcommand{\inputimage}[1]{\includegraphics[height=\height]{\lumigaussassets/folder_for_kacper/qual/#1}}

\newcommand{\qualitative}{
  \centering
  \tikzset{external/export next=false}
  \tikzsetnextfilename{qualitative_appendix}
  {
    \def\scale{1.0}
    \def\height{64bp}
    \def\distabove{0.5em}
    \def\spysize{24bp}
    \def\offset{36bp}
    \resizebox{\linewidth}{!}{
      \begin{tikzpicture}[
          >=stealth',
          title/.style={
              anchor=base,
              align=center,
              scale=\scale,
            },
          spy using outlines={circle, red, magnification=3, connect spies, size=\spysize}
        ]
        \matrix[matrix of nodes, column sep=0pt, row sep=0pt, ampersand replacement=\&, inner sep=0, outer sep=0, font=\Huge] (qualitative) {
          \inputimage{gt_1.jpg} \&
          \inputimage{ours_1_relit.png} \&
          \inputimage{osr_1.jpg} \&
          \inputimage{srtensor_1.png} \\
          \inputimage{gt_2.jpg} \&
          \inputimage{ours_2_relit.png} \&
          \inputimage{osr_2.jpg} \&
          \inputimage{srtensor_2.png} \\
          \inputimage{gt_3.jpg} \&
          \inputimage{ours_3_relit.png} \&
          \inputimage{osr_3.jpg} \&
          \inputimage{srtensor_3.png} \\
        };

        \matrix[
          right=16bp of qualitative.east,
          matrix of nodes,
          anchor=west,
          column sep=0pt,
          row sep=0pt,
          ampersand replacement=\&,
          inner sep=0,
          outer sep=0,
        ] (ours) {
          \inputimage{ours_1_albedo.png} \&
          \inputimage{ours_1_normals.png} \\
          \inputimage{ours_2_albedo.png} \&
          \inputimage{ours_2_normals.png} \\
          \inputimage{ours_3_albedo.png} \&
          \inputimage{ours_3_normals.png} \\
        };

        \node[above=\distabove of qualitative-1-1.north, title] {Ground Truth};
        \node[above=\distabove of qualitative-1-2.north, title] {\textbf{Ours}};
        \node[above=\distabove of qualitative-1-3.north, title] {NeRF-OSR~\cite{rudnev2022nerfosr}};
        \node[above=\distabove of qualitative-1-4.north, title] {SRTensoRF~\cite{chang2024srtensorf}};

        \node[above=\distabove of ours-1-1.north, title] {\textbf{Ours} Albedo};
        \node[above=\distabove of ours-1-2.north, title] {\textbf{Ours} Normals};

        \spy[anchor=center] on ([xshift=1em] qualitative-1-1.center) in node[below left=3.75bp and 3.75bp] at (qualitative-1-1.north east);
        \spy[anchor=center] on ([xshift=1em] qualitative-1-2.center) in node[below left=3.75bp and 3.75bp] at (qualitative-1-2.north east);
        \spy[anchor=center] on ([xshift=1em] qualitative-1-3.center) in node[below left=3.75bp and 3.75bp] at (qualitative-1-3.north east);
        \spy[anchor=center] on ([xshift=1em] qualitative-1-4.center) in node[below left=3.75bp and 3.75bp] at (qualitative-1-4.north east);

        \spy[anchor=center] on ([xshift=-1em] qualitative-2-1.center) in node[below left=3.75bp and 3.75bp] at (qualitative-2-1.north east);
        \spy[anchor=center] on ([xshift=-1em] qualitative-2-2.center) in node[below left=3.75bp and 3.75bp] at (qualitative-2-2.north east);
        \spy[anchor=center] on ([xshift=-1em] qualitative-2-3.center) in node[below left=3.75bp and 3.75bp] at (qualitative-2-3.north east);
        \spy[anchor=center] on ([xshift=-1em] qualitative-2-4.center) in node[below left=3.75bp and 3.75bp] at (qualitative-2-4.north east);

        \spy[anchor=center] on ([xshift=-1em] qualitative-3-1.center) in node[below left=3.75bp and 3.75bp] at (qualitative-3-1.north east);
        \spy[anchor=center] on ([xshift=-1em] qualitative-3-2.center) in node[below left=3.75bp and 3.75bp] at (qualitative-3-2.north east);
        \spy[anchor=center] on ([xshift=-1em] qualitative-3-3.center) in node[below left=3.75bp and 3.75bp] at (qualitative-3-3.north east);
        \spy[anchor=center] on ([xshift=-1em] qualitative-3-4.center) in node[below left=3.75bp and 3.75bp] at (qualitative-3-4.north east);
      \end{tikzpicture}
    }
  }
}

\begin{figure}[!t]
  \centering
  \qualitative
  \caption{
    \textbf{Qualitative results --}
    Showcase of novel view synthesis using shadowed radiance transfer.
    We present albedo and normals produced by our method.
    Our method generates much sharper renderings.
    Please see zoom-ins to see details on the quality difference, such as
    surface smoothness and edge sharpness of small building elements.
    We use visual results for SR-TensoRF and NeRF-OSR presented originally
    in~\cite{chang2024srtensorf}.
    Please note that, in this comparison the environment map used to create
    renders for NeRF-OSR and \lumigauss \textbf{does not match} the
    illumination in ground truth.
    \lumigauss and NeRF-OSR employ the \textbf{default} environment map provided by NeRF-OSR \textbf{for clear visualisation purpose only}.
    SR-TensoRF do not rely on any environment map, instead it utilizes daytime
    information.
  }

  \label{fig:lumigauss-qualitative_appendix}
\end{figure}

  
\begin{table*}[!t]
  \centering
  \resizebox{\linewidth}{!}{
    \begin{tabular}{lccccccccc}
      \toprule
      \multirow{3}[3]{*}{Parameter}    & \multicolumn{6}{c}{Real Data}          & \multicolumn{3}{c}{Synthetic Data}                                                                                                                                                                                                                                                                            \\
      \cmidrule(lr){2-7}\cmidrule(lr){8-10}
                                       & \multicolumn{3}{c}{Casual Expressions} & \multicolumn{3}{c}{Novel Pose Synthesis} & \multicolumn{3}{c}{Novel Pose Synthesis}                                                                                                                                                                                                                           \\
      \cmidrule(lr){2-4}\cmidrule(lr){5-7}\cmidrule(lr){8-10}
                                       & PSNR $\uparrow$                        & SSIM $\uparrow$                          & LPIPS $\downarrow$                       & PSNR $\uparrow$                    & SSIM $\uparrow$                   & LPIPS $\downarrow$                & PSNR $\uparrow$                    & SSIM $\uparrow$                   & LPIPS $\downarrow$                \\
      \midrule
      $|\neighbourhood(\vertex)| = 1$  & 27.5620                                & 0.9043                                   & 0.0893                                   & 29.7269                            & 0.9306                            & 0.0815                            & 32.2371                            & \cellcolor{secondbestcolor}0.9882 & 0.0234                            \\
      $|\neighbourhood(\vertex)| = 5$  & 27.5880                                & \cellcolor{secondbestcolor}0.9054        & 0.0864                                   & \cellcolor{firstbestcolor}29.7548  & \cellcolor{secondbestcolor}0.9312 & 0.0789                            & 32.2900                            & \cellcolor{secondbestcolor}0.9882 & 0.0231                            \\
      $|\neighbourhood(\vertex)| = 10$ & \cellcolor{secondbestcolor}27.5933     & \cellcolor{secondbestcolor}0.9054        & \cellcolor{secondbestcolor}0.0859        & \cellcolor{secondbestcolor}29.7456 & \cellcolor{secondbestcolor}0.9312 & \cellcolor{secondbestcolor}0.0785 & \cellcolor{secondbestcolor}32.3324 & \cellcolor{secondbestcolor}0.9882 & \cellcolor{secondbestcolor}0.0230 \\
      $|\neighbourhood(\vertex)| = 20$ & \cellcolor{firstbestcolor}27.5977      & \cellcolor{firstbestcolor}0.9056         & \cellcolor{firstbestcolor}0.0854         & 29.7372                            & 0.9311                            & \cellcolor{firstbestcolor}0.0782  & \cellcolor{firstbestcolor}32.7949  & \cellcolor{firstbestcolor}0.9887  & \cellcolor{firstbestcolor}0.0221  \\
      \midrule
      Without smoothing                & \cellcolor{secondbestcolor}27.2535     & \cellcolor{secondbestcolor}0.8959        & \cellcolor{secondbestcolor}0.0939        & \cellcolor{secondbestcolor}29.3726 & \cellcolor{secondbestcolor}0.9233 & \cellcolor{secondbestcolor}0.0846 & \cellcolor{secondbestcolor}32.2452 & \cellcolor{secondbestcolor}0.9876 & \cellcolor{secondbestcolor}0.0238 \\
      With smoothing                   & \cellcolor{firstbestcolor}27.5977      & \cellcolor{firstbestcolor}0.9056         & \cellcolor{firstbestcolor}0.0854         & \cellcolor{firstbestcolor}29.7372  & \cellcolor{firstbestcolor}0.9311  & \cellcolor{firstbestcolor}0.0782  & \cellcolor{firstbestcolor}32.7949  & \cellcolor{firstbestcolor}0.9887  & \cellcolor{firstbestcolor}0.0221  \\
      \bottomrule
    \end{tabular}
  }
  \caption{\textbf{Ablation study} -- {
  First, we check the effect of the neighborhood size $|\neighbourhood(\vertex)|$ on the results.
  Below that, we compare the effect of smoothing.
  % We highlight ablations results as its the only benchmark where the
  % deformable model is fit reliably. 
  The best results are colored in \mycoloredbox{firstbestcolor} and the second
  best in \mycoloredbox{secondbestcolor}.
  For the real dataset, changing the neighborhood size gives inconsistent
  results, while smoothing improves the rendering quality.
  In the synthetic scenario, setting $|\neighbourhood(\vertex)|{=}20$ and the
  Laplacian smoothing consistently gives the best results.
  The discrepancy between real and synthetic datasets is caused by inaccurate
  face tracking for the former.
  We describe this issue in detail in~\cref{subsec:blendfields-failures}.
  }
  }
  \label{tab:blendfields-ablation-study}
\end{table*}

  
\begin{figure}[!t]
  \centering
  \includegraphics[width=0.9\linewidth]{\lumigaussassets/wild2d-app_qual.pdf}
  \caption{ \textbf{Qualitative comparison --}
    Additional novel viewpoints.
    Results for NeRF-OSR and SR-TensoRF originally reported in
    \cite{chang2024srtensorf}.
    Please note, in this comparison renders for NeRF-OSR and \lumigauss
    \textbf{do not have to} reconstruct ground truth.
  }
  \label{fig:lumigauss-qual_more_viewpoints_supp_img}
\end{figure}
