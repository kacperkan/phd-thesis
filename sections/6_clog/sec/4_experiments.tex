
\begin{figure}[t]
  \centering
  \resizebox{0.8\linewidth}{!}{
    %% Creator: Matplotlib, PGF backend
%%
%% To include the figure in your LaTeX document, write
%%   \input{<filename>.pgf}
%%
%% Make sure the required packages are loaded in your preamble
%%   \usepackage{pgf}
%%
%% Also ensure that all the required font packages are loaded; for instance,
%% the lmodern package is sometimes necessary when using math font.
%%   \usepackage{lmodern}
%%
%% Figures using additional raster images can only be included by \input if
%% they are in the same directory as the main LaTeX file. For loading figures
%% from other directories you can use the `import` package
%%   \usepackage{import}
%%
%% and then include the figures with
%%   \import{<path to file>}{<filename>.pgf}
%%
%% Matplotlib used the following preamble
%%   \def\mathdefault#1{#1}
%%   \everymath=\expandafter{\the\everymath\displaystyle}
%%   
%%   \ifdefined\pdftexversion\else  % non-pdftex case.
%%     \usepackage{fontspec}
%%   \fi
%%   \makeatletter\@ifpackageloaded{underscore}{}{\usepackage[strings]{underscore}}\makeatother
%%
\begingroup%
\makeatletter%
\begin{pgfpicture}%
\pgfpathrectangle{\pgfpointorigin}{\pgfqpoint{6.000000in}{4.000000in}}%
\pgfusepath{use as bounding box}%
\begin{pgfscope}%
\pgfsetbuttcap%
\pgfsetmiterjoin%
\definecolor{currentfill}{rgb}{1.000000,1.000000,1.000000}%
\pgfsetfillcolor{currentfill}%
\pgfsetlinewidth{0.000000pt}%
\definecolor{currentstroke}{rgb}{1.000000,1.000000,1.000000}%
\pgfsetstrokecolor{currentstroke}%
\pgfsetdash{}{0pt}%
\pgfpathmoveto{\pgfqpoint{0.000000in}{0.000000in}}%
\pgfpathlineto{\pgfqpoint{6.000000in}{0.000000in}}%
\pgfpathlineto{\pgfqpoint{6.000000in}{4.000000in}}%
\pgfpathlineto{\pgfqpoint{0.000000in}{4.000000in}}%
\pgfpathlineto{\pgfqpoint{0.000000in}{0.000000in}}%
\pgfpathclose%
\pgfusepath{fill}%
\end{pgfscope}%
\begin{pgfscope}%
\pgfsetbuttcap%
\pgfsetmiterjoin%
\definecolor{currentfill}{rgb}{1.000000,1.000000,1.000000}%
\pgfsetfillcolor{currentfill}%
\pgfsetlinewidth{0.000000pt}%
\definecolor{currentstroke}{rgb}{0.000000,0.000000,0.000000}%
\pgfsetstrokecolor{currentstroke}%
\pgfsetstrokeopacity{0.000000}%
\pgfsetdash{}{0pt}%
\pgfpathmoveto{\pgfqpoint{0.750000in}{0.440000in}}%
\pgfpathlineto{\pgfqpoint{5.400000in}{0.440000in}}%
\pgfpathlineto{\pgfqpoint{5.400000in}{3.520000in}}%
\pgfpathlineto{\pgfqpoint{0.750000in}{3.520000in}}%
\pgfpathlineto{\pgfqpoint{0.750000in}{0.440000in}}%
\pgfpathclose%
\pgfusepath{fill}%
\end{pgfscope}%
\begin{pgfscope}%
\pgfpathrectangle{\pgfqpoint{0.750000in}{0.440000in}}{\pgfqpoint{4.650000in}{3.080000in}}%
\pgfusepath{clip}%
\pgfsetroundcap%
\pgfsetroundjoin%
\pgfsetlinewidth{1.003750pt}%
\definecolor{currentstroke}{rgb}{0.800000,0.800000,0.800000}%
\pgfsetstrokecolor{currentstroke}%
\pgfsetdash{}{0pt}%
\pgfpathmoveto{\pgfqpoint{1.348672in}{0.440000in}}%
\pgfpathlineto{\pgfqpoint{1.348672in}{3.520000in}}%
\pgfusepath{stroke}%
\end{pgfscope}%
\begin{pgfscope}%
\definecolor{textcolor}{rgb}{0.150000,0.150000,0.150000}%
\pgfsetstrokecolor{textcolor}%
\pgfsetfillcolor{textcolor}%
\pgftext[x=1.348672in,y=0.308056in,,top]{\color{textcolor}{\rmfamily\fontsize{13.750000}{16.500000}\selectfont\catcode`\^=\active\def^{\ifmmode\sp\else\^{}\fi}\catcode`\%=\active\def%{\%}$\mathdefault{10^{2}}$}}%
\end{pgfscope}%
\begin{pgfscope}%
\pgfpathrectangle{\pgfqpoint{0.750000in}{0.440000in}}{\pgfqpoint{4.650000in}{3.080000in}}%
\pgfusepath{clip}%
\pgfsetroundcap%
\pgfsetroundjoin%
\pgfsetlinewidth{1.003750pt}%
\definecolor{currentstroke}{rgb}{0.800000,0.800000,0.800000}%
\pgfsetstrokecolor{currentstroke}%
\pgfsetdash{}{0pt}%
\pgfpathmoveto{\pgfqpoint{2.251755in}{0.440000in}}%
\pgfpathlineto{\pgfqpoint{2.251755in}{3.520000in}}%
\pgfusepath{stroke}%
\end{pgfscope}%
\begin{pgfscope}%
\definecolor{textcolor}{rgb}{0.150000,0.150000,0.150000}%
\pgfsetstrokecolor{textcolor}%
\pgfsetfillcolor{textcolor}%
\pgftext[x=2.251755in,y=0.308056in,,top]{\color{textcolor}{\rmfamily\fontsize{13.750000}{16.500000}\selectfont\catcode`\^=\active\def^{\ifmmode\sp\else\^{}\fi}\catcode`\%=\active\def%{\%}$\mathdefault{10^{3}}$}}%
\end{pgfscope}%
\begin{pgfscope}%
\pgfpathrectangle{\pgfqpoint{0.750000in}{0.440000in}}{\pgfqpoint{4.650000in}{3.080000in}}%
\pgfusepath{clip}%
\pgfsetroundcap%
\pgfsetroundjoin%
\pgfsetlinewidth{1.003750pt}%
\definecolor{currentstroke}{rgb}{0.800000,0.800000,0.800000}%
\pgfsetstrokecolor{currentstroke}%
\pgfsetdash{}{0pt}%
\pgfpathmoveto{\pgfqpoint{3.154838in}{0.440000in}}%
\pgfpathlineto{\pgfqpoint{3.154838in}{3.520000in}}%
\pgfusepath{stroke}%
\end{pgfscope}%
\begin{pgfscope}%
\definecolor{textcolor}{rgb}{0.150000,0.150000,0.150000}%
\pgfsetstrokecolor{textcolor}%
\pgfsetfillcolor{textcolor}%
\pgftext[x=3.154838in,y=0.308056in,,top]{\color{textcolor}{\rmfamily\fontsize{13.750000}{16.500000}\selectfont\catcode`\^=\active\def^{\ifmmode\sp\else\^{}\fi}\catcode`\%=\active\def%{\%}$\mathdefault{10^{4}}$}}%
\end{pgfscope}%
\begin{pgfscope}%
\pgfpathrectangle{\pgfqpoint{0.750000in}{0.440000in}}{\pgfqpoint{4.650000in}{3.080000in}}%
\pgfusepath{clip}%
\pgfsetroundcap%
\pgfsetroundjoin%
\pgfsetlinewidth{1.003750pt}%
\definecolor{currentstroke}{rgb}{0.800000,0.800000,0.800000}%
\pgfsetstrokecolor{currentstroke}%
\pgfsetdash{}{0pt}%
\pgfpathmoveto{\pgfqpoint{4.057921in}{0.440000in}}%
\pgfpathlineto{\pgfqpoint{4.057921in}{3.520000in}}%
\pgfusepath{stroke}%
\end{pgfscope}%
\begin{pgfscope}%
\definecolor{textcolor}{rgb}{0.150000,0.150000,0.150000}%
\pgfsetstrokecolor{textcolor}%
\pgfsetfillcolor{textcolor}%
\pgftext[x=4.057921in,y=0.308056in,,top]{\color{textcolor}{\rmfamily\fontsize{13.750000}{16.500000}\selectfont\catcode`\^=\active\def^{\ifmmode\sp\else\^{}\fi}\catcode`\%=\active\def%{\%}$\mathdefault{10^{5}}$}}%
\end{pgfscope}%
\begin{pgfscope}%
\pgfpathrectangle{\pgfqpoint{0.750000in}{0.440000in}}{\pgfqpoint{4.650000in}{3.080000in}}%
\pgfusepath{clip}%
\pgfsetroundcap%
\pgfsetroundjoin%
\pgfsetlinewidth{1.003750pt}%
\definecolor{currentstroke}{rgb}{0.800000,0.800000,0.800000}%
\pgfsetstrokecolor{currentstroke}%
\pgfsetdash{}{0pt}%
\pgfpathmoveto{\pgfqpoint{4.961004in}{0.440000in}}%
\pgfpathlineto{\pgfqpoint{4.961004in}{3.520000in}}%
\pgfusepath{stroke}%
\end{pgfscope}%
\begin{pgfscope}%
\definecolor{textcolor}{rgb}{0.150000,0.150000,0.150000}%
\pgfsetstrokecolor{textcolor}%
\pgfsetfillcolor{textcolor}%
\pgftext[x=4.961004in,y=0.308056in,,top]{\color{textcolor}{\rmfamily\fontsize{13.750000}{16.500000}\selectfont\catcode`\^=\active\def^{\ifmmode\sp\else\^{}\fi}\catcode`\%=\active\def%{\%}$\mathdefault{10^{6}}$}}%
\end{pgfscope}%
\begin{pgfscope}%
\definecolor{textcolor}{rgb}{0.150000,0.150000,0.150000}%
\pgfsetstrokecolor{textcolor}%
\pgfsetfillcolor{textcolor}%
\pgftext[x=3.075000in,y=0.080416in,,top]{\color{textcolor}{\rmfamily\fontsize{15.000000}{18.000000}\selectfont\catcode`\^=\active\def^{\ifmmode\sp\else\^{}\fi}\catcode`\%=\active\def%{\%}\# Gaussians}}%
\end{pgfscope}%
\begin{pgfscope}%
\pgfpathrectangle{\pgfqpoint{0.750000in}{0.440000in}}{\pgfqpoint{4.650000in}{3.080000in}}%
\pgfusepath{clip}%
\pgfsetroundcap%
\pgfsetroundjoin%
\pgfsetlinewidth{1.003750pt}%
\definecolor{currentstroke}{rgb}{0.800000,0.800000,0.800000}%
\pgfsetstrokecolor{currentstroke}%
\pgfsetdash{}{0pt}%
\pgfpathmoveto{\pgfqpoint{0.750000in}{0.623901in}}%
\pgfpathlineto{\pgfqpoint{5.400000in}{0.623901in}}%
\pgfusepath{stroke}%
\end{pgfscope}%
\begin{pgfscope}%
\definecolor{textcolor}{rgb}{0.150000,0.150000,0.150000}%
\pgfsetstrokecolor{textcolor}%
\pgfsetfillcolor{textcolor}%
\pgftext[x=0.269912in, y=0.557633in, left, base]{\color{textcolor}{\rmfamily\fontsize{13.750000}{16.500000}\selectfont\catcode`\^=\active\def^{\ifmmode\sp\else\^{}\fi}\catcode`\%=\active\def%{\%}$\mathdefault{17.5}$}}%
\end{pgfscope}%
\begin{pgfscope}%
\pgfpathrectangle{\pgfqpoint{0.750000in}{0.440000in}}{\pgfqpoint{4.650000in}{3.080000in}}%
\pgfusepath{clip}%
\pgfsetroundcap%
\pgfsetroundjoin%
\pgfsetlinewidth{1.003750pt}%
\definecolor{currentstroke}{rgb}{0.800000,0.800000,0.800000}%
\pgfsetstrokecolor{currentstroke}%
\pgfsetdash{}{0pt}%
\pgfpathmoveto{\pgfqpoint{0.750000in}{1.075158in}}%
\pgfpathlineto{\pgfqpoint{5.400000in}{1.075158in}}%
\pgfusepath{stroke}%
\end{pgfscope}%
\begin{pgfscope}%
\definecolor{textcolor}{rgb}{0.150000,0.150000,0.150000}%
\pgfsetstrokecolor{textcolor}%
\pgfsetfillcolor{textcolor}%
\pgftext[x=0.269912in, y=1.008890in, left, base]{\color{textcolor}{\rmfamily\fontsize{13.750000}{16.500000}\selectfont\catcode`\^=\active\def^{\ifmmode\sp\else\^{}\fi}\catcode`\%=\active\def%{\%}$\mathdefault{20.0}$}}%
\end{pgfscope}%
\begin{pgfscope}%
\pgfpathrectangle{\pgfqpoint{0.750000in}{0.440000in}}{\pgfqpoint{4.650000in}{3.080000in}}%
\pgfusepath{clip}%
\pgfsetroundcap%
\pgfsetroundjoin%
\pgfsetlinewidth{1.003750pt}%
\definecolor{currentstroke}{rgb}{0.800000,0.800000,0.800000}%
\pgfsetstrokecolor{currentstroke}%
\pgfsetdash{}{0pt}%
\pgfpathmoveto{\pgfqpoint{0.750000in}{1.526415in}}%
\pgfpathlineto{\pgfqpoint{5.400000in}{1.526415in}}%
\pgfusepath{stroke}%
\end{pgfscope}%
\begin{pgfscope}%
\definecolor{textcolor}{rgb}{0.150000,0.150000,0.150000}%
\pgfsetstrokecolor{textcolor}%
\pgfsetfillcolor{textcolor}%
\pgftext[x=0.269912in, y=1.460147in, left, base]{\color{textcolor}{\rmfamily\fontsize{13.750000}{16.500000}\selectfont\catcode`\^=\active\def^{\ifmmode\sp\else\^{}\fi}\catcode`\%=\active\def%{\%}$\mathdefault{22.5}$}}%
\end{pgfscope}%
\begin{pgfscope}%
\pgfpathrectangle{\pgfqpoint{0.750000in}{0.440000in}}{\pgfqpoint{4.650000in}{3.080000in}}%
\pgfusepath{clip}%
\pgfsetroundcap%
\pgfsetroundjoin%
\pgfsetlinewidth{1.003750pt}%
\definecolor{currentstroke}{rgb}{0.800000,0.800000,0.800000}%
\pgfsetstrokecolor{currentstroke}%
\pgfsetdash{}{0pt}%
\pgfpathmoveto{\pgfqpoint{0.750000in}{1.977671in}}%
\pgfpathlineto{\pgfqpoint{5.400000in}{1.977671in}}%
\pgfusepath{stroke}%
\end{pgfscope}%
\begin{pgfscope}%
\definecolor{textcolor}{rgb}{0.150000,0.150000,0.150000}%
\pgfsetstrokecolor{textcolor}%
\pgfsetfillcolor{textcolor}%
\pgftext[x=0.269912in, y=1.911404in, left, base]{\color{textcolor}{\rmfamily\fontsize{13.750000}{16.500000}\selectfont\catcode`\^=\active\def^{\ifmmode\sp\else\^{}\fi}\catcode`\%=\active\def%{\%}$\mathdefault{25.0}$}}%
\end{pgfscope}%
\begin{pgfscope}%
\pgfpathrectangle{\pgfqpoint{0.750000in}{0.440000in}}{\pgfqpoint{4.650000in}{3.080000in}}%
\pgfusepath{clip}%
\pgfsetroundcap%
\pgfsetroundjoin%
\pgfsetlinewidth{1.003750pt}%
\definecolor{currentstroke}{rgb}{0.800000,0.800000,0.800000}%
\pgfsetstrokecolor{currentstroke}%
\pgfsetdash{}{0pt}%
\pgfpathmoveto{\pgfqpoint{0.750000in}{2.428928in}}%
\pgfpathlineto{\pgfqpoint{5.400000in}{2.428928in}}%
\pgfusepath{stroke}%
\end{pgfscope}%
\begin{pgfscope}%
\definecolor{textcolor}{rgb}{0.150000,0.150000,0.150000}%
\pgfsetstrokecolor{textcolor}%
\pgfsetfillcolor{textcolor}%
\pgftext[x=0.269912in, y=2.362661in, left, base]{\color{textcolor}{\rmfamily\fontsize{13.750000}{16.500000}\selectfont\catcode`\^=\active\def^{\ifmmode\sp\else\^{}\fi}\catcode`\%=\active\def%{\%}$\mathdefault{27.5}$}}%
\end{pgfscope}%
\begin{pgfscope}%
\pgfpathrectangle{\pgfqpoint{0.750000in}{0.440000in}}{\pgfqpoint{4.650000in}{3.080000in}}%
\pgfusepath{clip}%
\pgfsetroundcap%
\pgfsetroundjoin%
\pgfsetlinewidth{1.003750pt}%
\definecolor{currentstroke}{rgb}{0.800000,0.800000,0.800000}%
\pgfsetstrokecolor{currentstroke}%
\pgfsetdash{}{0pt}%
\pgfpathmoveto{\pgfqpoint{0.750000in}{2.880185in}}%
\pgfpathlineto{\pgfqpoint{5.400000in}{2.880185in}}%
\pgfusepath{stroke}%
\end{pgfscope}%
\begin{pgfscope}%
\definecolor{textcolor}{rgb}{0.150000,0.150000,0.150000}%
\pgfsetstrokecolor{textcolor}%
\pgfsetfillcolor{textcolor}%
\pgftext[x=0.269912in, y=2.813918in, left, base]{\color{textcolor}{\rmfamily\fontsize{13.750000}{16.500000}\selectfont\catcode`\^=\active\def^{\ifmmode\sp\else\^{}\fi}\catcode`\%=\active\def%{\%}$\mathdefault{30.0}$}}%
\end{pgfscope}%
\begin{pgfscope}%
\pgfpathrectangle{\pgfqpoint{0.750000in}{0.440000in}}{\pgfqpoint{4.650000in}{3.080000in}}%
\pgfusepath{clip}%
\pgfsetroundcap%
\pgfsetroundjoin%
\pgfsetlinewidth{1.003750pt}%
\definecolor{currentstroke}{rgb}{0.800000,0.800000,0.800000}%
\pgfsetstrokecolor{currentstroke}%
\pgfsetdash{}{0pt}%
\pgfpathmoveto{\pgfqpoint{0.750000in}{3.331442in}}%
\pgfpathlineto{\pgfqpoint{5.400000in}{3.331442in}}%
\pgfusepath{stroke}%
\end{pgfscope}%
\begin{pgfscope}%
\definecolor{textcolor}{rgb}{0.150000,0.150000,0.150000}%
\pgfsetstrokecolor{textcolor}%
\pgfsetfillcolor{textcolor}%
\pgftext[x=0.269912in, y=3.265175in, left, base]{\color{textcolor}{\rmfamily\fontsize{13.750000}{16.500000}\selectfont\catcode`\^=\active\def^{\ifmmode\sp\else\^{}\fi}\catcode`\%=\active\def%{\%}$\mathdefault{32.5}$}}%
\end{pgfscope}%
\begin{pgfscope}%
\definecolor{textcolor}{rgb}{0.150000,0.150000,0.150000}%
\pgfsetstrokecolor{textcolor}%
\pgfsetfillcolor{textcolor}%
\pgftext[x=0.214356in,y=1.980000in,,bottom,rotate=90.000000]{\color{textcolor}{\rmfamily\fontsize{15.000000}{18.000000}\selectfont\catcode`\^=\active\def^{\ifmmode\sp\else\^{}\fi}\catcode`\%=\active\def%{\%}PSNR $\uparrow$}}%
\end{pgfscope}%
\begin{pgfscope}%
\pgfpathrectangle{\pgfqpoint{0.750000in}{0.440000in}}{\pgfqpoint{4.650000in}{3.080000in}}%
\pgfusepath{clip}%
\pgfsetbuttcap%
\pgfsetroundjoin%
\definecolor{currentfill}{rgb}{0.003922,0.450980,0.698039}%
\pgfsetfillcolor{currentfill}%
\pgfsetlinewidth{1.003750pt}%
\definecolor{currentstroke}{rgb}{0.003922,0.450980,0.698039}%
\pgfsetstrokecolor{currentstroke}%
\pgfsetdash{}{0pt}%
\pgfsys@defobject{currentmarker}{\pgfqpoint{-0.098209in}{-0.098209in}}{\pgfqpoint{0.098209in}{0.098209in}}{%
\pgfpathmoveto{\pgfqpoint{-0.000000in}{-0.098209in}}%
\pgfpathlineto{\pgfqpoint{0.098209in}{0.000000in}}%
\pgfpathlineto{\pgfqpoint{0.000000in}{0.098209in}}%
\pgfpathlineto{\pgfqpoint{-0.098209in}{0.000000in}}%
\pgfpathlineto{\pgfqpoint{-0.000000in}{-0.098209in}}%
\pgfpathclose%
\pgfusepath{stroke,fill}%
}%
\begin{pgfscope}%
\pgfsys@transformshift{2.629665in}{1.660981in}%
\pgfsys@useobject{currentmarker}{}%
\end{pgfscope}%
\begin{pgfscope}%
\pgfsys@transformshift{3.348477in}{2.295099in}%
\pgfsys@useobject{currentmarker}{}%
\end{pgfscope}%
\begin{pgfscope}%
\pgfsys@transformshift{3.892187in}{2.682989in}%
\pgfsys@useobject{currentmarker}{}%
\end{pgfscope}%
\begin{pgfscope}%
\pgfsys@transformshift{4.210237in}{2.886669in}%
\pgfsys@useobject{currentmarker}{}%
\end{pgfscope}%
\begin{pgfscope}%
\pgfsys@transformshift{4.435897in}{3.193824in}%
\pgfsys@useobject{currentmarker}{}%
\end{pgfscope}%
\end{pgfscope}%
\begin{pgfscope}%
\pgfpathrectangle{\pgfqpoint{0.750000in}{0.440000in}}{\pgfqpoint{4.650000in}{3.080000in}}%
\pgfusepath{clip}%
\pgfsetbuttcap%
\pgfsetroundjoin%
\definecolor{currentfill}{rgb}{0.870588,0.560784,0.019608}%
\pgfsetfillcolor{currentfill}%
\pgfsetlinewidth{1.003750pt}%
\definecolor{currentstroke}{rgb}{0.870588,0.560784,0.019608}%
\pgfsetstrokecolor{currentstroke}%
\pgfsetdash{}{0pt}%
\pgfsys@defobject{currentmarker}{\pgfqpoint{-0.066046in}{-0.056182in}}{\pgfqpoint{0.066046in}{0.069444in}}{%
\pgfpathmoveto{\pgfqpoint{0.000000in}{0.069444in}}%
\pgfpathlineto{\pgfqpoint{-0.015591in}{0.021460in}}%
\pgfpathlineto{\pgfqpoint{-0.066046in}{0.021460in}}%
\pgfpathlineto{\pgfqpoint{-0.025227in}{-0.008197in}}%
\pgfpathlineto{\pgfqpoint{-0.040818in}{-0.056182in}}%
\pgfpathlineto{\pgfqpoint{-0.000000in}{-0.026525in}}%
\pgfpathlineto{\pgfqpoint{0.040818in}{-0.056182in}}%
\pgfpathlineto{\pgfqpoint{0.025227in}{-0.008197in}}%
\pgfpathlineto{\pgfqpoint{0.066046in}{0.021460in}}%
\pgfpathlineto{\pgfqpoint{0.015591in}{0.021460in}}%
\pgfpathlineto{\pgfqpoint{0.000000in}{0.069444in}}%
\pgfpathclose%
\pgfusepath{stroke,fill}%
}%
\begin{pgfscope}%
\pgfsys@transformshift{0.961364in}{0.580000in}%
\pgfsys@useobject{currentmarker}{}%
\end{pgfscope}%
\begin{pgfscope}%
\pgfsys@transformshift{1.972376in}{1.206999in}%
\pgfsys@useobject{currentmarker}{}%
\end{pgfscope}%
\begin{pgfscope}%
\pgfsys@transformshift{3.250795in}{2.004208in}%
\pgfsys@useobject{currentmarker}{}%
\end{pgfscope}%
\begin{pgfscope}%
\pgfsys@transformshift{3.795939in}{2.451742in}%
\pgfsys@useobject{currentmarker}{}%
\end{pgfscope}%
\begin{pgfscope}%
\pgfsys@transformshift{4.140025in}{2.525040in}%
\pgfsys@useobject{currentmarker}{}%
\end{pgfscope}%
\end{pgfscope}%
\begin{pgfscope}%
\pgfpathrectangle{\pgfqpoint{0.750000in}{0.440000in}}{\pgfqpoint{4.650000in}{3.080000in}}%
\pgfusepath{clip}%
\pgfsetbuttcap%
\pgfsetroundjoin%
\definecolor{currentfill}{rgb}{0.007843,0.619608,0.450980}%
\pgfsetfillcolor{currentfill}%
\pgfsetlinewidth{1.003750pt}%
\definecolor{currentstroke}{rgb}{0.007843,0.619608,0.450980}%
\pgfsetstrokecolor{currentstroke}%
\pgfsetdash{}{0pt}%
\pgfsys@defobject{currentmarker}{\pgfqpoint{-0.069444in}{-0.069444in}}{\pgfqpoint{0.069444in}{0.069444in}}{%
\pgfpathmoveto{\pgfqpoint{-0.034722in}{-0.069444in}}%
\pgfpathlineto{\pgfqpoint{0.000000in}{-0.034722in}}%
\pgfpathlineto{\pgfqpoint{0.034722in}{-0.069444in}}%
\pgfpathlineto{\pgfqpoint{0.069444in}{-0.034722in}}%
\pgfpathlineto{\pgfqpoint{0.034722in}{0.000000in}}%
\pgfpathlineto{\pgfqpoint{0.069444in}{0.034722in}}%
\pgfpathlineto{\pgfqpoint{0.034722in}{0.069444in}}%
\pgfpathlineto{\pgfqpoint{0.000000in}{0.034722in}}%
\pgfpathlineto{\pgfqpoint{-0.034722in}{0.069444in}}%
\pgfpathlineto{\pgfqpoint{-0.069444in}{0.034722in}}%
\pgfpathlineto{\pgfqpoint{-0.034722in}{0.000000in}}%
\pgfpathlineto{\pgfqpoint{-0.069444in}{-0.034722in}}%
\pgfpathlineto{\pgfqpoint{-0.034722in}{-0.069444in}}%
\pgfpathclose%
\pgfusepath{stroke,fill}%
}%
\begin{pgfscope}%
\pgfsys@transformshift{4.901464in}{2.274505in}%
\pgfsys@useobject{currentmarker}{}%
\end{pgfscope}%
\begin{pgfscope}%
\pgfsys@transformshift{5.188636in}{3.294690in}%
\pgfsys@useobject{currentmarker}{}%
\end{pgfscope}%
\end{pgfscope}%
\begin{pgfscope}%
\pgfpathrectangle{\pgfqpoint{0.750000in}{0.440000in}}{\pgfqpoint{4.650000in}{3.080000in}}%
\pgfusepath{clip}%
\pgfsetbuttcap%
\pgfsetroundjoin%
\definecolor{currentfill}{rgb}{0.835294,0.368627,0.000000}%
\pgfsetfillcolor{currentfill}%
\pgfsetlinewidth{1.003750pt}%
\definecolor{currentstroke}{rgb}{0.835294,0.368627,0.000000}%
\pgfsetstrokecolor{currentstroke}%
\pgfsetdash{}{0pt}%
\pgfsys@defobject{currentmarker}{\pgfqpoint{-0.069444in}{-0.069444in}}{\pgfqpoint{0.069444in}{0.069444in}}{%
\pgfpathmoveto{\pgfqpoint{0.000000in}{0.069444in}}%
\pgfpathlineto{\pgfqpoint{-0.069444in}{-0.069444in}}%
\pgfpathlineto{\pgfqpoint{0.069444in}{-0.069444in}}%
\pgfpathlineto{\pgfqpoint{0.000000in}{0.069444in}}%
\pgfpathclose%
\pgfusepath{stroke,fill}%
}%
\begin{pgfscope}%
\pgfsys@transformshift{4.716743in}{3.380000in}%
\pgfsys@useobject{currentmarker}{}%
\end{pgfscope}%
\end{pgfscope}%
\begin{pgfscope}%
\pgfpathrectangle{\pgfqpoint{0.750000in}{0.440000in}}{\pgfqpoint{4.650000in}{3.080000in}}%
\pgfusepath{clip}%
\pgfsetbuttcap%
\pgfsetroundjoin%
\definecolor{currentfill}{rgb}{0.800000,0.470588,0.737255}%
\pgfsetfillcolor{currentfill}%
\pgfsetlinewidth{1.003750pt}%
\definecolor{currentstroke}{rgb}{0.800000,0.470588,0.737255}%
\pgfsetstrokecolor{currentstroke}%
\pgfsetdash{}{0pt}%
\pgfsys@defobject{currentmarker}{\pgfqpoint{-0.069444in}{-0.069444in}}{\pgfqpoint{0.069444in}{0.069444in}}{%
\pgfpathmoveto{\pgfqpoint{-0.000000in}{-0.069444in}}%
\pgfpathlineto{\pgfqpoint{0.069444in}{0.069444in}}%
\pgfpathlineto{\pgfqpoint{-0.069444in}{0.069444in}}%
\pgfpathlineto{\pgfqpoint{-0.000000in}{-0.069444in}}%
\pgfpathclose%
\pgfusepath{stroke,fill}%
}%
\begin{pgfscope}%
\pgfsys@transformshift{4.505328in}{3.284388in}%
\pgfsys@useobject{currentmarker}{}%
\end{pgfscope}%
\end{pgfscope}%
\begin{pgfscope}%
\pgfpathrectangle{\pgfqpoint{0.750000in}{0.440000in}}{\pgfqpoint{4.650000in}{3.080000in}}%
\pgfusepath{clip}%
\pgfsetroundcap%
\pgfsetroundjoin%
\pgfsetlinewidth{1.505625pt}%
\definecolor{currentstroke}{rgb}{0.003922,0.450980,0.698039}%
\pgfsetstrokecolor{currentstroke}%
\pgfsetdash{}{0pt}%
\pgfpathmoveto{\pgfqpoint{2.629665in}{1.660981in}}%
\pgfpathlineto{\pgfqpoint{3.348477in}{2.295099in}}%
\pgfpathlineto{\pgfqpoint{3.892187in}{2.682989in}}%
\pgfpathlineto{\pgfqpoint{4.210237in}{2.886669in}}%
\pgfpathlineto{\pgfqpoint{4.435897in}{3.193824in}}%
\pgfusepath{stroke}%
\end{pgfscope}%
\begin{pgfscope}%
\pgfsetrectcap%
\pgfsetmiterjoin%
\pgfsetlinewidth{1.254687pt}%
\definecolor{currentstroke}{rgb}{0.800000,0.800000,0.800000}%
\pgfsetstrokecolor{currentstroke}%
\pgfsetdash{}{0pt}%
\pgfpathmoveto{\pgfqpoint{0.750000in}{0.440000in}}%
\pgfpathlineto{\pgfqpoint{0.750000in}{3.520000in}}%
\pgfusepath{stroke}%
\end{pgfscope}%
\begin{pgfscope}%
\pgfsetrectcap%
\pgfsetmiterjoin%
\pgfsetlinewidth{1.254687pt}%
\definecolor{currentstroke}{rgb}{0.800000,0.800000,0.800000}%
\pgfsetstrokecolor{currentstroke}%
\pgfsetdash{}{0pt}%
\pgfpathmoveto{\pgfqpoint{5.400000in}{0.440000in}}%
\pgfpathlineto{\pgfqpoint{5.400000in}{3.520000in}}%
\pgfusepath{stroke}%
\end{pgfscope}%
\begin{pgfscope}%
\pgfsetrectcap%
\pgfsetmiterjoin%
\pgfsetlinewidth{1.254687pt}%
\definecolor{currentstroke}{rgb}{0.800000,0.800000,0.800000}%
\pgfsetstrokecolor{currentstroke}%
\pgfsetdash{}{0pt}%
\pgfpathmoveto{\pgfqpoint{0.750000in}{0.440000in}}%
\pgfpathlineto{\pgfqpoint{5.400000in}{0.440000in}}%
\pgfusepath{stroke}%
\end{pgfscope}%
\begin{pgfscope}%
\pgfsetrectcap%
\pgfsetmiterjoin%
\pgfsetlinewidth{1.254687pt}%
\definecolor{currentstroke}{rgb}{0.800000,0.800000,0.800000}%
\pgfsetstrokecolor{currentstroke}%
\pgfsetdash{}{0pt}%
\pgfpathmoveto{\pgfqpoint{0.750000in}{3.520000in}}%
\pgfpathlineto{\pgfqpoint{5.400000in}{3.520000in}}%
\pgfusepath{stroke}%
\end{pgfscope}%
\begin{pgfscope}%
\pgfsetbuttcap%
\pgfsetmiterjoin%
\definecolor{currentfill}{rgb}{1.000000,1.000000,1.000000}%
\pgfsetfillcolor{currentfill}%
\pgfsetfillopacity{0.800000}%
\pgfsetlinewidth{1.003750pt}%
\definecolor{currentstroke}{rgb}{0.800000,0.800000,0.800000}%
\pgfsetstrokecolor{currentstroke}%
\pgfsetstrokeopacity{0.800000}%
\pgfsetdash{}{0pt}%
\pgfpathmoveto{\pgfqpoint{0.883681in}{1.955175in}}%
\pgfpathlineto{\pgfqpoint{2.714149in}{1.955175in}}%
\pgfpathquadraticcurveto{\pgfqpoint{2.752344in}{1.955175in}}{\pgfqpoint{2.752344in}{1.993369in}}%
\pgfpathlineto{\pgfqpoint{2.752344in}{3.386319in}}%
\pgfpathquadraticcurveto{\pgfqpoint{2.752344in}{3.424514in}}{\pgfqpoint{2.714149in}{3.424514in}}%
\pgfpathlineto{\pgfqpoint{0.883681in}{3.424514in}}%
\pgfpathquadraticcurveto{\pgfqpoint{0.845486in}{3.424514in}}{\pgfqpoint{0.845486in}{3.386319in}}%
\pgfpathlineto{\pgfqpoint{0.845486in}{1.993369in}}%
\pgfpathquadraticcurveto{\pgfqpoint{0.845486in}{1.955175in}}{\pgfqpoint{0.883681in}{1.955175in}}%
\pgfpathlineto{\pgfqpoint{0.883681in}{1.955175in}}%
\pgfpathclose%
\pgfusepath{stroke,fill}%
\end{pgfscope}%
\begin{pgfscope}%
\pgfsetbuttcap%
\pgfsetroundjoin%
\definecolor{currentfill}{rgb}{0.003922,0.450980,0.698039}%
\pgfsetfillcolor{currentfill}%
\pgfsetlinewidth{1.003750pt}%
\definecolor{currentstroke}{rgb}{0.003922,0.450980,0.698039}%
\pgfsetstrokecolor{currentstroke}%
\pgfsetdash{}{0pt}%
\pgfsys@defobject{currentmarker}{\pgfqpoint{-0.098209in}{-0.098209in}}{\pgfqpoint{0.098209in}{0.098209in}}{%
\pgfpathmoveto{\pgfqpoint{-0.000000in}{-0.098209in}}%
\pgfpathlineto{\pgfqpoint{0.098209in}{0.000000in}}%
\pgfpathlineto{\pgfqpoint{0.000000in}{0.098209in}}%
\pgfpathlineto{\pgfqpoint{-0.098209in}{0.000000in}}%
\pgfpathlineto{\pgfqpoint{-0.000000in}{-0.098209in}}%
\pgfpathclose%
\pgfusepath{stroke,fill}%
}%
\begin{pgfscope}%
\pgfsys@transformshift{1.112847in}{3.264575in}%
\pgfsys@useobject{currentmarker}{}%
\end{pgfscope}%
\end{pgfscope}%
\begin{pgfscope}%
\definecolor{textcolor}{rgb}{0.150000,0.150000,0.150000}%
\pgfsetstrokecolor{textcolor}%
\pgfsetfillcolor{textcolor}%
\pgftext[x=1.456597in,y=3.214444in,left,base]{\color{textcolor}{\rmfamily\fontsize{13.750000}{16.500000}\selectfont\catcode`\^=\active\def^{\ifmmode\sp\else\^{}\fi}\catcode`\%=\active\def%{\%}\textbf{Ours}}}%
\end{pgfscope}%
\begin{pgfscope}%
\pgfsetbuttcap%
\pgfsetroundjoin%
\definecolor{currentfill}{rgb}{0.870588,0.560784,0.019608}%
\pgfsetfillcolor{currentfill}%
\pgfsetlinewidth{1.003750pt}%
\definecolor{currentstroke}{rgb}{0.870588,0.560784,0.019608}%
\pgfsetstrokecolor{currentstroke}%
\pgfsetdash{}{0pt}%
\pgfsys@defobject{currentmarker}{\pgfqpoint{-0.066046in}{-0.056182in}}{\pgfqpoint{0.066046in}{0.069444in}}{%
\pgfpathmoveto{\pgfqpoint{0.000000in}{0.069444in}}%
\pgfpathlineto{\pgfqpoint{-0.015591in}{0.021460in}}%
\pgfpathlineto{\pgfqpoint{-0.066046in}{0.021460in}}%
\pgfpathlineto{\pgfqpoint{-0.025227in}{-0.008197in}}%
\pgfpathlineto{\pgfqpoint{-0.040818in}{-0.056182in}}%
\pgfpathlineto{\pgfqpoint{-0.000000in}{-0.026525in}}%
\pgfpathlineto{\pgfqpoint{0.040818in}{-0.056182in}}%
\pgfpathlineto{\pgfqpoint{0.025227in}{-0.008197in}}%
\pgfpathlineto{\pgfqpoint{0.066046in}{0.021460in}}%
\pgfpathlineto{\pgfqpoint{0.015591in}{0.021460in}}%
\pgfpathlineto{\pgfqpoint{0.000000in}{0.069444in}}%
\pgfpathclose%
\pgfusepath{stroke,fill}%
}%
\begin{pgfscope}%
\pgfsys@transformshift{1.112847in}{2.988811in}%
\pgfsys@useobject{currentmarker}{}%
\end{pgfscope}%
\end{pgfscope}%
\begin{pgfscope}%
\definecolor{textcolor}{rgb}{0.150000,0.150000,0.150000}%
\pgfsetstrokecolor{textcolor}%
\pgfsetfillcolor{textcolor}%
\pgftext[x=1.456597in,y=2.938681in,left,base]{\color{textcolor}{\rmfamily\fontsize{13.750000}{16.500000}\selectfont\catcode`\^=\active\def^{\ifmmode\sp\else\^{}\fi}\catcode`\%=\active\def%{\%}FLoD~\cite{seo2024flod}}}%
\end{pgfscope}%
\begin{pgfscope}%
\pgfsetbuttcap%
\pgfsetroundjoin%
\definecolor{currentfill}{rgb}{0.007843,0.619608,0.450980}%
\pgfsetfillcolor{currentfill}%
\pgfsetlinewidth{1.003750pt}%
\definecolor{currentstroke}{rgb}{0.007843,0.619608,0.450980}%
\pgfsetstrokecolor{currentstroke}%
\pgfsetdash{}{0pt}%
\pgfsys@defobject{currentmarker}{\pgfqpoint{-0.069444in}{-0.069444in}}{\pgfqpoint{0.069444in}{0.069444in}}{%
\pgfpathmoveto{\pgfqpoint{-0.034722in}{-0.069444in}}%
\pgfpathlineto{\pgfqpoint{0.000000in}{-0.034722in}}%
\pgfpathlineto{\pgfqpoint{0.034722in}{-0.069444in}}%
\pgfpathlineto{\pgfqpoint{0.069444in}{-0.034722in}}%
\pgfpathlineto{\pgfqpoint{0.034722in}{0.000000in}}%
\pgfpathlineto{\pgfqpoint{0.069444in}{0.034722in}}%
\pgfpathlineto{\pgfqpoint{0.034722in}{0.069444in}}%
\pgfpathlineto{\pgfqpoint{0.000000in}{0.034722in}}%
\pgfpathlineto{\pgfqpoint{-0.034722in}{0.069444in}}%
\pgfpathlineto{\pgfqpoint{-0.069444in}{0.034722in}}%
\pgfpathlineto{\pgfqpoint{-0.034722in}{0.000000in}}%
\pgfpathlineto{\pgfqpoint{-0.069444in}{-0.034722in}}%
\pgfpathlineto{\pgfqpoint{-0.034722in}{-0.069444in}}%
\pgfpathclose%
\pgfusepath{stroke,fill}%
}%
\begin{pgfscope}%
\pgfsys@transformshift{1.112847in}{2.702353in}%
\pgfsys@useobject{currentmarker}{}%
\end{pgfscope}%
\end{pgfscope}%
\begin{pgfscope}%
\definecolor{textcolor}{rgb}{0.150000,0.150000,0.150000}%
\pgfsetstrokecolor{textcolor}%
\pgfsetfillcolor{textcolor}%
\pgftext[x=1.456597in,y=2.652223in,left,base]{\color{textcolor}{\rmfamily\fontsize{13.750000}{16.500000}\selectfont\catcode`\^=\active\def^{\ifmmode\sp\else\^{}\fi}\catcode`\%=\active\def%{\%}Octree-GS~\cite{ren2024octree}}}%
\end{pgfscope}%
\begin{pgfscope}%
\pgfsetbuttcap%
\pgfsetroundjoin%
\definecolor{currentfill}{rgb}{0.835294,0.368627,0.000000}%
\pgfsetfillcolor{currentfill}%
\pgfsetlinewidth{1.003750pt}%
\definecolor{currentstroke}{rgb}{0.835294,0.368627,0.000000}%
\pgfsetstrokecolor{currentstroke}%
\pgfsetdash{}{0pt}%
\pgfsys@defobject{currentmarker}{\pgfqpoint{-0.069444in}{-0.069444in}}{\pgfqpoint{0.069444in}{0.069444in}}{%
\pgfpathmoveto{\pgfqpoint{0.000000in}{0.069444in}}%
\pgfpathlineto{\pgfqpoint{-0.069444in}{-0.069444in}}%
\pgfpathlineto{\pgfqpoint{0.069444in}{-0.069444in}}%
\pgfpathlineto{\pgfqpoint{0.000000in}{0.069444in}}%
\pgfpathclose%
\pgfusepath{stroke,fill}%
}%
\begin{pgfscope}%
\pgfsys@transformshift{1.112847in}{2.415895in}%
\pgfsys@useobject{currentmarker}{}%
\end{pgfscope}%
\end{pgfscope}%
\begin{pgfscope}%
\definecolor{textcolor}{rgb}{0.150000,0.150000,0.150000}%
\pgfsetstrokecolor{textcolor}%
\pgfsetfillcolor{textcolor}%
\pgftext[x=1.456597in,y=2.365765in,left,base]{\color{textcolor}{\rmfamily\fontsize{13.750000}{16.500000}\selectfont\catcode`\^=\active\def^{\ifmmode\sp\else\^{}\fi}\catcode`\%=\active\def%{\%}Scaffold-GS~\cite{lu2024scaffold}}}%
\end{pgfscope}%
\begin{pgfscope}%
\pgfsetbuttcap%
\pgfsetroundjoin%
\definecolor{currentfill}{rgb}{0.800000,0.470588,0.737255}%
\pgfsetfillcolor{currentfill}%
\pgfsetlinewidth{1.003750pt}%
\definecolor{currentstroke}{rgb}{0.800000,0.470588,0.737255}%
\pgfsetstrokecolor{currentstroke}%
\pgfsetdash{}{0pt}%
\pgfsys@defobject{currentmarker}{\pgfqpoint{-0.069444in}{-0.069444in}}{\pgfqpoint{0.069444in}{0.069444in}}{%
\pgfpathmoveto{\pgfqpoint{-0.000000in}{-0.069444in}}%
\pgfpathlineto{\pgfqpoint{0.069444in}{0.069444in}}%
\pgfpathlineto{\pgfqpoint{-0.069444in}{0.069444in}}%
\pgfpathlineto{\pgfqpoint{-0.000000in}{-0.069444in}}%
\pgfpathclose%
\pgfusepath{stroke,fill}%
}%
\begin{pgfscope}%
\pgfsys@transformshift{1.112847in}{2.129437in}%
\pgfsys@useobject{currentmarker}{}%
\end{pgfscope}%
\end{pgfscope}%
\begin{pgfscope}%
\definecolor{textcolor}{rgb}{0.150000,0.150000,0.150000}%
\pgfsetstrokecolor{textcolor}%
\pgfsetfillcolor{textcolor}%
\pgftext[x=1.456597in,y=2.079307in,left,base]{\color{textcolor}{\rmfamily\fontsize{13.750000}{16.500000}\selectfont\catcode`\^=\active\def^{\ifmmode\sp\else\^{}\fi}\catcode`\%=\active\def%{\%}3DGS~\cite{kerbl20233d}}}%
\end{pgfscope}%
\end{pgfpicture}%
\makeatother%
\endgroup%

  }
  \caption{\textbf{Rendering quality vs number of Gaussians for the DTU dataset~\cite{aanaes2016large} --}
    We show the PSNR values for each method with respect to the number of
    Gaussians (level of detail).
    Our method is able to provide a continuous level of detail, and provides
    the best rendering quality for the same budget.
    Our finest level provides rendering quality that is on par with the
    traditional baselines.
  }
  \label{fig:clog-num-gaussians-psnr}
\end{figure}

\begin{table}[!t]
  \centering
  \resizebox{\linewidth}{!}{
    \begin{tabular}{lccccccccc}
      \toprule
      \multirow{2}[2]{*}{Method}        & \multicolumn{3}{c}{AVA~\cite{martinez2024codec}} & \multicolumn{3}{c}{Blender~\cite{mildenhall2020nerf}} & \multicolumn{3}{c}{DTU~\cite{aanaes2016large}}                                                                                                                             \\
      \cmidrule(lr){2-4}\cmidrule(lr){5-7}\cmidrule(lr){8-10}
                                        & PSNR $\uparrow$                                  & SSIM $\uparrow$                                       & LPIPS $\downarrow$                             & PSNR $\uparrow$    & SSIM $\uparrow$   & LPIPS $\downarrow$ & PSNR $\uparrow$    & SSIM $\uparrow$   & LPIPS $\downarrow$ \\
      \midrule
      3DGS~\cite{kerbl20233d}           & 22.840 $\pm$ 1.291                               & 0.872 $\pm$ 0.030                                     & 0.175 $\pm$ 0.019                              & 33.783 $\pm$ 3.692 & 0.970 $\pm$ 0.027 & 0.031 $\pm$ 0.031  & 31.758 $\pm$ 3.966 & 0.944 $\pm$ 0.030 & 0.063 $\pm$ 0.027  \\
      Octree-GS~\cite{ren2024octree}    & 22.845 $\pm$ 1.229                               & 0.871 $\pm$ 0.035                                     & 0.173 $\pm$ 0.020                              & 32.268 $\pm$ 3.622 & 0.961 $\pm$ 0.028 & 0.042 $\pm$ 0.030  & 31.803 $\pm$ 3.751 & 0.941 $\pm$ 0.030 & 0.069 $\pm$ 0.027  \\
      Scaffold-GS~\cite{lu2024scaffold} & 18.664 $\pm$ 1.192                               & 0.729 $\pm$ 0.047                                     & 0.364 $\pm$ 0.035                              & 33.697 $\pm$ 3.630 & 0.968 $\pm$ 0.028 & 0.033 $\pm$ 0.031  & 32.224 $\pm$ 3.788 & 0.945 $\pm$ 0.027 & 0.065 $\pm$ 0.025  \\
      \midrule
      FLoD~\cite{seo2024flod}           & 22.295 $\pm$ 1.465                               & 0.871 $\pm$ 0.030                                     & 0.191 $\pm$ 0.020                              & 29.480 $\pm$ 8.381 & 0.923 $\pm$ 0.087 & 0.075 $\pm$ 0.079  & 29.572 $\pm$ 7.597 & 0.897 $\pm$ 0.118 & 0.106 $\pm$ 0.089  \\
      \textbf{Ours}                     & 27.896 $\pm$ 2.236                               & 0.858 $\pm$ 0.047                                     & 0.209 $\pm$ 0.031                              & 31.797 $\pm$ 3.341 & 0.963 $\pm$ 0.031 & 0.046 $\pm$ 0.041  & 30.727 $\pm$ 3.948 & 0.936 $\pm$ 0.035 & 0.085 $\pm$ 0.036  \\
      \bottomrule

    \end{tabular}
  }
  \caption{
    \textbf{Quantitative results (finest level of detail) --}
    We report the rendering quality our method, at the finest level of detail,
    compared to other baselines.
    we provide comparable rendering quality, and additionally allow continuous levels of detail.
    Note that NeRF Synthetic~\cite{mildenhall2020nerf} dataset results for
    FLoD~\cite{seo2024flod} should be interpreted with caution as it performs
    excessively poorly in some sequences.
  }
  \label{tab:clog-quantitative}
\end{table}

\section{Results}
  \label{sec:clog-result}

  We first discuss our evaluation setup, then present our results.
  We also provide ablation study on the design choices.

  \subsection{Experimental setup}

    We discuss the datasets, the baselines, and metrics.

    \paragraph{Datasets.}
      We evaluate our method on three different datasets.
      We use the NeRF Synthetic~\cite{mildenhall2020nerf} as this dataset is
      commonly used in previous work.
      We also use the DTU~\cite{aanaes2016large} dataset to evaluate
      performance on a non-synthetic dataset.
      Following the standard protocol~\cite{ren2024octree}, we use the
      provided foreground masks.
      Finally, we use the AVA~\cite{martinez2024codec} dataset to showcase
      that our model can be also applied when ground truth UV mapping is
      available.
      % Since the datasets contain ground-truth background masks, we use them
      % to remove the background and train the models to reconstruct the
      % central object.

    \paragraph{Baselines.}
      We evaluate method against other neural-based Gaussian Splatting, namely
      Scaffold-GS~\cite{lu2024scaffold} and Octree-GS~\cite{ren2024octree},
      and vanilla 3DGS~\cite{kerbl20233d}.
      Notably, Octree-GS implements a levels of detail mechanism which enables
      adapting the output to the available computational constraints.
      We additionally provide a comparison against FLoD~\cite{seo2024flod}, a
      concurrent method that trains several defined levels.
      For each baseline, we use default configurations provided by the
      authors, using the official implementations.

      % We compare the methods on NeRF Synthetic~\cite{mildenhall2020nerf} and
      % DTU~\cite{aanaes2016large}. 

      % drums: 0

% scan37: 6 scan114: 6 SCAN97: 0

% 20220412--1326--BGR645: 8 20220809--1034--BJM420: 3
\newcommand{\lodimg}[5]{
  \includegraphics[height=\height]{assets/\clogdirname/qualitative/#1_#2_#3_#4_#5.png}
}

\newcommand{\finallod}{
  \def\height{84bp}
  \tikzsetnextfilename{clog_qualitative}
  \tikzset{external/export next=false}
  \resizebox{\linewidth}{!}{
    \begin{tikzpicture}[
        >=stealth',
        overlay/.style={
            anchor=south west,
            draw=black,
            rectangle,
            line width=0pt,
            outer sep=0,
            inner sep=0,
          },
      ]
      \matrix[
        matrix of nodes,
        column sep=-3pt,
        row sep=-0.1pt,
        ampersand replacement=\&,
        inner sep=0,
        outer sep=0
      ] (gts) {
        \lodimg{gt}{blender}{hotdog}{0000}{1.0} \\
        \lodimg{gt}{dtu}{scan105}{0000}{1.0}    \\
        % \img{gt}{ava}{20220412--1326--BGR645}{0008}{1.0} \\
        % \img{gt}{ava}{20220809--1034--BJM420}{0003}{1.0} \\
      };
      \node[above=0em of gts-1-1.north] {Ground Truth};

      \matrix[
        matrix of nodes,
        right=1em of gts,
        column sep=-3pt,
        row sep=-0.1pt,
        ampersand replacement=\&,
        inner sep=0,
        outer sep=0
      ] (preds) {
        \lodimg{ours}{blender}{hotdog}{0000}{1.0} \&
        \lodimg{flod}{blender}{hotdog}{0000}{5} \&
        \lodimg{octree}{blender}{hotdog}{0000}{1} \&
        \lodimg{scaffold}{blender}{hotdog}{0000}{1} \&
        \lodimg{3dgs}{blender}{hotdog}{0000}{1} \\
        \lodimg{ours}{dtu}{scan105}{0000}{1.0} \&
        \lodimg{flod}{dtu}{scan105}{0000}{5} \&
        \lodimg{octree}{dtu}{scan105}{0000}{1} \&
        \lodimg{scaffold}{dtu}{scan105}{0000}{1} \&
        \lodimg{3dgs}{dtu}{scan105}{0000}{1}    \\
      };
      \node[above=0em of gts-1-1.north] {Ground Truth};
      \node[above=0em of preds-1-1.north] {\textbf{Ours}};
      \node[above=0em of preds-1-2.north] {FLoD~\cite{seo2024flod}};
      \node[above=0em of preds-1-3.north] {OctreeGS~\cite{ren2024octree}};
      \node[above=0em of preds-1-4.north] {ScaffoldGS~\cite{lu2024scaffold}};
      \node[above=0em of preds-1-5.north] {3DGS~\cite{kerbl20233d}};

    \end{tikzpicture}
  }
}
\begin{figure}[t]
  \centering
  % \includegraphics[width=\linewidth, height=15em]{example-image-a}
  \finallod
  \caption{
    \textbf{Qualitative examples (finest level of detail) --}
    We show qualitative examples of our method, at the finest level of detail,
    compared to other baselines.
    Our method is able to provide comparable rendering quality at the finest
    level, and additionally allow continuous levels of detail as demonstrated
    in \cref{fig:clog-level-of-details}.
  }
  \label{fig:clog-qualitative}
\end{figure}
      
\newcommand{\secondversion}{
  \begin{tabular}{lcccccccc}
    \toprule
    \multirow{2}[2]{*}{Method}            & \multicolumn{4}{c}{Blender~\cite{mildenhall2020nerf}} & \multicolumn{4}{c}{DTU~\cite{aanaes2016large}}                                                                                                                                                                                            \\
    \cmidrule(lr){2-5}\cmidrule(lr){6-9}
                                          & PSNR $\uparrow$                                       & SSIM $\uparrow$                                & LPIPS $\downarrow$ & \#Gs $\downarrow$                                 & PSNR $\uparrow$    & SSIM $\uparrow$   & LPIPS $\downarrow$ & \#Gs $\downarrow$                                 \\
    \midrule
    Octree-GS~\cite{ren2024octree}, LoD=0 & 22.025 $\pm$ 2.371                                    & 0.884 $\pm$ 0.035                              & 0.115 $\pm$ 0.034  & $4.722 \times 10^{5}$ $\pm$ $2.998 \times 10^{5}$ & 26.609 $\pm$ 4.240 & 0.863 $\pm$ 0.082 & 0.146 $\pm$ 0.069  & $9.062 \times 10^{5}$ $\pm$ $4.178 \times 10^{5}$ \\
    Octree-GS~\cite{ren2024octree}, LoD=1 & 32.268 $\pm$ 3.622                                    & 0.961 $\pm$ 0.028                              & 0.042 $\pm$ 0.030  & $1.237 \times 10^{6}$ $\pm$ $4.392 \times 10^{5}$ & 31.803 $\pm$ 3.751 & 0.941 $\pm$ 0.030 & 0.069 $\pm$ 0.027  & $1.809 \times 10^{6}$ $\pm$ $8.495 \times 10^{5}$ \\
    \midrule
    FLoD~\cite{seo2024flod}, LoD=0        & 16.815 $\pm$ 2.169                                    & 0.736 $\pm$ 0.069                              & 0.236 $\pm$ 0.058  & $810$ $\pm$ $1318$                                & 17.915 $\pm$ 3.212 & 0.670 $\pm$ 0.105 & 0.295 $\pm$ 0.079  & $43$ $\pm$ $29$                                   \\
    FLoD~\cite{seo2024flod}, LoD=1        & 19.317 $\pm$ 2.631                                    & 0.799 $\pm$ 0.053                              & 0.227 $\pm$ 0.058  & $1249$ $\pm$ $1132$                               & 22.083 $\pm$ 4.777 & 0.763 $\pm$ 0.111 & 0.291 $\pm$ 0.091  & $632$ $\pm$ $428$                                 \\
    FLoD~\cite{seo2024flod}, LoD=2        & 23.338 $\pm$ 5.252                                    & 0.862 $\pm$ 0.066                              & 0.156 $\pm$ 0.059  & $1.117 \times 10^{4}$ $\pm$ $6841$                & 26.598 $\pm$ 6.897 & 0.833 $\pm$ 0.123 & 0.211 $\pm$ 0.097  & $1.389 \times 10^{4}$ $\pm$ $6795$                \\
    FLoD~\cite{seo2024flod}, LoD=3        & 27.880 $\pm$ 7.470                                    & 0.915 $\pm$ 0.084                              & 0.088 $\pm$ 0.074  & $5.927 \times 10^{4}$ $\pm$ $3.83 \times 10^{4}$  & 29.122 $\pm$ 7.623 & 0.889 $\pm$ 0.117 & 0.129 $\pm$ 0.087  & $6.157 \times 10^{4}$ $\pm$ $4.403 \times 10^{4}$ \\
    FLoD~\cite{seo2024flod}, LoD=4        & 29.480 $\pm$ 8.381                                    & 0.923 $\pm$ 0.087                              & 0.075 $\pm$ 0.079  & $1.879 \times 10^{5}$ $\pm$ $1.411 \times 10^{5}$ & 29.572 $\pm$ 7.597 & 0.897 $\pm$ 0.118 & 0.106 $\pm$ 0.089  & $1.654 \times 10^{5}$ $\pm$ $1.676 \times 10^{5}$ \\
    \midrule
    \textbf{Ours}, LoD=0                  & 21.109 $\pm$ 1.767                                    & 0.870 $\pm$ 0.037                              & 0.197 $\pm$ 0.047  & $2601$ $\pm$ $0$                                  & 23.648 $\pm$ 3.183 & 0.798 $\pm$ 0.087 & 0.283 $\pm$ 0.090  & $2621$ $\pm$ $0$                                  \\
    \textbf{Ours}, LoD=1                  & 24.535 $\pm$ 2.250                                    & 0.897 $\pm$ 0.035                              & 0.136 $\pm$ 0.039  & $1.638 \times 10^{4}$ $\pm$ $0$                   & 26.985 $\pm$ 3.526 & 0.851 $\pm$ 0.078 & 0.222 $\pm$ 0.077  & $1.638 \times 10^{4}$ $\pm$ $0$                   \\
    \textbf{Ours}, LoD=2                  & 27.401 $\pm$ 2.571                                    & 0.934 $\pm$ 0.023                              & 0.087 $\pm$ 0.027  & $6.554 \times 10^{4}$ $\pm$ $0$                   & 29.083 $\pm$ 3.680 & 0.895 $\pm$ 0.058 & 0.163 $\pm$ 0.057  & $6.554 \times 10^{4}$ $\pm$ $0$                   \\
    \textbf{Ours}, LoD=3                  & 29.238 $\pm$ 2.745                                    & 0.955 $\pm$ 0.017                              & 0.058 $\pm$ 0.019  & $1.475 \times 10^{5}$ $\pm$ $0$                   & 30.173 $\pm$ 3.864 & 0.921 $\pm$ 0.042 & 0.120 $\pm$ 0.039  & $1.475 \times 10^{5}$ $\pm$ $0$                   \\
    \textbf{Ours}, LoD=4                  & 32.076 $\pm$ 3.423                                    & 0.974 $\pm$ 0.014                              & 0.032 $\pm$ 0.015  & $2.621 \times 10^{5}$ $\pm$ $0$                   & 31.766 $\pm$ 4.353 & 0.946 $\pm$ 0.032 & 0.076 $\pm$ 0.025  & $2.621 \times 10^{5}$ $\pm$ $0$                   \\
    \bottomrule
  \end{tabular}
}
\newcommand{\firstoption}{
  \begin{tabular}{lcccccccc}
    \toprule
    \multirow{2}[2]{*}{Method}            & \multicolumn{4}{c}{Blender~\cite{mildenhall2020nerf}} & \multicolumn{4}{c}{DTU~\cite{aanaes2016large}}                                                                                                                                                                                                    \\
    \cmidrule(lr){2-5}\cmidrule(lr){6-9}
                                          & PSNR $\uparrow$                                       & SSIM $\uparrow$                                & LPIPS $\downarrow$ & Number of Gaussians $\downarrow$                      & PSNR $\uparrow$    & SSIM $\uparrow$   & LPIPS $\downarrow$ & Number of Gaussians $\downarrow$                      \\
    \midrule
    Octree-GS~\cite{ren2024octree}, LoD=0 & 22.025 $\pm$ 2.371                                    & 0.884 $\pm$ 0.035                              & 0.115 $\pm$ 0.034  & $4.722 {\times} 10^{5}$ $\pm$ $2.998 {\times} 10^{5}$ & 26.609 $\pm$ 4.240 & 0.863 $\pm$ 0.082 & 0.146 $\pm$ 0.069  & $9.062 {\times} 10^{5}$ $\pm$ $4.178 {\times} 10^{5}$ \\
    Octree-GS~\cite{ren2024octree}, LoD=1 & 32.268 $\pm$ 3.622                                    & 0.961 $\pm$ 0.028                              & 0.042 $\pm$ 0.030  & $1.237 {\times} 10^{6}$ $\pm$ $4.392 {\times} 10^{5}$ & 31.803 $\pm$ 3.751 & 0.941 $\pm$ 0.030 & 0.069 $\pm$ 0.027  & $1.809 {\times} 10^{6}$ $\pm$ $8.495 {\times} 10^{5}$ \\
    \midrule
    FLoD~\cite{seo2024flod}, LoD=0        & 16.815 $\pm$ 2.169                                    & 0.736 $\pm$ 0.069                              & 0.236 $\pm$ 0.058  & $0.810{\times} 10^3$ $\pm$ $1.318{\times} 10^3$       & 17.915 $\pm$ 3.212 & 0.670 $\pm$ 0.105 & 0.295 $\pm$ 0.079  & $0.043{\times} 10^3$ $\pm$ $0.029{\times} 10^3$       \\
    FLoD~\cite{seo2024flod}, LoD=1        & 19.317 $\pm$ 2.631                                    & 0.799 $\pm$ 0.053                              & 0.227 $\pm$ 0.058  & $1.249{\times} 10^3$ $\pm$ $1.132{\times} 10^3$       & 22.083 $\pm$ 4.777 & 0.763 $\pm$ 0.111 & 0.291 $\pm$ 0.091  & $0.632{\times} 10^3$ $\pm$ $0.428{\times} 10^3$       \\
    FLoD~\cite{seo2024flod}, LoD=2        & 23.338 $\pm$ 5.252                                    & 0.862 $\pm$ 0.066                              & 0.156 $\pm$ 0.059  & $1.117 {\times} 10^{4}$ $\pm$ $6.841{\times} 10^3$    & 26.598 $\pm$ 6.897 & 0.833 $\pm$ 0.123 & 0.211 $\pm$ 0.097  & $1.389 {\times} 10^{4}$ $\pm$ $6.795{\times} 10^3$    \\
    FLoD~\cite{seo2024flod}, LoD=3        & 27.880 $\pm$ 7.470                                    & 0.915 $\pm$ 0.084                              & 0.088 $\pm$ 0.074  & $5.927 {\times} 10^{4}$ $\pm$ $3.830 {\times} 10^{4}$ & 29.122 $\pm$ 7.623 & 0.889 $\pm$ 0.117 & 0.129 $\pm$ 0.087  & $6.157 {\times} 10^{4}$ $\pm$ $4.403 {\times} 10^{4}$ \\
    FLoD~\cite{seo2024flod}, LoD=4        & 29.480 $\pm$ 8.381                                    & 0.923 $\pm$ 0.087                              & 0.075 $\pm$ 0.079  & $1.879 {\times} 10^{5}$ $\pm$ $1.411 {\times} 10^{5}$ & 29.572 $\pm$ 7.597 & 0.897 $\pm$ 0.118 & 0.106 $\pm$ 0.089  & $1.654 {\times} 10^{5}$ $\pm$ $1.676 {\times} 10^{5}$ \\
    \midrule
    \textbf{Ours}, LoD=0                  & 21.109 $\pm$ 1.767                                    & 0.870 $\pm$ 0.037                              & 0.197 $\pm$ 0.047  & $2.601{\times} 10^3$ $\pm$ $0$                        & 23.648 $\pm$ 3.183 & 0.798 $\pm$ 0.087 & 0.283 $\pm$ 0.090  & $2.601{\times} 10^3$ $\pm$ $0$                        \\
    \textbf{Ours}, LoD=1                  & 24.535 $\pm$ 2.250                                    & 0.897 $\pm$ 0.035                              & 0.136 $\pm$ 0.039  & $1.638 {\times} 10^{4}$ $\pm$ $0$                     & 26.985 $\pm$ 3.526 & 0.851 $\pm$ 0.078 & 0.222 $\pm$ 0.077  & $1.638 {\times} 10^{4}$ $\pm$ $0$                     \\
    \textbf{Ours}, LoD=2                  & 27.401 $\pm$ 2.571                                    & 0.934 $\pm$ 0.023                              & 0.087 $\pm$ 0.027  & $6.554 {\times} 10^{4}$ $\pm$ $0$                     & 29.083 $\pm$ 3.680 & 0.895 $\pm$ 0.058 & 0.163 $\pm$ 0.057  & $6.554 {\times} 10^{4}$ $\pm$ $0$                     \\
    \textbf{Ours}, LoD=3                  & 29.238 $\pm$ 2.745                                    & 0.955 $\pm$ 0.017                              & 0.058 $\pm$ 0.019  & $1.475 {\times} 10^{5}$ $\pm$ $0$                     & 30.173 $\pm$ 3.864 & 0.921 $\pm$ 0.042 & 0.120 $\pm$ 0.039  & $1.475 {\times} 10^{5}$ $\pm$ $0$                     \\
    \textbf{Ours}, LoD=4                  & 32.076 $\pm$ 3.423                                    & 0.974 $\pm$ 0.014                              & 0.032 $\pm$ 0.015  & $2.621 {\times} 10^{5}$ $\pm$ $0$                     & 31.766 $\pm$ 4.353 & 0.946 $\pm$ 0.032 & 0.076 $\pm$ 0.025  & $2.621 {\times} 10^{5}$ $\pm$ $0$                     \\
    \bottomrule
  \end{tabular}

}

\begin{table}[!t]
  \centering
  \resizebox{\linewidth}{!}{
    \firstoption
  }
  \caption{\textbf{Quantitative results (varying levels of detail) } --
    We report the performance of each method that supports different levels of detail.
    Our method provides the best tradeoff in terms of number of Gaussians vs
    rendering quality.
    At the finest resolution, our method is comparable to the original
    3DGS~\cite{kerbl20233d} and also Octree-GS~\cite{ren2024octree}.
    Note that NeRF Synthetic~\cite{mildenhall2020nerf} dataset results for
    FLoD~\cite{seo2024flod} should be interpreted with caution as it performs
    excessively poorly in some sequences.
  }
  \label{tab:clog-lod}
\end{table}
      
\begin{table*}[!t]
  \centering
  \resizebox{\linewidth}{!}{
    \begin{tabular}{lccccccccc}
      \toprule
      \multirow{3}[3]{*}{Parameter}    & \multicolumn{6}{c}{Real Data}          & \multicolumn{3}{c}{Synthetic Data}                                                                                                                                                                                                                                                                            \\
      \cmidrule(lr){2-7}\cmidrule(lr){8-10}
                                       & \multicolumn{3}{c}{Casual Expressions} & \multicolumn{3}{c}{Novel Pose Synthesis} & \multicolumn{3}{c}{Novel Pose Synthesis}                                                                                                                                                                                                                           \\
      \cmidrule(lr){2-4}\cmidrule(lr){5-7}\cmidrule(lr){8-10}
                                       & PSNR $\uparrow$                        & SSIM $\uparrow$                          & LPIPS $\downarrow$                       & PSNR $\uparrow$                    & SSIM $\uparrow$                   & LPIPS $\downarrow$                & PSNR $\uparrow$                    & SSIM $\uparrow$                   & LPIPS $\downarrow$                \\
      \midrule
      $|\neighbourhood(\vertex)| = 1$  & 27.5620                                & 0.9043                                   & 0.0893                                   & 29.7269                            & 0.9306                            & 0.0815                            & 32.2371                            & \cellcolor{secondbestcolor}0.9882 & 0.0234                            \\
      $|\neighbourhood(\vertex)| = 5$  & 27.5880                                & \cellcolor{secondbestcolor}0.9054        & 0.0864                                   & \cellcolor{firstbestcolor}29.7548  & \cellcolor{secondbestcolor}0.9312 & 0.0789                            & 32.2900                            & \cellcolor{secondbestcolor}0.9882 & 0.0231                            \\
      $|\neighbourhood(\vertex)| = 10$ & \cellcolor{secondbestcolor}27.5933     & \cellcolor{secondbestcolor}0.9054        & \cellcolor{secondbestcolor}0.0859        & \cellcolor{secondbestcolor}29.7456 & \cellcolor{secondbestcolor}0.9312 & \cellcolor{secondbestcolor}0.0785 & \cellcolor{secondbestcolor}32.3324 & \cellcolor{secondbestcolor}0.9882 & \cellcolor{secondbestcolor}0.0230 \\
      $|\neighbourhood(\vertex)| = 20$ & \cellcolor{firstbestcolor}27.5977      & \cellcolor{firstbestcolor}0.9056         & \cellcolor{firstbestcolor}0.0854         & 29.7372                            & 0.9311                            & \cellcolor{firstbestcolor}0.0782  & \cellcolor{firstbestcolor}32.7949  & \cellcolor{firstbestcolor}0.9887  & \cellcolor{firstbestcolor}0.0221  \\
      \midrule
      Without smoothing                & \cellcolor{secondbestcolor}27.2535     & \cellcolor{secondbestcolor}0.8959        & \cellcolor{secondbestcolor}0.0939        & \cellcolor{secondbestcolor}29.3726 & \cellcolor{secondbestcolor}0.9233 & \cellcolor{secondbestcolor}0.0846 & \cellcolor{secondbestcolor}32.2452 & \cellcolor{secondbestcolor}0.9876 & \cellcolor{secondbestcolor}0.0238 \\
      With smoothing                   & \cellcolor{firstbestcolor}27.5977      & \cellcolor{firstbestcolor}0.9056         & \cellcolor{firstbestcolor}0.0854         & \cellcolor{firstbestcolor}29.7372  & \cellcolor{firstbestcolor}0.9311  & \cellcolor{firstbestcolor}0.0782  & \cellcolor{firstbestcolor}32.7949  & \cellcolor{firstbestcolor}0.9887  & \cellcolor{firstbestcolor}0.0221  \\
      \bottomrule
    \end{tabular}
  }
  \caption{\textbf{Ablation study} -- {
  First, we check the effect of the neighborhood size $|\neighbourhood(\vertex)|$ on the results.
  Below that, we compare the effect of smoothing.
  % We highlight ablations results as its the only benchmark where the
  % deformable model is fit reliably. 
  The best results are colored in \mycoloredbox{firstbestcolor} and the second
  best in \mycoloredbox{secondbestcolor}.
  For the real dataset, changing the neighborhood size gives inconsistent
  results, while smoothing improves the rendering quality.
  In the synthetic scenario, setting $|\neighbourhood(\vertex)|{=}20$ and the
  Laplacian smoothing consistently gives the best results.
  The discrepancy between real and synthetic datasets is caused by inaccurate
  face tracking for the former.
  We describe this issue in detail in~\cref{subsec:blendfields-failures}.
  }
  }
  \label{tab:blendfields-ablation-study}
\end{table*}

    \paragraph{Metrics.}
      We evaluate the novel-view rendering quality of each method via the
      standard metrics: PSNR, SSIM~\cite{wang2004image} and
      LPIPS~\cite{zhang2018perceptual}.
      % To measure the visual quality of the methods, we use 
      To provide an idea of the compute/memory budget for each method, we also
      provide the number of Gaussians used to render images for each method.
      For Scaffold-GS~\cite{lu2024scaffold} and
      Octree-GS~\cite{ren2024octree}, we measure the final number Gaussians
      used for rendering, which is by default ten times the number of point
      cloud points used in the representation.

      % as the number of points in the point cloud multiplied by the number of
      % Gaussians spawned at each point which defaults to 10 in the original
      % implementation. 

  \subsection{Continuous Level of Details}

    As the main focus of our work is to introduce continous levels of detail,
    we first demonstrate its performance qualitatively
    (\cref{fig:clog-level-of-details}) and quantitatively
    (\cref{fig:clog-num-gaussians-psnr} and \cref{tab:clog-lod}).
    As shown in \cref{fig:clog-level-of-details}, our method is able to render
    scenes at varying levels of detail.
    We additionally provide results for a concurrent work,
    FLoD~\cite{seo2024flod}, which is also able to provide levels of detail at
    pre-defined set of levels.
    Our method can render at \emph{any} number of Gaussians between $32^2$ and
    $512^2$, and provides sharper renderings.

    We note that results for FLoD~\cite{seo2024flod} for the NeRF
    Synthetic~\cite{mildenhall2020nerf} dataset should be interpreted with
    caution, as they performed particularly badly for \texttt{ficus} and
    \texttt{mic} sequences, despite our best efforts using the official
    implementation.
    We present metrics for each subject in the dataset and additional renders
    in~\supplementary{}.

    In \cref{fig:clog-num-gaussians-psnr}, we summarize the quantitative
    results in \cref{tab:clog-lod} for the DTU dataset.
    We use DTU only for this comparison as FLoD does not work well for the two
    sequences from the NeRF Synthetic dataset.
    As shown, our method provides the best rendering quality given a
    predetermined number of Gaussians.
    FLoD~\cite{seo2024flod} provides slightly inferior performance compared to
    ours, and while Octree-GS is able to be used to extract a LOD, its quality
    degrades quickly.
    Finally, at the finest level, our results are comparable to traditional 3D
    Gaussian Splatting.
    We emphasize once more, that our results are obtained with a \emph{single}
    underlying representation for all LoDs, simply rendered at a different
    level of detail.

    % We present that our method is capable of producing continuous levels of
    % details for a chosen value $l{\in}[0.1, 1]$. We do not use any values
    % than that as~\ours{} tends to produce illegible renderings.
    % Octree-GS~\cite{ren2024octree} uses two levels across all datasets.
    % FLoD~\cite{seo2024flod} applies 5 scales where the lowest scale tends to
    % produce less than 20 Gaussians on average. We show the results according
    % to predefined level. 

    % As the number of Gaussians is not comparable among levels of all the used
    % methods, we summarize the metrics according to their definition of level
    % in~\cref{tab:lod}. We additionally present in~\cref{fig:level-of-details}
    % how choosing a particular level affects the results. We show those values
    % on~\cref{fig:num-gaussians-psnr} where we can see that~\ours{} achieves
    % better PSNR performance than FLoD given the similar number of Gaussians.
    % Moreover, our method achieves high-fidelity results, comparable to other
    % baselines with a predefined number of LoDs, while enabling using a
    % continuous LoD.

  \subsection{At the finest level}
    To demonstrate that our method is also able to provide fine details if
    necessary, we provide a summary of the results at the finest level of
    detail.
    We provide qualitative examples in \cref{fig:clog-qualitative} and a
    summary in \cref{tab:clog-quantitative}.
    Our method provides comparable rendering quality to other state-of-the-art
    Gaussian Splatting methods, and on top, is able to provide continuous
    levels of detail.
    Additionally, we note that our method trains given a pre-defined budget,
    which makes deployment also easy.

    % We measure the capability of the proposed approach to reconstruct the
    % scene with high-fidelity. Please note that in this experiment our primary
    % goal is to match the quality of the baselines as measured by PSNR, SSIM,
    % LPIPS and number of Gaussians. \ours{} should not affect the fidelity but
    % merely imbue the Gaussian Splatting representation with continuous level
    % of details capabilities. We summarize the achieved results
    % in~\cref{tab:quantitative} and showcase the reconstructions
    % in~\cref{fig:qualitative}. Our approach fits between the baselines in
    % terms of the quality. In contrast to the baselines, its number of output
    % Gaussians is predictable---since we assume the maximum resolution of the
    % grid at the start, there are no changes in the final number of Gaussians.
    % We posit such a behavior to be beneficial in practice as the compute
    % requirements are predictable. The number can be also constrained if
    % required.

  \subsection{Ablation study}
    To motivate our design choices, we provide an ablation study where various
    components of our method are disabled.
    Specifically, we look into the impact of the modulator $\modulator$, the
    noise in $\modulated{\mean}'$~\cref{eq:clog-noise-loc}, the sorting with
    PLAS~\cite{morgenstern2024compact} in the feature space for Stage II
    training, the use of regularizers $\regularizer(\cdot)$, and disabling
    error-based relocation~\cite{bulo2024revising}.
    Others can trivially be disabled, and for the modulator $\modulator$ we
    use directly the downsampled descriptors~$\cloglatents$ without
    refinement, \ie, we apply $\down{\cloglatents}$ instead of
    $\modulated{\cloglatents}$ in the rendering
    equation~(\cref{eq:clog-rendering-image}).
    % To check how $\modulator$ affects the results, 
    We present our results in~\cref{tab:clog-ablation}.
