\begin{figure}
  \centering
  \begin{subfigure}[b]{0.39\linewidth}
    \centering
    \includegraphics[width=\linewidth]{fig/\conerfdirname/assets/implicit.pdf}
    \vspace{0.0em}
    \caption{Neural radiance field with lifting~\cite{park2021hypernerf}}
    \label{fig:conerf-implicit}
  \end{subfigure}
  \hfill
  \begin{subfigure}[b]{0.60\linewidth}
    \centering
    \includegraphics[width=\linewidth]{fig/\conerfdirname/assets/lac.pdf}
    \caption{Controllable neural radiance field (our method)}
    \label{fig:conerf-laced-implicit}
  \end{subfigure}
  \vspace{-0.7em}
  \caption{{\bf Framework -- }
    We depict in (a) the HyperNeRF~\cite{park2021hypernerf} formulation, and
    (b) our Controllable-NeRF~(CoNeRF).
    In (a), both point coordinates $\mathbf{x}$ and latent representation
    $\boldsymbol{\beta}$ are respectively processed by a canonicalizer
    $\Canonicalizer$ and a hyper map $\hypermap$, which are then turned into
    radiance and density field values by $\Representation$.
    In (b), we introduce regressors $\Attribute$ and $\MaskNet$ that regress
    the attribute and the corresponding mask that enable few-shot
    attribute-based control of the NeRF model.
    See~\cref{sec:conerf-inference} for details.
  }
  \label{fig:conerf-pipeline}
\end{figure}
\section{Controllable NeRF (CoNeRF)}
  Given a collection of $\nImages$ color images $\images \in [0,1]^{\width \times \height \times 3}$, we train our controllable neural radiance field model by an auto-decoding optimization~\cite{park2019deepsdf} whose losses can be grouped into two main subsets:
  \begin{align}
    \argmin_{\allpars=\pars, \latents}
    \underbrace{\loss{rep}(\allpars \given \images)}_\text{\cref{sec:conerf-autodecode}}{+}
    \underbrace{\loss{ctrl}(\allpars \given
      \{\Mask^\gt_{\iImage, \iAttribute} \}, \{\attribute^\gt_{\iImage,\iAttribute}\} )}_\text{\cref{sec:conerf-control}}
    .
  \end{align}
  The first group consists of the classical HyperNeRF~\cite{park2021hypernerf} auto-decoder losses, attempting to optimize neural network parameters $\pars$ jointly with latent codes $\latents$ to \textit{reproduce} the corresponding input images $\images$:
  \begin{align}
    \loss{rep}(\cdot) =
    \loss{recon}(\pars, \latents \given \images) +
    \loss{enc}(\latents)
    .
    \label{eq:conerf-loss_render}
  \end{align}
  The latter allow us to inject \textit{explicit control} into the representation, and are our core contribution:
  \begin{align}
    \loss{ctrl}(\cdot)
     & = \loss{mask}(\pars, \latents \given \{\Mask^\gt_{\iImage, \iAttribute} \})     & \text{g.t. masks}      \\
     & + \loss{attr}(\pars, \latents \given \{\attribute^\gt_{\iImage,\iAttribute}\}). & \text{g.t. attributes}
    \label{eq:loss_control}
  \end{align}
  As mentioned earlier in \cref{sec:conerf-intro}, we aim for a neural 3D
  appearance model that is controlled by a collection of attributes
  $\attributes {=} \{\attribute_\iAttribute\}$, and we expect each image to be
  a manifestation of a different value of attributes, that is, each image
  $\image_\iImage$, and hence each latent code $\latent_\iImage$, will have a
  corresponding attribute~$\attributes_\iImage$.
  The learnable connection between latent codes $\latent$ and the attributes
  $\attributes$, which we represent via regressors, is detailed
  in~\cref{sec:conerf-inference}.

  \subsection{Reconstruction losses}
    \label{sec:conerf-autodecode}
    The primary loss guiding the training of the NeRF model is the
    reconstruction loss, which simply aims to reconstruct observations
    $\images$.
    As in other neural radiance field models~\cite{mildenhall2020nerf, martin2021nerf, park2020deformable, park2021hypernerf} we simply minimize the L2 photometric reconstruction error with respect to ground truth images:
    \begin{equation}
      \loss{recon}(\cdot) = \sum_\iImage \expect_{\ray \sim \image_\iImage} \left[ \left\|\bC(\ray \given \latent_\iImage, \pars)-\bC^\text{gt}(\ray) \right\|_2^2 \right]
      .
      \label{eq:conerf-recon}
    \end{equation}
    As is typical in auto-decoders, and following~\cite{park2019deepsdf}, we impose a zero-mean Gaussian prior on the latent codes~$\latents$:
    \begin{equation}
      \loss{enc}(\cdot) = \sum_\iImage \left\| \latent_\iImage \right\|^2_2
      .
      \label{eq:conerf-latent_prior}
    \end{equation}

  \subsection{Control losses}
    \label{sec:conerf-control}
    The user defines a \textit{discrete} set of $\nAttributes$ number of
    attributes that they seek to control, that are \textit{sparsely}
    supervised across frames---we only supervise attributes \textit{when} we
    have an annotation, and let others be discovered on their own throughout
    the training process, as guided by~\cref{eq:conerf-loss_render}.
    More specifically, for a particular image $\image_\iImage$, and a particular attribute $\attribute_\iAttribute$, the user specifies the quantities:
    \begin{itemize}
      \item $\attribute_{\iImage, \iAttribute} \in [-1,1]$: specifying the value for the $\iAttribute$-th attribute in the $\iImage$-th image; see the \textit{sliders} in~\cref{fig:conerf-teaser};
      \item $\Mask_{\iImage, \iAttribute} \in [0,1]^{\width \times \height}$: roughly specifying the image region that is controlled by the $\iAttribute$-th attribute in the $\iImage$-th image; see the \textit{masks} in~\cref{fig:conerf-teaser}.
    \end{itemize}
    To formalize sparse supervision, we employ an indicator function
    $\indicator_{\iImage, \iAttribute}$, where $\indicator_{\iImage,
    \iAttribute}=1$ if an annotation for attribute $\iAttribute$ for image
    $\iImage$ is provided, otherwise $\indicator_{\iImage, \iAttribute}=0$.
    We then write the loss for \textit{attribute} supervision as:
    \begin{equation}
      \loss{attr}(\cdot) =
      \sum_\iImage \sum_\iAttribute
      \indicator_{\iImage, \iAttribute}
      |\attribute_{\iImage,\iAttribute} - \attribute^\gt_{\iImage,\iAttribute}|^2
      .
    \end{equation}

    For the mask few-shot supervision, \
    we employ the volume rendering in~\cref{eq:conerf-volren} to project the 3D volumetric neural mask field $\mask_\iAttribute(\bx)$ into image space, and then supervise it as:
    \newcommand{\crossentropy}[2]{\text{CE}\left(#1,#2\right)}
    \begin{equation}
      \loss{mask}(\cdot) \!= \!\!\sum_{\iImage, \iAttribute}
      \indicator_{\iImage, \iAttribute} \:
      \expect_\ray
      \left[
        \crossentropy
        {\Mask(\ray \given \latent_\iImage, \pars)}
        {\Mask^\gt_{\iImage, \iAttribute}(\ray)}
        \right]
      ,
      \label{eq:conerf-mask}
    \end{equation}
    where $\crossentropy{\cdot}{\cdot}$ denotes cross entropy, and the field $\sigma(\bx)$ in~\cref{eq:conerf-volren} is learned by minimizing~\cref{eq:conerf-recon}.
    Importantly, as we do not wish for~\cref{eq:conerf-mask} to interfere with
    the training of the underlying 3D representation learned through
    \cref{eq:conerf-recon}, we \textit{stop gradients}
    in~\cref{eq:conerf-mask} w.r.t. $\sigma(\bx)$.
    Furthermore, in practice, because the attribute mask vs.
    background distribution can be highly imbalanced depending on which attribute the user is trying to control (\eg an eye only covers a very small portion of an image), we employ a \textit{focal loss}~\cite{lin2017focal} in place of the standard cross entropy loss.

  \subsection{Controlling and rendering images}
    \label{sec:conerf-inference}
    In what follows, we drop the image subscript $\iImage$ to simplify
    notation without any loss of generality.
    Given a $B$-dimensional latent code $\latent$ representing the 3D scene behind an image, we derive a mapping to our $A$ attributes via a neural map $\Attribute$ with learnable parameters $\pars$:
    \begin{equation}
      \{ \attribute_\iAttribute \} = \AttributeNet(\latent \given \pars), \quad
      \AttributeNet : \real^\latentDim \rightarrow [0,1]^\nAttributes
      ,
    \end{equation}
    where these correspond to the \textit{sliders} in~\cref{fig:conerf-teaser}.
    In the same spirit of~\cref{eq:conerf-hypermap}, to allow for complex
    topological changes that may not be represented by the change in a single
    scalar value alone, we lift the attributes to a hyperspace.
    In addition, since each attribute governs different aspects of the scene, we employ \textit{per-attribute} learnable hypermaps $\{\hypermap_\iAttribute\}$, which we write:
    \begin{align}
      \attribute_\iAttribute(\point) & = \hypermap_\iAttribute(\point, \attribute_\iAttribute \given \pars) \quad \hypermap_\iAttribute: \real^3 \times \real \rightarrow \real^\hyperAttributeDim
      .
    \end{align}
    Note that while $\attribute_\iAttribute$ is a scalar \textit{value},
    $\attribute_\iAttribute(\point)$ is a \textit{field} that can be queried
    at any point $\point$ in space.
    These fields are concatenated to form
    $\attributes(\point)=\{\attribute_\iAttribute(\point)\}$.

    We then provide all this information to generate an \textit{attribute
    masking field} via a network $\MaskNet(\cdot \given \pars)$.
    This field determines which attribute \textit{attends} to which position in space~$\point$:
    \begin{align}
      \label{eq:conerf-mask_c}
      \mask_0(\point) & \oplus \mask(\point) = \MaskNet(\Canonicalizer(\point), \bbeta(\point), \attributes(\point) \given \pars),                   \\
      \MaskNet        & : \real^3 \times \real^\latentDim \times \real^{\nAttributes \times \hyperAttributeDim} \rightarrow \real^{\nAttributes+1}_+
      ,
    \end{align}
    where $\oplus$ is a concatenation operator, $\mask(\point){=}\{\mask_\iAttribute(\point)\}$, and the additional mask $\mask_0(\point)$ denotes space that is not affected by \textit{any} attribute.
    Note that because the mask location should be affected by both the
    particular attribute of interest (\eg, the selected eye status) and the
    global appearance of the scene (\eg, head movement), $\MaskNet$ takes both
    $\bbeta(\point)$ and $\attributes(\point)$ as input in addition to
    $\Canonicalizer(\point)$.
    In addition, because the mask is modeling the attention related to attributes, collectively, these masks satisfy the partition of unity property:
    \begin{align}
      \mask_0(\point) + \Sigma_\iAttribute [\mask_\iAttribute(\point)] = 1 \quad \forall \point \in \real^3
      .
    \end{align}
    Finally, in a similar spirit to~\cref{eq:conerf-hypernerf}, all of this information is processed by a neural network that produces the desired radiance and density fields used in volume rendering:
    \begin{equation}
      \begin{rcases}
        \bc(\point) \\
        \sigma(\point)
      \end{rcases} \!\!=\! \Representation(\Canonicalizer(\point),
      \underbrace{\mask(\point) \odot \attributes(\point)}_\text{attribute controls},
      \underbrace{\mask_0(\point) \cdot \latent(\point)}_\text{everything else}
      \given \pars)
      .
      \label{eq:conerf-representation}
    \end{equation}
    In particular, note that $\mask(\point){=}0$ implies
    $\mask_0(\point){=}1$, hence our solution has the capability of reverting
    to classical HyperNeRF~\cref{eq:conerf-hypernerf}, where all change in the
    scene is globally encoded in $\bbeta(\point)$.
    Finally, these fields can be used to render the mask in image space, following a process analogous to volume rendering of radiance:
    \begin{equation}
      \Mask(\ray\given\allpars) \!=\!\!
      \int_{t_n}^{t_f} \!\!\!\!
      T(t) \cdot \sigma(\br(t)) \cdot [\mask_0(\ray(t)) \oplus \mask(\ray(t))]
      \, dt .
      \label{eq:conerf-volren}
    \end{equation}
    We depict our inference flow in~\cref{fig:conerf-pipeline}~(b).

  \subsection{Implementation details}
    We implement our method for NeRF based on the JAX \cite{jax2018github}
    implementation of HyperNeRF~\cite{park2021hypernerf}.
    We use both the scheduled windowed positional encoding and weight
    initialization of \cite{park2020deformable}, as well as the coarse-to-fine
    training strategy~\cite{park2021hypernerf}.

    Besides the newly added networks, we follow the same architecture as
    HyperNeRF.
    For the attribute network $\Attribute$ we use a six-layer multi-layer
    perceptron (MLP) with 32 neurons at each layer, with a skip connection at
    the fifth layer, following \cite{park2020deformable,park2021hypernerf}.
    For the lifting network $\hypermap_\iAttribute$, we use the same
    architecture as $\hypermap$, except for the input and output dimension
    sizes.
    For the masking network $\MaskNet$ we use a four-layer MLP with 128
    neurons at each layer, followed by an additional 64 neuron layer with a
    skip connection.
    The network $\Representation$ also shares the same architecture as
    HyperNeRF, but with a different input dimension size to accommodate for
    the changes our method introduces.

    \paragraph{2D implementation}
      To show that our idea is not limited to neural radiance fields, we also
      test a 2D version of our framework that can be used to directly
      represent images, without going through volume rendering.
      We use the same architecture and training procedure as in the NeRF case,
      with the exception that we do not predict the density $\sigma$, and we
      also do not have the notion of depth---each ray is directly the pixel.
      We center crop each video and resize each frame to be $128\times128$.

    \paragraph{Hyperparameters}
      We train all our NeRF models with $480\times270$ images and with 128
      samples per ray.
      We train for 250k iterations with a batch size of 512 rays.
      During training, we maintain that 10\% of rays are sampled from
      annotated images.
      We set $\loss{attr}=10^{-1}$, $\loss{mask}=10^{-2}$ and
      $\loss{enc}=10^{-4}$.
      For the number of hyper dimensions we set $\hyperAttributeDim = 8$.
      For the 2D implementation experiments, we sample 64 random images from
      the scene and further subsample 1024 pixels from each of them.
      For all experiments we use Adam~\cite{kingma2014adam} with learning rate
      $10^{-4}$ exponentially decaying to $10^{-5}$ in 250k iterations.
      We provide additional details in the supplementary material.
      Training a single model takes around 12 hours on an NVIDIA V100 GPU.
