\section{Conclusions}
We have introduced CoNeRF, an intuitive controllable NeRF model that can be trained with few-shot annotations in the form of attribute masks.
The core contribution of our method is that we represent attributes as localized masks, which are then treated as latent variables within the framework.
To do so we regress the attribute and their corresponding masks with neural networks.
This leads to a few-shot learning setup, where the network learns to regress provided annotations, and if they are not provided for a given image, proper attributes and masked are discovered throughout training automatically.
We have shown that our method allows users to easily annotate what to control and how, within a single video simply by annotating a few frames, which then allows rendering of the scene from novel views and with novel attributes, at high quality.

\paragraph{Limitations}
While our method delivers controllability to NeRF models, there is room for improvement.
First, our disentanglement of attribute strictly relies on the locality assumption---if multiple attributes act on a single pixel, our method is likely to have entangled outcomes when rendering with different attributes. 
An interesting direction would therefore be to incorporate manifold disentanglement approaches~\cite{li2020markov,zhang2021product} to our method.
Second, while very few, we still require sparse annotations.
Unsupervised discovery of controllable attributes, for example as in \cite{kulkarni2019unsupervised}, in a scene remains yet to be explored.
Lastly, we resort to user intuition on which frames should be annotated---we heuristically choose frames with extreme attributes (\eg, mouth fully open).
While this is a valid strategy, an interesting direction for future research would be to employ active learning techniques for this purpose~\cite{ren2020survey,belharbi2021deep}

We further discuss potential societal impact of our work in the \SupplementaryMaterial.


\endinput