\section{Introduction}
  \label{sec:blendfields-intro}

  % --- neural rendering and importance in digital doubles

  Recent advances in neural rendering of 3D scenes~\cite{tewari2022advances}
  offer 3D reconstructions of unprecedented quality~\cite{mildenhall2020nerf}
  with an ever-increasing degree of control ~\cite{kania2022conerf,
  liu2021editing}.
  Human faces are of particular interest to the research
  community~\cite{athar2022rignerf, gafni2021dynamic, garbin2024voltemorph,
  gao2022reconstructing} due to their application in creating realistic
  digital doubles~\cite{ma2021pixel, tewari2022advances, zhang2022avatargen,
  zhi2022dualspace}.

  % --- parametric models cause lack of details
  To render facial expressions not observed during training, current
  solutions~\cite{athar2022rignerf, gafni2021dynamic, garbin2024voltemorph,
  gao2022reconstructing} rely on \textit{parametric} face
  models~\cite{blanz1999morphable}.
  These allow radiance fields~\cite{mildenhall2020nerf} to be controlled by
  facial parameters estimated by off-the-shelf face
  trackers~\cite{li2017flame}.
  However, parametric models primarily capture smooth deformations and lead to
  digital doubles that lack realism because fine-grained and
  expression-dependent phenomena like wrinkles are not faithfully reproduced.
  % \mkc{while this is true, even if wrinkles were modelled by the 3dmm, they
  % would not appear,as nerfs do not model illumination and wrinkles are only
  % visible thanis to shadows} --- how this problem is addressed for now (DATA)
  % \at{

  Authentic Volumetric Avatars (AVA)~\cite{cao2022authentic} overcomes this
  issue by learning from a large multi-view dataset of synchronized and
  calibrated images captured under controlled lighting.
  Their dataset covers a series of dynamic facial expressions and multiple
  subjects.
  However, this dataset remains unavailable to the public and is expensive to
  reproduce.
  Additionally, training models from such a large amount of data requires
  significant compute resources.
  To democratize digital face avatars, more efficient solutions in terms of
  hardware, data, and compute are necessary.

  % --- What we do (high level) \at{
  We address the efficiency concerns by building on the recent works in Neural
  Radiance Fields~\cite{garbin2024voltemorph,xu2022deforming,yuan2022nerf}.
  % VolTeMorph~\cite{garbin2022voltemorph}.
  In particular, we extend VolTeMorph~\cite{garbin2024voltemorph} to render
  facial details learned from images of a sparse set of expressions.
  To achieve this, we draw inspiration from blend-shape
  correctives~\cite{lewis2014practice}, which are often used in computer
  graphics as a data-driven way to correct potential mismatches between a
  simplified model and the complex phenomena it aims to represent.
  In our setting, this mismatch is caused by the low-frequency deformations
  that the tetrahedral mesh from VolTeMorph~\cite{garbin2024voltemorph},
  designed for real-time performance, can capture, and the high-frequency
  nature of expression \mbox{wrinkles}.

  % --- what we do (low level) \at{
  We train multiple radiance fields, one for each of the $\nExpr$ sparse
  expressions present in the input data, and blend them to correct the
  low-frequency estimate provided by VolTeMorph~\cite{garbin2024voltemorph};
  see \cref{fig:blendfields-teaser}.
  % Hence, 
  We call our method \blendfields since it resembles the way blend shapes are
  employed in 3DMMs~\cite{blanz1999morphable}.
  To keep $\nExpr$ small (\ie, to maintain a few-shot regime), we perform
  local blending to exploit the known correlation between wrinkles and changes
  in local differential properties~\cite{irving2004invertible, raman2022mesh}.
  Using the dynamic geometry of~\cite{garbin2024voltemorph}, local changes in
  differential properties can be easily extracted by analyzing the tetrahedral
  representation underlying the corrective blendfields of our model.
  % underlying used to train and drive the }

  % --- contributions \at{
  \paragraph{Contributions}
    We outline the main qualitative differences between our approach and
    related works in \cref{tab:blendfields-ava-ours-comparison}, and our
    empirical evaluations confirm these advantages.
    In summary, we:
    \begin{itemize}
      \item extend VolTeMorph~\cite{garbin2024voltemorph} to enable modeling of high-frequency information, such as expression wrinkles in a few-shot setting;
      \item introduce correctives~\cite{blanz1999morphable} to neural field representations and activate them according to local deformations~\cite{raman2022mesh};
      \item
            make this topic more accessible
            with an alternative to techniques that are data and compute-intensive~\cite{cao2022authentic};
      \item show that our model generalizes beyond facial modeling, for example, in the modeling of wrinkles on a deformable object made of rubber.
    \end{itemize}
    \begin{table*}[!t]
  \centering
  \resizebox{\linewidth}{!}{
    \begin{tabular}{lccccccc|c}
      \toprule
                                 & NeRF~\cite{mildenhall2020nerf} & NeRFies~\cite{park2021nerfies} & HyperNeRF~\cite{park2021hypernerf} & NeRFace~\cite{gafni2021dynamic} & NHA~\cite{grassal2022neural} & AVA \cite{cao2022authentic} & VolTeMorph~\cite{garbin2024voltemorph} & \textbf{Ours} \\
      \midrule
      Applicability beyond faces & \cmark                         & \cmark                         & \cmark                             & \xmark                          & \xmark                       & \xmark                      & \cmark                                 & \cmark        \\
      Interpretable control      & \xmark                         & \xmark                         & \xmark                             & \cmark                          & \cmark                       &
      \xmark                     & \cmark                         & \cmark                                                                                                                                                                                                                      \\ Data efficiency & \xmark & \cmark & \cmark &
      \xmark                     & \cmark                         & \xmark                         & \cmark                             & \cmark                                                                                                                                                \\ Expression-dependent
      changes                    & \xmark                         & \xmark                         & \cmark                             & \cmark                          & \cmark                       & \cmark                      & \xmark                                 &
      \cmark                                                                                                                                                                                                                                                                                    \\ Generalizability to unknown expressions & \xmark & \xmark &
      \xmark                     & \cmark                         & \cmark                         & \xmark                             & \cmark                          & \cmark                                                                                                              \\
      \bottomrule\end{tabular} }
  \caption{\textbf{Comparison} -- We compare
    several methods to our approach.
    Other methods fall short in data efficiency and applicability.
    For example, AVA~\cite{cao2022authentic} requires 3.1 million training
    images while VolTeMorph \cite{garbin2024voltemorph} cannot model
    expression-dependent wrinkles realistically.
  }
  \label{tab:blendfields-ava-ours-comparison}
\end{table*}