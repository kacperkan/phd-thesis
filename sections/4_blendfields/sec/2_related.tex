\section{Related Works}
  \label{sec:blendfields-related}
  Neural Radiance Fields (NeRF)~\cite{mildenhall2020nerf} is a method for
  generating 3D content from images taken with commodity cameras.
  It has prompted many follow-up works
  \cite{zhang2020nerf++,park2021nerfies,martin2021nerfw,barron2021mip,barron2022mip,verbin2022ref,tancik2022block,huang2022hdr,suhail2022light,xiangli2021citynerf,mildenhall2022nerf,rematas2022urban}
  and a major change in the field for its photorealism.
  The main limitations of NeRF are its rendering speed, being constrained to
  static scenes, and lack of ways to control the scene.
  Rendering speed has been successfully addressed by multiple follow-up
  works~\cite{hedman2021baking,garbin2021fastnerf,yu2021plenoctrees}.
  Works solving the limitation to static
  scenes\cite{xian2021space,athar2022rignerf,noguchi2022watch,wang2022fourier,attal2021torf,weng2022humannerf,zhao2022humannerf,jiang2022neuman}
  and adding explicit
  control~\cite{kania2022conerf,wang2022clip,kim2022ae,cheng2022cross,yuan2022nerf,sun2022fenerf,yang2022neumesh}
  have limited resolution or require large amounts of training data because
  they rely on controllable coarse models of the scene (\eg, 3DMM face
  model~\cite{blanz1999morphable}) or a conditioning
  signal~\cite{park2021hypernerf}.
  Methods built on an explicit model are more accessible because they require
  less training data but are limited by the model's resolution.
  Our technique finds a sweet spot between these two regimes by using a
  limited amount of data to learn details missing in the controlled model and
  combining them together.
  Our experiments focus on faces because high-quality data and 3DMM face
  models are publicly available, which are the key component for creating
  digital humans.

  \subsection{Radiance Fields}
    Volumetric representations~\cite{vicini2021nonexponential} have grown in
    popularity because they can represent complex geometries like hair more
    accurately than mesh-based ones.
    Neural Radiance Fields (NeRFs)~\cite{mildenhall2020nerf} model a radiance
    volume with a coordinate-based MLP learned from posed images.
    The MLP predicts density~$\density(\pos)$ and color~$\outputcolor(\pos,
    \viewdirection)$ for each point $\pos$ in the volume and view
    direction~$\viewdirection$ of a given camera.
    % \mkc{the colour as defined here is not a function of view direction, is
    % this deliberate? In our implementation its a function of both view
    % direction and position.}. 
    To supervise the radiance volume with the input images, each image pixel is associated with a ray $\ray(t)$ cast from the camera center to the pixel, and samples along the ray are accumulated to determine the value of the image pixel $\pixelcolor(\ray)$:
    % follows: 
    \begin{equation}
      \label{eq:blendfields-rendering-equation}
      \pixelcolor(\ray)=\int^{t_f}_{t_n}T(t)\:\density(\ray(t))\:\outputcolor(\ray(t), \viewdirection)dt,
    \end{equation}
    where $t_n$ and $t_f$ are near and far planes, and
    \begin{equation}
      T(t) = \exp\left(-\int^t_{t_n}\density(\ray(s))ds\right) ,
      \end{equation} is the transmittance
      function~\cite{tagliasacchi2022volume}.
    The weights of the MLP are optimized to minimize the mean squared
    reconstruction error between the target pixel and the output pixel.
    % \kk{% Note that we deliberately avoid mentioning view directions
    % \cite{mildenhall2020nerf} as some of the recent
    % approaches~\cite{nerfstudio} treat both 3D coordinates and view
    % directions as the same entity, without impairing the performance. } 
    Several methods have shown that replacing the implicit functions
    approximated with an MLP for a function discretized on an explicit voxel
    grid results in a significant rendering and training
    speed-up~\cite{garbin2021fastnerf, hedman2021baking, yu2021plenoctrees,
    sun2021direct, liu2020neural}.

  \subsection{Animating Radiance Fields}
    Several works exist to animate the scene represented as a NeRF.
    D-NeRF uses an implicit deformation model that maps sample positions back
    to a canonical space~\cite{pumarola2021d}, but it cannot generalize to
    unseen deformations.
    Several works \cite{park2021nerfies, park2021hypernerf, gafni2021dynamic,
    tretschk2021non} additionally account for changes in the observed scenes
    with a per-image latent code to model changes in color as well as shape,
    but it is unclear how to generalize the latents when animating a sequence
    without input images.
    Similarly, works focusing on faces
    \cite{gafni2021dynamic,athar2022rignerf,zhuang2021mofanerf,gao2022reconstructing}
    use parameters of a face model to condition NeRF's MLP, or learn a latent
    space of images and
    geometry~\cite{cao2022authentic,wang2022morf,lombardi2019neural,mihajlovic2022keypointnerf,ma2021pixel,lombardi2021mixture}
    that does not extrapolate beyond expressions seen during training.

    In contrast to these approaches, we focus on using as little temporal
    training data as possible (\ie five frames) while ensuring generalization.
    For this reason, we build our method on top of
    VolTeMorph~\cite{garbin2024voltemorph}, that uses a parametric model of
    the face to track the deformation of points in a volume around the face
    and builds a radiance field controlled by the parameters of a 3DMM.
    After training, the user can render an image for any expression of the
    face model.
    However, the approach cannot generate expression-dependent high-frequency
    details; see~\cref{fig:blendfields-teaser}.

    Similarly, NeRF-Editing~\cite{yuan2022nerf} and NeRF
    Cages~\cite{xu2022deforming} propose to use tetrahedral meshes to deform a
    single-frame NeRF reconstruction.
    % \ve{This sentence is incomplete :-)} predicts the deformation of a coarse
    % geometric cage and computes the displacement of contained geometry using
    % mean value coordinates~\cite{meanValueCoordinatesPaper}. 
    The resolution of the rendered scenes in these methods is limited by the
    resolution of the tetrahedral cage, which is constrained to a few thousand
    elements.

    We discuss additional concurrent works in \supplementary{}.

  \subsection{Tetrahedral Cages}
    To apply parametric mesh models, it is necessary to extend them to the
    volume to support the volumetric representation of NeRF.
    Tetrahedral cages are a common choice for their simplicity and ubiquity in
    computer
    graphics~\cite{garbin2024voltemorph,yang2022neumesh,xu2022deforming}.
    For example, VolTeMorph uses dense landmarks~\cite{wood20223d} to fit a
    parametric face model whose blendshapes have been extended to a
    tetrahedral cage with finite elements method~\cite{clough1960thefe}.
    These cages can be quickly deformed and
    raytraced~\cite{molino2003tetrahedral} using parallel computation on
    GPUs~\cite{cook2012cuda} while driving the volume into the target pose and
    allowing early ray termination for fast rendering.
    We further leverage the tetrahedral cage and use its differential
    properties~\cite{irving2004invertible}, such as a local volume change, to
    model high-frequency details.
    For example, a change from one expression to another changes the volume of
    tetrahedra in regions where wrinkle formation takes place while it remains
    unchanged in flat areas.
    We can use this change in volume to select which of the trained NeRF
    expressions should be used for each tetrahedron to render high-frequency
    details.
