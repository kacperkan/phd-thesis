
\renewcommand{\imagewidth}{2.7cm}
\renewcommand{\imageheight}{4.0263671874999996cm}
\renewcommand{\smallimagewidth}{0.8cm}
\renewcommand{\expressionsep}{0em}

\renewcommand{\firstindex}{0580}
\renewcommand{\secondindex}{0378}
\renewcommand{\thirdindex}{0226}
\renewcommand{\fourthindex}{0720}

\renewcommand{\supimageoneindex}[1]{
  \includegraphics[width=\imagewidth, trim={0 0 0 0}, clip]{assets/\blendfieldsdirname/supplementary/baselines/#1_2183941_selected_rom_\firstindex.png}
}
\renewcommand{\supimagetwoindex}[1]{
  \includegraphics[width=\imagewidth, trim={0 0 0 0}, clip]{assets/\blendfieldsdirname/supplementary/baselines/#1_5372021_selected_rom_\secondindex.png}
}
\renewcommand{\supimagethreeindex}[1]{
  \includegraphics[width=\imagewidth, trim={0 0 0 0}, clip]{assets/\blendfieldsdirname/supplementary/baselines/#1_6795937_selected_rom_\thirdindex.png}
}
\renewcommand{\supimagefourindex}[1]{
  \includegraphics[width=\imagewidth, trim={0 0 0 0}, clip]{assets/\blendfieldsdirname/supplementary/baselines/#1_7889059_selected_rom_\fourthindex.png}
}

\renewcommand{\versionone}{
  \begin{tikzpicture}[
      >=stealth',
      overlay/.style={
          anchor=south west,
          draw=black,
          rectangle,
          line width=0.8pt,
          outer sep=0,
          inner sep=0,
        },
    ]
    \matrix[
      matrix of nodes,
      column sep=0pt,
      row sep=0pt,
      ampersand replacement=\&,
      inner sep=0,
      outer sep=0,
      draw=black,
      rectangle,
      line width=0.5pt
    ] (gt) {
      \supimageoneindex{gt}   \\
      \supimagetwoindex{gt}   \\
      \supimagethreeindex{gt} \\
      \supimagefourindex{gt}  \\
    };

    \matrix[
      matrix of nodes,
      column sep=0pt,
      row sep=0pt,
      ampersand replacement=\&,
      inner sep=0,
      outer sep=0,
      right=0pt of gt-1-1.east,
      draw=black,
      rectangle,
      line width=0.5pt
    ] (k1) {
      \supimageoneindex{bv_aux} \&
      \supimageoneindex{nerf}   \&
      \supimageoneindex{dnerf} \&
      \supimageoneindex{nerfies} \&
      \supimageoneindex{hypernerf_static}    \&
      \supimageoneindex{hypernerf_dynamic} \\
    };

    \matrix[
      matrix of nodes,
      column sep=0pt,
      row sep=0pt,
      ampersand replacement=\&,
      inner sep=0,
      outer sep=0,
      right=0pt of gt-2-1.east,
      draw=black,
      rectangle,
      line width=0.5pt
    ] (k2) {
      \supimagetwoindex{bv_aux} \&
      \supimagetwoindex{nerf}   \&
      \supimagetwoindex{dnerf} \&
      \supimagetwoindex{nerfies} \&
      \supimagetwoindex{hypernerf_static}    \&
      \supimagetwoindex{hypernerf_dynamic} \\
    };

    \matrix[
      matrix of nodes,
      column sep=0pt,
      row sep=0pt,
      ampersand replacement=\&,
      inner sep=0,
      outer sep=0,
      right=0pt of gt-3-1.east,
      draw=black,
      rectangle,
      line width=0.5pt
    ] (k3) {
      \supimagethreeindex{bv_aux} \&
      \supimagethreeindex{nerf}   \&
      \supimagethreeindex{dnerf} \&
      \supimagethreeindex{nerfies} \&
      \supimagethreeindex{hypernerf_static}    \&
      \supimagethreeindex{hypernerf_dynamic} \\
    };

    \matrix[
      matrix of nodes,
      column sep=0pt,
      row sep=0pt,
      ampersand replacement=\&,
      inner sep=0,
      outer sep=0,
      right=0pt of gt-4-1.east,
      draw=black,
      rectangle,
      line width=0.5pt
    ] (k4) {
      \supimagefourindex{bv_aux} \&
      \supimagefourindex{nerf}   \&
      \supimagefourindex{dnerf} \&
      \supimagefourindex{nerfies} \&
      \supimagefourindex{hypernerf_static}    \&
      \supimagefourindex{hypernerf_dynamic} \\
    };

    \node[above=0em of gt-1-1.north, align=center, anchor=south]{Ground Truth};
    \node[above=0em of k1-1-1.north, align=center, anchor=south]{\textbf{\methodname{}}};
    \node[above=0em of k1-1-2.north, align=center, anchor=south]{NeRF};
    \node[above=0em of k1-1-3.north, align=center, anchor=south]{NeRF+expr};
    \node[above=0em of k1-1-4.north, align=center, anchor=south]{NeRFies};
    \node[above=0em of k1-1-5.north, align=center, anchor=south]{HyperNeRF-AP};
    \node[above=0em of k1-1-6.north, align=center, anchor=south]{HyperNeRF-DS};

    \node[left=0em of gt-1-1.west, align=center, anchor=east]{\rotatebox{90}{\textsc{Subject} 2183941}};
    \node[above=0em of gt-2-1.west, align=center, anchor=east]{\rotatebox{90}{\textsc{Subject} 5372021}};
    \node[above=0em of gt-3-1.west, align=center, anchor=east]{\rotatebox{90}{\textsc{Subject} 6795937}};
    \node[above=0em of gt-4-1.west, align=center, anchor=east]{\rotatebox{90}{\textsc{Subject} 7889059}};
  \end{tikzpicture}
}

\begin{figure*}[htb]
  \centering
  \resizebox{\linewidth}{!}{\versionone}
  \caption{\textbf{Comparison to strictly data-driven approaches} --
    We compare \blendfields to other baselines that do not rely on mesh-driven rendering: NeRF~\cite{mildenhall2020nerf}, NeRF conditioned on the expression code (NeRF+expr)~\cite{mildenhall2020nerf}, NeRFies~\cite{park2021nerfies}, and HyperNeRF-AP/DS~\cite{park2021hypernerf}.
    As a static model, NeRF converges to an average face from available
    ($\nExpr{=}5$) expressions.
    All other baselines exhibit severe artifacts compared to \blendfields.
    Those baselines rely on the data continuity in the training set (\eg, from
    a video), and cannot generalize to any other expression.
  }
  \label{fig:blendfields-qualitative-other-baselines}
\end{figure*}

