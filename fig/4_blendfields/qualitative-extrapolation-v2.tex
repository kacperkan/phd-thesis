
\renewcommand{\imagewidth}{2.3cm}
\renewcommand{\imageheight}{4.0263671874999996cm}
\renewcommand{\smallimagewidth}{0.8cm}
\newcommand{\expressionsep}{0em}

\newcommand{\imageoneindex}[2]{
  \includegraphics[height=\imageheight, trim={0 0 0 0}, clip]{assets/\blendfieldsdirname/qualitative/main_v2/#1_6795937_extrapolation_#2.png}
}
\newcommand{\imagetwoindex}[2]{
  \includegraphics[height=\imageheight, trim={0 0 0 0}, clip]{assets/\blendfieldsdirname/qualitative/main_v2/#1_7889059_extrapolation_#2.png}
}
\renewcommand{\versionone}{
  \tikzsetnextfilename{blendfields_qualitative_extrapolation_v2}
  \begin{tikzpicture}[
      >=stealth',
      overlay/.style={
          anchor=south west,
          draw=black,
          rectangle,
          line width=0.8pt,
          outer sep=0,
          inner sep=0,
        },
    ]
    \matrix[
      matrix of nodes,
      column sep=0pt,
      row sep=0pt,
      ampersand replacement=\&,
      inner sep=0,
      outer sep=0,
    ] (voltemorphstaticone) {
      \imageoneindex{voltemorph_static}{0000} \&
      \imageoneindex{voltemorph_static}{0002} \\
    };

    \matrix[
      matrix of nodes,
      column sep=0pt,
      row sep=0pt,
      ampersand replacement=\&,
      inner sep=0,
      outer sep=0,
      right=\expressionsep of voltemorphstaticone
    ] (voltemorphstatictwo) {
      \imagetwoindex{voltemorph_static}{0000} \&
      \imagetwoindex{voltemorph_static}{0002} \\
    };

    \matrix[
      matrix of nodes,
      column sep=0pt,
      row sep=0pt,
      ampersand replacement=\&,
      inner sep=0,
      outer sep=0,
      below=0pt of voltemorphstaticone
    ] (voltemorphone) {
      \imageoneindex{voltemorph}{0000} \&
      \imageoneindex{voltemorph}{0002} \\
    };

    \matrix[
      matrix of nodes,
      column sep=0pt,
      row sep=0pt,
      ampersand replacement=\&,
      inner sep=0,
      outer sep=0,
      right=\expressionsep of voltemorphone
    ] (voltemorphtwo) {
      \imagetwoindex{voltemorph}{0000} \&
      \imagetwoindex{voltemorph}{0002} \\
    };

    \matrix[
      matrix of nodes,
      column sep=0pt,
      row sep=0pt,
      ampersand replacement=\&,
      inner sep=0,
      outer sep=0,
      below=0pt of voltemorphone
    ] (oursone) {
      \imageoneindex{blendvolumes_aux}{0000} \&
      \imageoneindex{blendvolumes_aux}{0002} \\
    };

    \matrix[
      matrix of nodes,
      column sep=0pt,
      row sep=0pt,
      ampersand replacement=\&,
      inner sep=0,
      outer sep=0,
      right=\expressionsep of oursone
    ] (ourstwo) {
      \imagetwoindex{blendvolumes_aux}{0000} \&
      \imagetwoindex{blendvolumes_aux}{0002} \\
    };
    \node[above=0.0em of voltemorphstaticone-1-1.north, align=center, anchor=south]{Neutral};
    \node[above=0.0em of voltemorphstaticone-1-2.north, align=center, anchor=south]{Frowned and Smiled};

    \node[above=0.0em of voltemorphstatictwo-1-1.north, align=center, anchor=south]{Neutral};
    \node[above=0.0em of voltemorphstatictwo-1-2.north, align=center, anchor=south]{Eyes Squint, Lips Moved Left};

    % \node[above=0.0em of voltemorphone-1-1.north, align=center,
    % anchor=south]{Neutral}; \node[above=0.0em of voltemorphone-1-2.north,
    % align=center, anchor=south]{Eyes Squint, Lips Moved Left};

    % \node[above=0.0em of hypernerfstatic-1-1.north, align=center,
    % anchor=south]{HyperNeRF-AP}; \node[above=0.0em of
    % hypernerfdynamic-1-1.north, align=center, anchor=south]{HyperNeRF-DS};
    % \node[above=0.0em of voltemorphstatic-1-1.north, align=center,
    % anchor=south]{VolTeMorph$^\dagger$}; \node[above=0.0em of
    % voltemorph-1-1.north, align=center, anchor=south]{VolTeMorph};
    % \node[above=0.2em of ours-1-1.north, align=center,
    % anchor=south]{\textbf{Ours}};

    % \node[left=0em of voltemorphstaticone-1-1.west, align=center,
    % anchor=east]{\rotatebox{90}{Neutral}}; \node[left=0em of nerf-3-1.west,
    % align=center, anchor=east]{\rotatebox{90}{Neutral}};

    % \node[left=0em of nerf-2-1.west, align=center,
    % anchor=east]{\rotatebox{90}{Frowned and Smile}}; \node[left=0em of
    % nerf-4-1.west, align=center, anchor=east]{\rotatebox{90}{Eyes Squint,
    % Lips Moved Left}};

    \node[left=0em of voltemorphstaticone-1-1.west, align=center, anchor=east]{\rotatebox{90}{VolTeMorph$_1$}};
    \node[left=0em of voltemorphone-1-1.west, align=center, anchor=east]{\rotatebox{90}{VolTeMorph$_\text{avg}$}};
    \node[left=0em of oursone-1-1.west, align=center, anchor=east]{\rotatebox{90}{\textbf{Ours}}};

    % \node[left=0em of nerf-2-1.west, align=center,
    % anchor=east]{\rotatebox{90}{Frowned and Smile}}; \node[left=0em of
    % nerf-4-1.west, align=center, anchor=east]{\rotatebox{90}{Eyes Squint,
    % Lips Moved Left}};
  \end{tikzpicture}
}

\begin{figure*}[htb]
  \centering
  \resizebox{\linewidth}{!}{
    \versionone
  }
  \caption{\textbf{Novel expression synthesis} -- {
      We compare qualitatively \blendfields with selected baselines (vertical)
      across two selected subjects (horizontal).
      To our surprise, VolTeMorph$_\text{avg}$ trained on multiple frames
      renders some details too but with much lower fidelity.
      We argue that VolTeMorph considers rendering wrinkles as artifacts that
      depend on the view direction
      (see~\cref{eq:blendfields-rendering-equation}).
      VolTeMorph$_1$ is limited to producing wrinkles it was trained for.
      In contrast to those baselines, \textbf{\blendfields} captures the
      details and generalizes outside of the distribution.
      Please refer to the \supplementary{} for animated sequences and results
      for other methods.
    }}
  \label{fig:blendfields-qualitative-comparison}
\end{figure*}