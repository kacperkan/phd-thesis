
\renewcommand{\imagewidth}{2.3cm}
\renewcommand{\imageheight}{4.0263671874999996cm}
\renewcommand{\smallimagewidth}{0.8cm}
\newcommand{\expressionsep}{0em}

\newcommand{\imageoneindex}[2]{
  \includegraphics[height=\imageheight, trim={0 0 0 0}, clip]{assets/\blendfieldsdirname/qualitative/main_v2/#1_6795937_extrapolation_#2.png}
}
\newcommand{\imagetwoindex}[2]{
  \includegraphics[height=\imageheight, trim={0 0 0 0}, clip]{assets/\blendfieldsdirname/qualitative/main_v2/#1_7889059_extrapolation_#2.png}
}
\newcommand{\versionone}{
  \begin{tikzpicture}[
      >=stealth',
      overlay/.style={
          anchor=south west,
          draw=black,
          rectangle,
          line width=0.8pt,
          outer sep=0,
          inner sep=0,
        },
    ]
    \matrix[
      matrix of nodes,
      column sep=0pt,
      row sep=0pt,
      ampersand replacement=\&,
      inner sep=0,
      outer sep=0
    ] (gt) {
      \imageoneindex{gt}{0505} \&
      \imageoneindex{gt}{0000} \&
      \imagetwoindex{gt}{0000} \&
      \imagetwoindex{gt}{0852} \\
    };
    \matrix[
      matrix of nodes,
      column sep=0pt,
      row sep=0pt,
      ampersand replacement=\&,
      inner sep=0,
      outer sep=0,
      below=0pt of gt,
    ] (voltemorphstatic) {
      \imageoneindex{voltemorph_static}{0505} \&
      \imageoneindex{voltemorph_static}{0000} \&
      \imagetwoindex{voltemorph_static}{0000} \&
      \imagetwoindex{voltemorph_static}{0852} \\
    };

    \matrix[
      matrix of nodes,
      column sep=0pt,
      row sep=0pt,
      ampersand replacement=\&,
      inner sep=0,
      outer sep=0,
      below=0pt of voltemorphstatic
    ] (voltemorph) {
      \imageoneindex{voltemorph}{0505} \&
      \imageoneindex{voltemorph}{0000} \&
      \imagetwoindex{voltemorph}{0000} \&
      \imagetwoindex{voltemorph}{0852} \\
    };

    \matrix[
      matrix of nodes,
      column sep=0pt,
      row sep=0pt,
      ampersand replacement=\&,
      inner sep=0,
      outer sep=0,
      below=0pt of voltemorph
    ] (ours) {
      \imageoneindex{bv_aux}{0505} \&
      \imageoneindex{bv_aux}{0000} \&
      \imagetwoindex{bv_aux}{0000} \&
      \imagetwoindex{bv_aux}{0852} \\
    };

    \node[left=0em of gt-1-1.west, align=center, anchor=east]{\rotatebox{90}{Ground Truth}};
    \node[left=0em of voltemorphstatic-1-1.west, align=center, anchor=east]{\rotatebox{90}{VolTeMorph$_1$}};
    \node[left=0em of voltemorph-1-1.west, align=center, anchor=east]{\rotatebox{90}{VolTeMorph$_\text{avg}$}};
    \node[left=0em of ours-1-1.west, align=center, anchor=east]{\rotatebox{90}{\textbf{Ours}}};

    \node[above=0em of gt-1-1.north, align=center, anchor=south]{Neutral};
    \node[above=0em of gt-1-2.north, align=center, anchor=south]{Posed};
    \node[above=0em of gt-1-3.north, align=center, anchor=south]{Neutral};
    \node[above=0em of gt-1-4.north, align=center, anchor=south]{Posed};
  \end{tikzpicture}
}

\begin{figure*}[htb]
  \centering
  \resizebox{\linewidth}{!}{\versionone}
  \caption{\textbf{Novel expression synthesis} --
    We compare qualitatively \blendfields with selected baselines (vertical) across two selected subjects (horizontal).
    Firstly, we show a neutral pose of the subject and then any of the
    available expressions.
    To our surprise, VolTeMorph$_\text{avg}$ trained on multiple frames
    renders some details but with much lower fidelity.
    We argue that VolTeMorph$_\text{arg}$ considers rendering wrinkles as
    artifacts that depend on the view direction
    (see~\cref{eq:blendfields-rendering-equation}).
    VolTeMorph$_1$ is limited to producing the wrinkles it was trained for.
    In contrast to those baselines, \textbf{\blendfields} captures the details
    and generalizes outside of the distribution.
    Please refer to the \supplementary{} for animated sequences and results
    for other methods.
  }
  \label{fig:blendfields-qualitative-comparison}
\end{figure*}