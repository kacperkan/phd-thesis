\renewcommand{\imagewidth}{2.3cm}
\renewcommand{\imageheight}{4.5263671874999996cm}
\renewcommand{\smallimagewidth}{0.8cm}

\newcommand{\imageoneindex}[2]{
  \includegraphics[height=\imageheight, trim={0.7cm 1.8cm 0.7cm 0}, clip]{assets/\blendfieldsdirname/qualitative/extrapolation/#1_6795937_extrapolation_#2.png}
}
\newcommand{\imagetwoindex}[2]{
  \includegraphics[height=\imageheight, trim={1.4cm 1.8cm 0 0}, clip]{assets/\blendfieldsdirname/qualitative/extrapolation/#1_7889059_extrapolation_#2.png}
}
\renewcommand{\versionone}{
  \tikzsetnextfilename{blendfields_qualitative_comparison_extrapolation}
  \begin{tikzpicture}[
      >=stealth',
      overlay/.style={
          anchor=south west,
          draw=black,
          rectangle,
          line width=0.8pt,
          outer sep=0,
          inner sep=0,
        },
    ]
    \matrix[
      matrix of nodes,
      column sep=0pt,
      row sep=0pt,
      ampersand replacement=\&,
      inner sep=0,
      outer sep=0
    ] (nerf) {
      \imageoneindex{nerf}{0000} \\
      % \imageoneindex{nerf}{0002} \\
      \imageoneindex{nerf}{0004} \\

      \imagetwoindex{nerf}{0000} \\
      % \imagetwoindex{nerf}{0002} \\
      \imagetwoindex{nerf}{0004} \\
    };

    \matrix[
      matrix of nodes,
      column sep=0pt,
      row sep=0pt,
      ampersand replacement=\&,
      inner sep=0,
      outer sep=0,
      right=1pt of nerf
    ] (dnerf) {
      \imageoneindex{dnerf}{0000} \\
      % \imageoneindex{dnerf}{0002} \\
      \imageoneindex{dnerf}{0004} \\

      \imagetwoindex{dnerf}{0000} \\
      % \imagetwoindex{dnerf}{0002} \\
      \imagetwoindex{dnerf}{0004} \\
    };

    \matrix[
      matrix of nodes,
      column sep=0pt,
      row sep=0pt,
      ampersand replacement=\&,
      inner sep=0,
      outer sep=0,
      right=1pt of dnerf
    ] (nerfies) {
      \imageoneindex{nerfies}{0000} \\
      % \imageoneindex{nerfies}{0002} \\
      \imageoneindex{nerfies}{0004} \\

      \imagetwoindex{nerfies}{0000} \\
      % \imagetwoindex{nerfies}{0002} \\
      \imagetwoindex{nerfies}{0004} \\
    };

    \matrix[
      matrix of nodes,
      column sep=0pt,
      row sep=0pt,
      ampersand replacement=\&,
      inner sep=0,
      outer sep=0,
      right=1pt of nerfies
    ] (hypernerfstatic) {
      \imageoneindex{hypernerf_static}{0000} \\
      % \imageoneindex{hypernerf_static}{0002} \\
      \imageoneindex{hypernerf_static}{0004} \\

      \imagetwoindex{hypernerf_static}{0000} \\
      % \imagetwoindex{hypernerf_static}{0002} \\
      \imagetwoindex{hypernerf_static}{0004} \\
    };

    \matrix[
      matrix of nodes,
      column sep=0pt,
      row sep=0pt,
      ampersand replacement=\&,
      inner sep=0,
      outer sep=0,
      right=1pt of hypernerfstatic
    ] (hypernerfdynamic) {
      \imageoneindex{hypernerf_dynamic}{0000} \\
      % \imageoneindex{hypernerf_dynamic}{0002} \\
      \imageoneindex{hypernerf_dynamic}{0004} \\

      \imagetwoindex{hypernerf_dynamic}{0000} \\
      % \imagetwoindex{hypernerf_dynamic}{0002} \\
      \imagetwoindex{hypernerf_dynamic}{0004} \\
    };

    \matrix[
      matrix of nodes,
      column sep=0pt,
      row sep=0pt,
      ampersand replacement=\&,
      inner sep=0,
      outer sep=0,
      right=1pt of hypernerfdynamic
    ] (voltemorphstatic) {
      \imageoneindex{voltemorph_static}{0000} \\
      % \imageoneindex{voltemorph_static}{0002} \\
      \imageoneindex{voltemorph_static}{0004} \\

      \imagetwoindex{voltemorph_static}{0000} \\
      % \imagetwoindex{voltemorph_static}{0002} \\
      \imagetwoindex{voltemorph_static}{0004} \\
    };

    \matrix[
      matrix of nodes,
      column sep=0pt,
      row sep=0pt,
      ampersand replacement=\&,
      inner sep=0,
      outer sep=0,
      right=1pt of voltemorphstatic
    ] (voltemorph) {
      \imageoneindex{voltemorph}{0000} \\
      % \imageoneindex{voltemorph}{0002} \\
      \imageoneindex{voltemorph}{0004} \\

      \imagetwoindex{voltemorph}{0000} \\
      % \imagetwoindex{voltemorph}{0002} \\
      \imagetwoindex{voltemorph}{0004} \\
    };

    \matrix[
      matrix of nodes,
      column sep=0pt,
      row sep=0pt,
      ampersand replacement=\&,
      inner sep=0,
      outer sep=0,
      right=1pt of voltemorph
    ] (ours) {
      \imageoneindex{blendvolumes_aux}{0000} \\
      % \imageoneindex{blendvolumes_aux}{0002} \\
      \imageoneindex{blendvolumes_aux}{0004} \\

      \imagetwoindex{blendvolumes_aux}{0000} \\
      % \imagetwoindex{blendvolumes_aux}{0002} \\
      \imagetwoindex{blendvolumes_aux}{0004} \\
    };
    \node[above=0.2em of nerf-1-1.north, align=center, anchor=south]{NeRF};
    \node[above=0.2em of dnerf-1-1.north, align=center, anchor=south]{Conditioned NeRF};
    \node[above=0.2em of nerfies-1-1.north, align=center, anchor=south]{NeRFies};
    \node[above=0.0em of hypernerfstatic-1-1.north, align=center, anchor=south]{HyperNeRF-AP};
    \node[above=0.0em of hypernerfdynamic-1-1.north, align=center, anchor=south]{HyperNeRF-DS};
    \node[above=0.0em of voltemorphstatic-1-1.north, align=center, anchor=south]{VolTeMorph$^\dagger$};
    \node[above=0.0em of voltemorph-1-1.north, align=center, anchor=south]{VolTeMorph};
    \node[above=0.2em of ours-1-1.north, align=center, anchor=south]{\textbf{Ours}};

    \node[left=0em of nerf-1-1.west, align=center, anchor=east]{\rotatebox{90}{Neutral}};
    \node[left=0em of nerf-3-1.west, align=center, anchor=east]{\rotatebox{90}{Neutral}};

    \node[left=0em of nerf-2-1.west, align=center, anchor=east]{\rotatebox{90}{Frowned and Smile}};
    \node[left=0em of nerf-4-1.west, align=center, anchor=east]{\rotatebox{90}{Eyes Squint, Lips Moved Left}};
  \end{tikzpicture}
}

\begin{figure*}[htb]
  \centering
  \resizebox{\linewidth}{!}{
    \versionone
  }
  % \includegraphics[height=0.75\textheight, width=\linewidth]{example-image-a}
  \caption{\textbf{Novel expression synthesis} --
    We compare qualitatively \blendfields with selected baselines (horizontal) across two selected subjects (vertical).
    As clearly seen, our approach renders the most realistic frames given any
    of the expressions.
    VolTeMorph, while being capable of rendering already realistic, controlled
    images, it cannot capture expression-dependent details.
    In contrast, \blendfields captures these details and generalizes outside
    of the distribution.
    Please refer to the \supplementary{} for animated sequences and results
    for other methods.
  }
  \label{fig:blendfields-qualitative-comparison}
\end{figure*}