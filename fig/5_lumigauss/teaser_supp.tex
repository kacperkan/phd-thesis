\def\firstrowimageheight{724px}
\def\secondrowimagewidth{536px}
\def\imageheight{2cm}
\newcommand{\teasermanual}{
  \begin{subfigure}[b]{\linewidth}
    \centering
    \resizebox{\linewidth}{!}{

      \includegraphics[width=\linewidth]{images/wild2d-ablation.pdf}
      \label{fig:lumigauss-ablation_img}

    }
  \end{subfigure}
}
\newcommand{\teaserfigure}{
  \captionsetup{type=figure}

  \fboxsep=0pt % padding thickness
  \fboxrule=0.4pt % border thickness

  % \begin{subfigure}[b]{\linewidth} \centering
  % \includegraphics[width=\linewidth]{images/wild2d-Teaser_v3.pdf}
  % \end{subfigure}
  \vspace{-1em}
  \teasermanual
  \vspace{-1.6em}

  \setcounter{figure}{0} % Not sure why this is necessary.
  \captionof{figure}{
    {
        \textbf{Ablation study for relightning with external environment map.}
        The full model results in the clearest render.
        The strongest quality drop is observed when components restricting
        $\transferfun_k$ are omitted.
        % \textbf{Teaser} -- \ours{} reconstructs environment maps and object's
        % surface from \textit{in-the-wild} images. Our model decouples the
        % scene color and its normals (\textit{second and fourth column in the
        % top row}). At inference, it can synthesize novel views
        % (\textit{bottom row}) and realistic lighting (\textit{first and third
        % columns in the bottom}) with high-fidelity shadows (\textit{second
        % and fourth columns in the bottom}).
      }
    % \ours~reconstruction and relightning. Top row: Original image,
    % reconstruction, normals. Bottom rows: New viewpoints relighted with
    % external environment map from [x-envmap paper]. Unshadowed relightning,
    % irradiance difference, shadowed relightning. \asia{we have bigger teaser
    % with illuminance and unshadwoed version, but it was too much info to
    % process. Unshadowed version moved below}
  }
  \vspace{1em}
  \label{fig:teaser}
}
