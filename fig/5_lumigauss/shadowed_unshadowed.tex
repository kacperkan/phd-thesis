\def\scale{1.0}
\def\distbelow{2.5em}
\def\distabove{2.5em}
\def\height{256bp}

\newcommand{\shadowedoverpicone}{
  \begin{overpic}[abs,height=\height]{\lumigaussdirname/folder_for_kacper/wild2d-shadowed_unshadowed_fig1.pdf}
    \put(90,200){$\normal$}
    \put(155,240){$\direction_1$}
    \put(65,130){$\direction_2$}
  \end{overpic}
}

\newcommand{\shadowedoverpictwo}{
  \begin{overpic}[abs,height=\height]{\lumigaussdirname/folder_for_kacper/wild2d-shadowed_unshadowed_fig2.pdf}
    \put(155,200){$\normal$}
    \put(220,240){$\direction_1$}
    \put(130,130){$\direction_2$}
  \end{overpic}
}

\newcommand{\shadowedoverpicthree}{
  \begin{overpic}[abs,height=\height]{\lumigaussdirname/folder_for_kacper/wild2d-shadowed_unshadowed_fig3.pdf}
    \put(155,200){$\normal$}
    \put(220,240){$\direction_1$}
    \put(130,130){$\direction_2$}
  \end{overpic}
}
\newcommand{\shadowedunshadowed}{
  {
      \tikzset{external/export=false}
      \tikzsetnextfilename{shadowed_unshadowed}
      \resizebox{0.9\linewidth}{!}{
        \begin{tikzpicture}[
            >=stealth',
            overlay/.style={
                anchor=south west,
                draw=black,
                rectangle,
                line width=0.8pt,
                outer sep=0,
                inner sep=0,
              },
            font=\Huge
          ]
          \matrix[matrix of nodes, column sep=1pt, row sep=0pt, ampersand replacement=\&, inner sep=0, outer sep=0, font=\Huge] (shadowedunshadowed) {
            \shadowedoverpicone{} \&
            \shadowedoverpictwo{} \&
            \shadowedoverpicthree{} \\
          };
          \node[below=\distbelow of shadowedunshadowed-1-1.south, anchor=base, scale=\scale] {
            $\begin{aligned}
                \transferfunsh(\direction_1) & {=}\max(0, \cos(\normal, \direction_1)) \\
                \transferfunsh(\direction_2) & {=}\max(0, \cos(\normal, \direction_2))
              \end{aligned}$
          };

          \node[below=\distbelow of shadowedunshadowed-1-2.south, anchor=base, scale=\scale] {
            $\begin{aligned}
                \transferfunsh(\direction_1) & {=}\max(0, \cos(\normal, \direction_1)) \\
                \transferfunsh(\direction_2) & {=}\max(0, \cos(\normal, \direction_2))
              \end{aligned}$
          };

          \node[below=\distbelow of shadowedunshadowed-1-3.south, anchor=base, scale=\scale] {
            $\begin{aligned}
                \transferfunsh(\direction_1) & {=}\max(0, \cos(\normal, \direction_1)) \\
                \transferfunsh(\direction_2) & {<}\max(0, \cos(\normal, \direction_2))
              \end{aligned}$
          };

          \node[above=\distabove of shadowedunshadowed-1-1.north, anchor=base, scale=\scale, align=center]{
            % Unshadowed \\ = shadowed radiance \\
            $\radiance = \shadowedradiance$ \\
            (\cref{eq:lumigauss-unshadowed-radiance} \& \cref{eq:lumigauss-shadowed-radiance})
          };

          \node[above=\distabove of shadowedunshadowed-1-2.north, anchor=base, scale=\scale, align=center]{
            $\radiance$ does not incorporate \\
            the light blocker
          };

          \node[above=\distabove of shadowedunshadowed-1-3.north, anchor=base, scale=\scale, align=center]{
            $\shadowedradiance$ incorporates \\
            the light blocker
          };

        \end{tikzpicture}
      }
    }
}

\begin{figure}[!t]
  \centering
  % \includegraphics[width=\linewidth]{\lumigaussdirname/wild2d-shadowed_unshadowed.pdf}
  \shadowedunshadowed
  \caption{
    Unshadowed $\radiance$ (\cref{eq:lumigauss-unshadowed-radiance}) and
    shadowed $\shadowedradiance$ (\cref{eq:lumigauss-shadowed-radiance}) may
    give the same output color if a Gaussian is fully exposed to the
    environment light.
    In the case of any occluder, $\radiance$ does not handle, and the color
    does not change.
    However, our proposed $\shadowedradiance$ properly reacts to the occluder
    and makes the output color darker.
  }

  \label{fig:lumigauss-shadowed_unshadowed}
\end{figure}