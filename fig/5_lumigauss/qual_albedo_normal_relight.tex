\renewcommand{\inputimage}[1]{\includegraphics[height=\height,trim={8bp 0 8bp 0}, clip]{\lumigaussassets/folder_for_kacper/qual_relight_albedo_norm/#1}}
\renewcommand{\inputimageshadows}[1]{\includegraphics[height=\height,trim={8bp 0 8bp 0}, clip]{\lumigaussassets/folder_for_kacper/qual_gt_shadows/#1}}

\newcommand{\newqualitative}{
  \centering
  \tikzsetnextfilename{qual_albedo_normal_relight}
  \tikzset{external/export=false}
  \resizebox{\linewidth}{!}{
    {
        \def\height{64bp}
        \def\spysize{24bp}
        \def\offset{36bp}
        \def\scale{0.7}
        \def\scaleleft{0.7}
        \def\distabove{0.3em}
        \def\distleft{0.0em}
        \def\clipleft{2cm}
        \def\clipright{2cm}
        \begin{tikzpicture}[
            >=stealth',
            title/.style={
                anchor=base,
                align=center,
                scale=\scale,
              },
            spy using outlines={circle, red, magnification=3, connect spies, size=\spysize}
          ]
          \matrix[matrix of nodes, column sep=0pt, row sep=0pt, ampersand replacement=\&, inner sep=0, outer sep=0, font=\Huge] (qualitative) {
            \inputimage{osr-reconstruction.png} \&
            \inputimage{osr-albedo.png} \&
            \inputimage{osr-normals.png} \&
            \inputimage{osr-relit.png}       \\
            \inputimage{neusky-reconstruction.png} \&
            \inputimage{neusky-albedo.png} \&
            \inputimage{neusky-normals.png} \&
            \inputimage{neusky-relit.png}    \\
            \inputimage{lumigauss-reconstruction.png} \&
            \inputimage{lumigauss-albedo.png} \&
            \inputimage{lumigauss-normals.png} \&
            \inputimage{lumigauss-relit.png} \\
            % \inputimage{gt_1.jpg} \& \inputimage{ours_1_relit.png} \&
            % \inputimage{osr_1.jpg} \& \inputimage{srtensor_1.png} \\
            % \inputimage{gt_2.jpg} \& \inputimage{ours_2_relit.png} \&
            % \inputimage{osr_2.jpg} \& \inputimage{srtensor_2.png} \\
            % \inputimage{gt_3.jpg} \& \inputimage{ours_3_relit.png} \&
            % \inputimage{osr_3.jpg} \& \inputimage{srtensor_3.png} \\
          };

          \node[above=\distabove of qualitative-1-1.north, title] {Reconstruction};
          \node[above=\distabove of qualitative-1-2.north, title] {Albedo};
          \node[above=\distabove of qualitative-1-3.north, title] {Normals};
          \node[above=\distabove of qualitative-1-4.north, title] {Novel Lightning};

          \node[left=\distleft of qualitative-1-1.west, align=center,scale=\scaleleft]{\rotatebox{90}{NeRF-OSR~\cite{rudnev2022nerfosr}}};
          \node[left=\distleft of qualitative-2-1.west, align=center,scale=\scaleleft]{\rotatebox{90}{NeuSky~\cite{gardner2023neusky}}};
          \node[left=\distleft of qualitative-3-1.west, align=center,scale=\scaleleft]{\rotatebox{90}{\textbf{Ours}}};

          \foreach \n in {1,...,3}{
              \edef\temp{\noexpand\spy[anchor=center,magnification=3] on ([xshift=0.0em,yshift=-0.5em] qualitative-\n-1.center) in node[below left=3.75bp and 3.75bp] at (qualitative-\n-1.north east);}
              \temp
            }
          % https://tex.stackexchange.com/questions/170664/foreach-not-behaving-in-axis-environment
          \foreach \n in {1,...,3}{
              \edef\temp{\noexpand\spy[anchor=center,magnification=3] on ([xshift=-1.0em,yshift=0.0em] qualitative-\n-2.center) in node[below left=3.75bp and 3.75bp] at (qualitative-\n-2.north east);}
              \temp
            }
          \foreach \n in {1,...,3}{
              \edef\temp{\noexpand\spy[anchor=center,magnification=3] on ([xshift=1.0em,yshift=0.5em] qualitative-\n-3.center) in node[below left=3.75bp and 3.75bp] at (qualitative-\n-3.north east);}
              \temp
            }
          \foreach \n in {1,...,3}{
              \edef\temp{\noexpand\spy[anchor=center,magnification=3] on ([xshift=-0.7em,yshift=-2em] qualitative-\n-4.center) in node[below left=3.75bp and 3.75bp] at (qualitative-\n-4.north east);}
              \temp
            }
        \end{tikzpicture}
      }
  }
}

\begin{figure*}[!t]
  \centering
  \newqualitative
  % \includegraphics[width=\linewidth]{images/folder_for_kacper/qual_relight_albedo_norm/wild2d-qual_relight.pdf}
  \caption{\textbf{Qualitative comparison of albedo, normals, and relighting under similar lighting conditions on Trevi Fountain.} \lumigauss produces albedo with fewer baked-in shadows, sharp normals, smooth surfaces, and more accurate novel lighting compared to the baselines.
    Results for NeuSky originally reported in \cite{gardner2023neusky}.
    Please, zoom in for details.
  }
  \label{fig:lumigauss-qual_albedo_norm_relight}
\end{figure*}