\newcommand{\inpscene}[1]{\includegraphics[height=\height]{\lumigaussdirname/folder_for_kacper/relighting/#1.png}}

\newcommand{\relighting}{
  \centering
  \tikzsetnextfilename{qualitative_ours}
  {
    \def\spysize{24px}
    \def\offset{36px}
    \def\height{64px}
    \resizebox{\linewidth}{!}{
      \begin{tikzpicture}[
          >=stealth',
          overlay/.style={
              anchor=south west,
              draw=black,
              rectangle,
              line width=0.8pt,
              outer sep=0,
              inner sep=0,
            },
          font=\Huge
        ]
        \matrix[
          matrix of nodes,
          column sep=0pt,
          row sep=0pt,
          ampersand replacement=\&,
          inner sep=0,
          outer sep=0
        ] (pictures) {
          \inpscene{orig_1} \&
          \inpscene{rec_1} \&
          \inpscene{novel_1} \\
          \inpscene{orig_2} \&
          \inpscene{rec_2} \&
          \inpscene{novel_2} \\
          \inpscene{orig_3} \&
          \inpscene{rec_3} \&
          \inpscene{novel_3} \\
          \inpscene{orig_4} \&
          \inpscene{rec_4} \&
          \inpscene{novel_4} \\
        };
        \matrix[
          matrix of nodes,
          column sep=0pt,
          row sep=0pt,
          ampersand replacement=\&,
          inner sep=0,
          outer sep=0,
          right=of pictures
        ] (chair) {
          \inpscene{obj_1} \\
          \inpscene{obj_2} \\
          \inpscene{obj_3} \\
          \inpscene{obj_4} \\
        };
        \matrix[
          matrix of nodes,
          column sep=0pt,
          row sep=0pt,
          ampersand replacement=\&,
          inner sep=0,
          outer sep=0,
          right=of chair
        ] (chair) {
          \inpscene{fic_1} \\
          \inpscene{fic_2} \\
          \inpscene{fic_3} \\
          \inpscene{fic_4} \\
        };

      \end{tikzpicture}
    }
  }
}

\begin{figure}[!t]
  \centering
  % \begin{overpic}[abs,width=\linewidth]{images/folder_for_kacper/relighting/relit.pdf}
  \begin{overpic}[percent,width=\linewidth]{\lumigaussdirname/folder_for_kacper/relighting/wild2d-relight_items_3items.pdf}
    \put(5,49.5) {\footnotesize Target Image}
    \put(26.5,49.5) {\footnotesize Reconstruction}
    \put(49.5,49.5) {\footnotesize Novel View}
    \put(75.2,49.5) {\footnotesize Relit Objects}
  \end{overpic}
  \vspace{-3em}
  % \relighting
  \caption{\textbf{Scene reconstruction and relightning --}
    Reconstruction and relighting capabilities of \lumigauss.
    \lumigauss reproduces sharp and clean landmarks, and the learned environment lighting enables accurate scene relighting.
    We use learned environment maps to relight the scene from novel viewpoints
    and then relight arbitrary objects within a graphics engine.
  }
  \label{fig:lumigauss-qualitative_ours}
\end{figure}