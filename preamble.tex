\usepackage{graphicx}
\usepackage{relsize}
\usepackage{booktabs}
\usepackage{color}
\usepackage{multirow}
\usepackage{bm}
\usepackage[percent]{overpic}
\usepackage{tcolorbox}

\usepackage{anyfontsize}
\usepackage{pgfplots}
\usepackage[normalem]{ulem}
\usepackage[dvipsnames]{xcolor}
\usepackage{colortbl}
\definecolor{cvprblue}{rgb}{0.21,0.49,0.74}
% \usepackage[pagebackref,breaklinks,colorlinks,allcolors=cvprblue]{hyperref}
\usepackage{tikz}

\def\mathdefault#1{#1}
\everymath=\expandafter{\the\everymath\displaystyle}
\pgfplotsset{compat=1.18}

\usetikzlibrary{calendar,fpu,matrix, positioning,arrows,arrows.meta,calc,decorations.pathreplacing,decorations.text,spy}
\tikzset{
  % style for inserting images as nodes
  img/.style={
      % text width=2cm, %% don't use this text height=2cm, %% don't use this
      inner sep=0pt,     % % use this
      outer sep=0pt,     % % and this
      rectangle,
      align=center} % only for debugging..
}

\newcommand{\KK}[1]{\textbf{\color{teal}[KK: #1]}}
\newcommand{\kk}[1]{{\color{teal}#1}}

\newcommand{\todo}[1]{{\color{red}#1}}  % a proposal text, use sparingly
\newcommand{\TODO}[1]{\textbf{\color{red}[TODO: #1]}}

\DeclareMathOperator*{\argmax}{arg\,max}
\DeclareMathOperator*{\argmin}{arg\,min}
\DeclareMathOperator*{\mlp}{MLP}
\DeclareMathOperator*{\upsample}{upsample}
\DeclareMathOperator*{\downsample}{downsample}
\DeclareMathOperator*{\diag}{diag}
\DeclarePairedDelimiter\ceil{\lceil}{\rceil}
\DeclarePairedDelimiter\floor{\lfloor}{\rfloor}

% Support for easy cross-referencing
\usepackage[capitalize]{cleveref}
\crefname{section}{Sec.}{Secs.}
\Crefname{section}{Section}{Sections}
\Crefname{table}{Table}{Tables}
\crefname{table}{Tab.}{Tabs.}

\newcommand\mycoloredbox[1]{\textcolor{#1}{\rule{0.5em}{0.5em}}}
\definecolor{secondbestcolor}{HTML}{16E9CB}
\definecolor{firstbestcolor}{HTML}{169EE9}
\newcommand{\tablefirstbest}[0]{\cellcolor{firstbestcolor}}
\newcommand{\tablesecondbest}[0]{\cellcolor{secondbestcolor}}

\newsavebox\neuralnet
\begin{lrbox}{\neuralnet}
  \begin{tikzpicture}
    \node[circle, thick, fill=white, draw] (x1) {};
    \node[circle, thick, fill=white, draw, below=0.1em of x1] (x2) {};
    \node[circle, thick, fill=white, draw, below=0.1em of x2] (x3) {};
    \node[circle, thick, fill=white, draw, above=0.1em of x1] (x4) {};
    \node[circle, thick, fill=white, draw, above=0.1em of x4] (x5) {};

    \node[circle, thick, fill=white, draw, right=2em of x1] (x11) {};
    \node[circle, thick, fill=white, draw, below=0.1em of x11] (x12) {};
    \node[circle, thick, fill=white, draw, above=0.1em of x11] (x13) {};

    \foreach \x in {1,...,5}
    \foreach \y in {1,...,3}
    \draw (x\x) -- (x1\y);
  \end{tikzpicture}
\end{lrbox}

\usetikzlibrary{calendar,fpu,matrix, positioning,arrows,arrows.meta,calc,decorations.pathreplacing,decorations.text,spy}
\tikzset{
  % style for inserting images as nodes
  img/.style={
      % text width=2cm, %% don't use this text height=2cm, %% don't use this
      inner sep=0pt,     % % use this
      outer sep=0pt,     % % and this
      rectangle,
      align=center} % only for debugging..
}

\tikzset{
  moon colour/.style={
      moon fill/.style={
          fill=#1
        },
      scale=0.5,
    },
  sky colour/.style={
      sky draw/.style={
          draw=#1
        },
      sky fill/.style={
          fill=#1
        }
    },
  southern hemisphere/.style={
      rotate=180
    }
}

\makeatletter
\pgfcalendardatetojulian{2010-01-15}{\c@pgf@counta} % 2010-01-15 07:11 UTC --
% http://aa.usno.navy.mil/cgi-bin/aa_moonphases.pl?year=2010&ZZZ=END
\def\synodicmonth{29.530588853}
\newcommand{\moon}[2][]{
  {
      \tikzset{external/export=false}
      \edef\checkfordate{\noexpand\in@{-}{#2}}
      \checkfordate
      \ifin@
        \pgfcalendardatetojulian{#2}{\c@pgf@countb}% %
        \pgfkeys{/pgf/fpu=true,/pgf/fpu/output format=fixed}% %
        \pgfmathsetmacro\dayssincenewmoon{\the\c@pgf@countb-\the\c@pgf@counta-(7/24+11/(24*60))}% %
        \pgfmathsetmacro\lunarage{mod(\dayssincenewmoon,\synodicmonth)}
        \pgfkeys{/pgf/fpu=false}% %
      \else
        \def\lunarage{#2}
      \fi
      \pgfmathsetmacro\leftside{ifthenelse(\lunarage<=\synodicmonth/2,cos(360*(\lunarage/\synodicmonth)),1)}% %
      \pgfmathsetmacro\rightside{ifthenelse(\lunarage<=\synodicmonth/2,-1,-cos(360*(\lunarage/\synodicmonth))}% %%
      \tikz [moon colour=white,sky colour=black,#1]{
        \draw [moon fill, sky draw] (0,0) circle [radius=1ex];
        \draw [sky draw, sky fill] (0,1ex)
        arc (90:-90:\rightside ex and 1ex)
        arc (-90:90:\leftside ex and 1ex)
        -- cycle;
      }% %%
    }
}
\newcommand{\newmoon}{\moon{0}}
\newcommand{\waxingcrescent}{\moon{\synodicmonth/8}}
\newcommand{\firstquartermoon}{\moon{2*\synodicmonth/8}}
\newcommand{\waxinggibbous}{\moon{3*\synodicmonth/8}}
\newcommand{\fullmoon}{\moon{4*\synodicmonth/8}}
\newcommand{\waninggibbous}{\moon{5*\synodicmonth/8}}
\newcommand{\thirdquartermoon}{\moon{6*\synodicmonth/8}}
\newcommand{\waningcrescent}{\moon{7*\synodicmonth/8}}
\newcommand{\shadowed}{\moon{8}}
\newcommand{\unshadowed}{\moon{16}}
\newcommand{\removespace}[1]{\!\!\!\!\!\!\!\!\!\!#1\!}

\newcommand{\real}{\mathbb{R}}
\newcommand{\latentdimension}{D}
\newcommand{\projecteddimension}{d}

\makeatletter
\DeclareCiteCommand{\fullciteallauthors}
{\defcounter{maxnames}{99}
  \usebibmacro{prenote}}
{\usedriver
  {\DeclareNameAlias{sortname}{default}}
  {\thefield{entrytype}}}
{\multicitedelim}
{\usebibmacro{postnote}}

\renewcommand*{\mkbibnamegiven}[1]{
  \ifitemannotation{highlight}
  {\textbf{#1}}
  {#1}}

\renewcommand*{\mkbibnamefamily}[1]{
  \ifitemannotation{highlight}
  {\textbf{#1}}
  {#1}}

\makeatother
% \renewcommand{\paragraph}[1]{ \vspace{.5em}\noindent\textbf{#1} }

\newcommand{\conerf}{CoNeRF\xspace}
\newcommand{\blendfields}{BlendFields\xspace}
\newcommand{\lumigauss}{LumiGauss\xspace}
\newcommand{\clog}{CLoG\xspace}

% ======== CoNeRF ======= %

% % % basic math symbols %
\usepackage{dsfont}
\newcommand{\loss}[1]{\mathcal{L}_\text{#1}}
\newcommand{\calL}{\mathcal{L}}
\newcommand{\calN}{\mathcal{N}}
\newcommand{\calF}{\mathcal{F}}
\newcommand{\expect}{\mathbb{E}}
\newcommand{\IR}{{\mathbb{R}}}
\newcommand{\IE}{{\mathbb{E}}}
\newcommand{\balpha}{{\boldsymbol{\alpha}}}
\newcommand{\bbeta}{{\boldsymbol{\beta}}}
\newcommand{\bzero}{{\mathbf{0}}}
\newcommand{\bx}{{\mathbf{x}}}
\newcommand{\bc}{{\mathbf{c}}}
\newcommand{\bd}{{\mathbf{d}}}
\newcommand{\bC}{{\mathbf{C}}}
\newcommand{\boldm}{{\mathbf{m}}}
\newcommand{\bM}{{\mathbf{M}}}
\newcommand{\br}{{\mathbf{r}}}
\newcommand{\bv}{{\mathbf{v}}}
\newcommand{\bI}{{\mathbf{I}}}
\newcommand{\beps}{{\boldsymbol{\epsilon}}}
\newcommand{\Canonicalizer}{{\mathcal{K}}}
% \newcommand{\Canonicalizer}{{\mathcal{S}}}
% \newcommand{\LifterA}{{\mathcal{G}}} %< OBSOLETE
% \newcommand{\LifterB}{{\mathcal{H}}} %< OBSOLETE
\newcommand{\hypermap}{\mathcal{H}}
\newcommand{\Representation}{{\mathcal{R}}}
\newcommand{\Attribute}{{\mathcal{A}}}
\newcommand{\AttributeSet}{{\mathfrak{A}}}
\newcommand{\MaskSet}{{\mathfrak{M}}}
\newcommand{\reglambda}[1]{\lambda_\text{#1}}
\newcommand{\pars}{\boldsymbol{\theta}}
\newcommand{\given}{;}

\newcommand{\iImage}{c}
\newcommand{\nImages}{C}
\newcommand{\image}{\mathbf{C}}
\newcommand{\images}{\{\image_\iImage\}}
\newcommand{\latents}{\{\boldsymbol\beta_\iImage\}}
\newcommand{\latent}{\boldsymbol\beta}
\newcommand{\latentDim}{B}
\newcommand{\attrib}{\boldsymbol\alpha}
\newcommand{\allpars}{\boldsymbol\theta}
\newcommand{\iFrame}{k}
\newcommand{\width}{W}
\newcommand{\height}{H}
\newcommand{\nAttributes}{A}
\newcommand{\hyperAttributeDim}{d}
\newcommand{\iAttribute}{a}
\newcommand{\attribute}{\alpha}
\newcommand{\attributes}{\boldsymbol\alpha}
\newcommand{\Mask}{\mathbf{M}}
\newcommand{\mask}{\mathbf{m}}
\newcommand{\indicator}{\delta}
\newcommand{\gt}{\text{gt}}
\newcommand{\AttributeNet}{\mathcal{A}}
\newcommand{\point}{\mathbf{x}}
\newcommand{\MaskNet}{\mathcal{M}}
\newcommand{\stopgrad}{\xcancel{\nabla}}
\newcommand{\field}{\mathbf{f}}
\newcommand{\VolRend}{\mathcal{V}}
\newcommand{\ray}{\br}

\newcommand{\CIRCLE}[1]{\raisebox{.5pt}{\footnotesize \textcircled{\raisebox{-.6pt}{#1}}}}
\newcommand{\SupplementaryMaterial}{\texttt{Supplementary}\xspace}